\chapter{三角形的中心 Triangle Centers}

% ----------------------------------- 预备知识 ------------------------------------
\section{重心坐标 Barycentric Coordinates}

\begin{theorem}
已知点$P$是$\triangle{ABC}$内任意一点, 则存在 
\[ 
  S_{\triangle{PBC}}\cdot\overline{PA} + S_{\triangle{PCA}}\cdot\overline{PB} + S_{\triangle{PAB}}\cdot\overline{PC} = \boldsymbol{0}.
\] 
\end{theorem}

\begin{figure}[!htb]
\begin{center}
\begin{tikzpicture}
  \coordinate (A) at (0,0);
  \coordinate (B) at (7,0);
  \coordinate (C) at (2,5);
  \coordinate (P) at (2.5,1);
  \draw[thick] (A) -- (B) -- (C) -- cycle;
  \draw[-latex] (P) -- (A);
  \draw[-latex] (P) -- (B);
  \draw[-latex] (P) -- (C);
  \path[intersect={B,C,P,A}] coordinate (D);
  \draw[dashed] (P) -- (D);
  \node at (A) [left] {$A$};
  \node at (B) [right] {$B$};
  \node at (C) [above] {$C$};
  \node at (P) [below] {$P$};
  \node at (D) [above right] {$D$};
  \foreach \p in {A,B,C,D,P}
    \fill[red] (\p) circle (2pt);
\end{tikzpicture}
\end{center}
\caption{重心坐标}
\end{figure}

\begin{proof}

方法 1:(面积法)

记 $S_{\triangle{PBC}},S_{\triangle{PCA}},S_{\triangle{PAB}}$ 分别为 $S_A,S_B,S_C$, 延长 $AP$ 交 $BC$ 于点$D$,
则有:
\begin{align*}
  \cfrac{BD}{DC} &= \cfrac{S_C}{S_B} \\
  \overline{PD}  &= \cfrac{DC}{BC} \cdot \overline{PB} + \cfrac{BD}{BC} \cdot \overline{PC} \\
                 &= \cfrac{S_B}{S_B+S_C} \cdot \overline{PB} + \cfrac{S_C}{S_B+S_C} \cdot \overline{PC} \\
  \cfrac{PD}{AD} &= \cfrac{S_A}{S_A+S_B+S_C} \\
  \cfrac{PD}{PA} &= \cfrac{PD}{AD-PD} \\
                 &= \cfrac{S_A}{S_B+S_C} \\
  \overline{PD}  &= -\cfrac{PD}{PA} \cdot \overline{PA} \\
                 &= -\cfrac{S_A}{S_B+S_C} \cdot \overline{PA} 
\end{align*}
\begin{align*}
  S_A \cdot \overline{PA} + S_B \cdot \overline{PA} + S_C \cdot \overline{PA} = \boldsymbol{0}
\end{align*}

方法 2:(向量法)

因为不共线的两个向量构成一个坐标系, 存在实数$a,b,c$,使得
\[
  a \cdot \overline{PA} + b \cdot \overline{PA} + c \cdot \overline{PA} = \boldsymbol{0}.
\]

\begin{align*}
  \overline{AP} &= \cfrac{b}{a} \cdot \overline{PB} + \cfrac{c}{a} \cdot \overline{PC} \\
                &= \cfrac{b}{a} \cdot (\overline{AB}-\overline{AP}) +
                   \cfrac{c}{a} \cdot (\overline{AC}-\overline{AP}) \\
                &= \cfrac{b}{a+b+c} \cdot \overline{AB} + \cfrac{c}{a+b+c} \cdot \overline{AC} \\
  \overline{AP} \times \overline{AB} &= \cfrac{c}{a+b+c} \cdot \overline{AC} \times \overline{AB} \\
  \cfrac{\|\overline{AP} \times \overline{AB}\|}{\|\overline{AC} \times \overline{AB}\|}
    &= \cfrac{S_{\triangle{PAB}}}{S_{\triangle{ABC}}} \\
    &= \cfrac{c}{a+b+c}
\end{align*}
同理, 我们可以依次得到:
\[
  S_{\triangle{PBC}}:S_{\triangle{PCA}}:S_{\triangle{PAB}} = a:b:c
\]
\end{proof}

这个结论对于点$P$在三角形的外部也是成立的, 实际上系数的正负号与三个向量的相对位置有关, 在三角形的内部三个向量依次是逆时针排列的,
如果在三角形外部, 则出现顺序颠倒, 因此系数有正有负.

对于三角形的重心, 外心, 垂心 内心 外心 可以用统一的公式形式
\footnote{\url{https://en.wikipedia.org/wiki/Triangle_center}}
\footnote{\url{https://mathworld.wolfram.com/TriangleCenter.html}}.

\begin{enumerate}

  \item 重心 (Centroid) G

  \[
    \overline{GA}+\overline{GB}+\overline{GC}=\boldsymbol{0}
  \]

  \item 外心 (Circumcenter) O

  \[
    \sin{2A} \cdot \overline{OA}+\sin{2B} \cdot \overline{OB}+\sin{2C} \cdot \overline{OC}=\boldsymbol{0}
  \]

  \item 垂心 (Orthocenter) H

  \[
    \tan{A} \cdot \overline{HA}+\tan{B} \cdot \overline{HB}+\tan{C} \cdot \overline{HC}=\boldsymbol{0}
  \]

  由于直角的正切不存在, 此时需要使用 $\tan{\theta} = \sin{\theta}/\cos{\theta}$ 进行转化.

  \item 内心 (Incenter) I

  \[
    a \cdot \overline{IA}+b \cdot \overline{IB}+c \cdot \overline{IC}=\boldsymbol{0}
  \]

  $a,b,c$是对应边长

  \item 旁心 (Excenter) J

  \[
    -a \cdot \overline{J_AA}+b \cdot \overline{J_AB}+c \cdot \overline{J_AC}=\boldsymbol{0}
  \]

  上面是顶点$A$对应的外心, 其它两个系数分别是 $(a,-b,c)$ 和 $(a,b,-c)$.

\end{enumerate}

一些重要三角形中心的重心坐标如下表
\footnote{\url{https://mathworld.wolfram.com/BarycentricCoordinates.html}} 
:

\begin{table}[H]
  \centering
  \caption{Barycentric Coordinates of Triangle Centers}
  \begin{tabular}{cc}
  \toprule
  triangle center	& barycentric coordinates \\
  \midrule
  circumcenter $O$      & $\begin{pmatrix}
                            a^2(b^2+c^2-a^2)\\
                            b^2(c^2+a^2-b^2)\\
                            c^2(a^2+b^2-c^2)
                          \end{pmatrix}$\\
  excenter $J_A$        &	$(-a,b,c)$ \\
  excenter $J_B$	      & $(a,-b,c)$ \\
  excenter $J_C$	      & $(a,b,-c)$ \\
  Gergonne point $Ge$	  & $((s-b)(s-c), (s-c)(s-a), (s-a)(s-b))$ \\
  incenter $I$          &	$(a,b,c)$ \\  
  Nagel point $Na$      &	$(s-a,s-b,s-c)$ \\
  orthocenter $H$       & $\begin{pmatrix}
                            (a^2+b^2-c^2)(c^2+a^2-b^2)\\
                            (b^2+c^2-a^2)(a^2+b^2-c^2)\\
                            (c^2+a^2-b^2)(b^2+c^2-a^2)
                          \end{pmatrix}$ \\
  symmedian point $K$   & $(a^2,b^2,c^2)$ \\
  triangle centroid $G$ &	$(1,1,1)$ \\
  nine-point center $N$ &	$\begin{pmatrix}
                            a^2(b^2+c^2)-(b^2-c^2)^2\\
                            b^2(c^2+a^2)-(c^2-a^2)^2\\
                            c^2(a^2+b^2)-(a^2-b^2)^2
                          \end{pmatrix}$ \\
  \bottomrule
  \end{tabular}
\end{table}

% ------------------------------------ 重心 -------------------------------------
\section{三角形的重心 Centroid}

\begin{definition}[三角形的重心 Centroid]
  三角形三条中线的交点
\end{definition}

{\bfseries 重心的性质}

\begin{enumerate}
  \item 重心到顶点的距离与重心到对边中点的距离之比为 2:1.
  \item 重心和三角形任意两个顶点组成的 3 个三角形面积相等. 即重心到三条边的距离与三条边的长成反比. 
  \item 重心到三角形 3 个顶点距离的平方和最小. 
  \item 在平面直角坐标系中, 重心的坐标是顶点坐标的算术平均数, 即其重心坐标为 
  \[\left(\cfrac{x_1+x_2+x_3}{3}, \cfrac{y_1+y_2+y_3}{3}\right) \]
  \item 以重心为起点, 以三角形三顶点为终点的三条向量之和等于零向量. 
\end{enumerate}

\begin{center}
\begin{tikzpicture}
  \coordinate (A) at (-2,0);
  \coordinate (B) at (4,0);
  \coordinate (C) at (-1,4);
  \coordinate (D) at ($(B)!0.5!(C)$);
  \coordinate (E) at ($(C)!0.5!(A)$);
  \coordinate (F) at ($(A)!0.5!(B)$);
  \path[centroid={A,B,C}] coordinate (G);
  \fill (G) [red] circle (2pt);
  \draw (G) node[below] {$G$};
  \draw[thick] (A) -- (B) -- (C) -- cycle;
  \draw[blue] (A) -- (D) (B) -- (E) (C) -- (F);
  \draw (A) node[left] {$A$};
  \draw (B) node[right] {$B$};
  \draw (C) node[above] {$C$};
  \draw (A) -- (B) node[near start,sloped] {$|$} node[near end,sloped] {$|$};
  \draw (B) -- (C) node[near start,sloped] {$||$} node[near end,sloped] {$||$};
  \draw (C) -- (A) node[near start,sloped] {$|||$} node[near end,sloped] {$|||$};
\end{tikzpicture}
\end{center}

% ------------------------------------ 外心 -------------------------------------
\section{三角形的外心 Circumcenter}

\begin{definition}[三角形的外心 Circumcenter]
  三角形三条边的垂直平分线
\end{definition}

{\bfseries 外心的性质}

\begin{enumerate}
  \item 当三角形为锐角三角形时, 外心在三角形内部.
  \item 当三角形为钝角三角形时, 外心在三角形外部.
  \item 当三角形为直角三角形时, 外心在斜边上, 与斜边的中点重合. 
  \item 外心到三顶点的距离相等.
\end{enumerate}

外心的坐标由下面的公式计算\footnote{https://www.cuemath.com/geometry/circumcenter/}:

\begin{align}
  \label{circumcenter-formula}
  \sin{2A} \cdot \overline{OA}+\sin{2B} \cdot \overline{OB}+\sin{2C} \cdot \overline{OC}=\boldsymbol{0}
\end{align}

令:
\begin{align}
  \boldsymbol{a} &= \overline{BC} \\
  \boldsymbol{b} &= \overline{CA} \\
  \boldsymbol{c} &= \overline{AB}
\end{align}
\begin{align}
  \boldsymbol{c} \times \boldsymbol{b} &= \|\boldsymbol{c}\| \|\boldsymbol{b}\| \sin{A} \\
  \boldsymbol{a} \times \boldsymbol{c} &= \|\boldsymbol{a}\| \|\boldsymbol{c}\| \sin{B} \\
  \boldsymbol{b} \times \boldsymbol{a} &= \|\boldsymbol{b}\| \|\boldsymbol{a}\| \sin{C} \\
  \boldsymbol{c} \cdot \boldsymbol{b} &= -\|\boldsymbol{c}\| \|\boldsymbol{b}\| \cos{A} \\
  \boldsymbol{a} \cdot \boldsymbol{c} &= -\|\boldsymbol{a}\| \|\boldsymbol{c}\| \cos{B} \\
  \boldsymbol{b} \cdot \boldsymbol{a} &= -\|\boldsymbol{b}\| \|\boldsymbol{a}\| \cos{C} 
\end{align}

由这些式子可以计算\footnote{使用 TikZ 计算作图时要注意使每一步计算的数值尽可能小, 否则出现 Dimension too large 错误.} 
$\sin{A},\cos{A},\sin{B},\cos{B},\sin{C},\cos{C}$, 
然后由倍角公式可以计算出$\sin{2A},\sin{2B},\sin{2C}$, 
代入\ref{circumcenter-formula}中,就可以得到外心的坐标.

\begin{figure}[!htb]
\begin{center}
\begin{tikzpicture}
  \coordinate (A) at (-2,0);
  \coordinate (B) at (4,0);
  \coordinate (C) at (-1,4);
  \path[circumcenter={A,B,C}] coordinate (O);
  \fill (O) [red] circle (2pt);
  \draw (O) node[below] {$O$};
  \node[draw,red] at (O) [circle through=(A)]{};
  \draw[thick] (A) -- (B) -- (C) -- cycle;
  \draw[blue] (A) -- (O) (B) -- (O) (C) -- (O);
  \draw (A) node[left] {$A$};
  \draw (B) node[right] {$B$};
  \draw (C) node[above] {$C$};
  \coordinate (D) at ($(B)!(O)!(C)$);
  \coordinate (E) at ($(C)!(O)!(A)$);
  \coordinate (F) at ($(A)!(O)!(B)$);
  \draw[red] (O) -- (D) (O) -- (E) (O) -- (F);
  \draw (A) -- (B) node[near start,sloped] {$|$} node[near end,sloped] {$|$};
  \draw (B) -- (C) node[near start,sloped] {$||$} node[near end,sloped] {$||$};
  \draw (C) -- (A) node[near start,sloped] {$|||$} node[near end,sloped] {$|||$};
  \pic [draw,red,angle radius=6pt] {right angle=O--D--C};
  \pic [draw,red,angle radius=6pt] {right angle=O--E--A};
  \pic [draw,red,angle radius=6pt] {right angle=O--F--B};
\end{tikzpicture}
\end{center}
\caption{三角形的外心}
\end{figure}

\begin{figure}[!htb]
\begin{center}
\begin{tikzpicture}[scale=1.2]
  \coordinate (A) at (0,0);
  \coordinate (B) at (3,0);
  \coordinate (C) at (0,4);
  \path[circumcenter={A,B,C}] coordinate (O);
  \fill (O) [red] circle (2pt);
  \draw (O) node[below] {$O$};
  \node[draw,red] at (O) [circle through=(A)]{};
  \draw[thick] (A) -- (B) -- (C) -- cycle;
  \draw[blue] (A) -- (O) (B) -- (O) (C) -- (O);
  \draw (A) node[left] {$A$};
  \draw (B) node[right] {$B$};
  \draw (C) node[above] {$C$};
\end{tikzpicture}
\end{center}
\caption{直角三角形的外心}
\end{figure}

\begin{figure}[!htb]
\begin{center}
\begin{tikzpicture}
  \coordinate (A) at (-3,0);
  \coordinate (B) at (0,0);
  \coordinate (C) at (60:4);
  \path[circumcenter={A,B,C}] coordinate (O);
  \fill (O) [red] circle (2pt);
  \draw (O) node[below] {$O$};
  \node[draw,red] at (O) [circle through=(A)]{};
  \draw[thick] (A) -- (B) -- (C) -- cycle;
  \draw[blue] (A) -- (O) (B) -- (O) (C) -- (O);
  \draw (A) node[left] {$A$};
  \draw (B) node[right] {$B$};
  \draw (C) node[above right] {$C$};
\end{tikzpicture}
\end{center}
\caption{钝角三角形的外心}
\end{figure}

% ------------------------------------ 垂心 -------------------------------------
\section{三角形的垂心 Orthocenter}

\begin{definition}[三角形的垂心 Orthocenter]
  三角形三条高线的交点
\end{definition}

{\bfseries 垂心的性质}

\begin{enumerate}
  \item 三角形三个顶点, 三个垂足, 垂心这7个点可以得到6个四点圆. 
  \item 三角形外心O, 重心G和垂心H三点共线\footnote{此直线称为三角形的欧拉线 Euler line}, 且$OG:GH=1:2$. 
  \item 垂心到三角形一顶点距离为此三角形外心到此顶点对边距离的2倍. 
  \item 垂心分每条高线的两部分乘积相等. 
\end{enumerate}

\begin{figure}[!htb]
\begin{center}
\begin{tikzpicture}
  \coordinate (A) at (-2,0);
  \coordinate (B) at (4,0);
  \coordinate (C) at (-1,4);
  \path[orthocenter={A,B,C}] coordinate (H);
  \fill (H) [red] circle (2pt);
  \draw (H) node[below] {$H$};
  \draw[thick] (A) -- (B) -- (C) -- cycle;
  \coordinate (D) at ($(B)!(A)!(C)$);
  \coordinate (E) at ($(A)!(B)!(C)$);
  \coordinate (F) at ($(B)!(C)!(A)$);
  \draw[blue] (A) -- (D) (B) -- (E) (C) -- (F);
  \draw (A) node[left] {$A$};
  \draw (B) node[right] {$B$};
  \draw (C) node[above] {$C$};
  \pic [draw,red,angle radius=6pt] {right angle=H--D--C};
  \pic [draw,red,angle radius=6pt] {right angle=H--E--A};
  \pic [draw,red,angle radius=6pt] {right angle=H--F--B};
\end{tikzpicture}
\end{center}
\caption{三角形的垂心}
\end{figure}


\begin{figure}[!htb]
\begin{center}
\begin{tikzpicture}
  \coordinate (A) at (0,0);
  \coordinate (B) at (3,0);
  \coordinate (C) at (0,4);
  \path[orthocenter={A,B,C}] coordinate (H);
  \fill (H) [red] circle (2pt);
  \draw (H) node[below] {$H$};
  \draw[thick] (A) -- (B) -- (C) -- cycle;
  \coordinate (D) at ($(B)!(A)!(C)$);
  \coordinate (E) at ($(A)!(B)!(C)$);
  \coordinate (F) at ($(B)!(C)!(A)$);
  \draw[blue] (A) -- (D) (B) -- (E) (C) -- (F);
  \draw (A) node[left] {$A$};
  \draw (B) node[right] {$B$};
  \draw (C) node[above] {$C$};
\end{tikzpicture}
\end{center}
\caption{直角三角形的垂心}
\end{figure}

\begin{figure}[!htb]
\begin{center}
\begin{tikzpicture}
  \coordinate (A) at (-3,0);
  \coordinate (B) at (0,0);
  \coordinate (C) at (60:4);
  \path[orthocenter={A,B,C}] coordinate (H);
  \fill (H) [red] circle (2pt);
  \draw (H) node[below] {$H$};
  \draw[thick] (A) -- (B) -- (C) -- cycle;
  \coordinate (D) at ($(B)!(A)!(C)$);
  \coordinate (E) at ($(A)!(B)!(C)$);
  \coordinate (F) at ($(B)!(C)!(A)$);
  \draw[blue] (H) -- (A) (H) -- (B) (H) -- (C);
  \draw (A) node[left] {$A$};
  \draw (B) node[right] {$B$};
  \draw (C) node[above] {$C$};
\end{tikzpicture}
\end{center}
\caption{钝角三角形的垂心}
\end{figure}

% ------------------------------------ 内心 -------------------------------------
\section{三角形的内心 Incenter}

\begin{definition}[三角形的内心 Incenter]
  三角形三条角线的交点
\end{definition}

{\bfseries 内心的性质}

\begin{enumerate}
  \item 直角三角形的内心到边的距离等于两直角边的和与斜边的差的二分之一. 
  \item 欧拉定理$\triangle ABC$中, $R$和$r$分别为外接圆为和内切圆的半径, $O$和$I$分别为其外心和内心, 则$OI^2=R^2-2Rr$.
  \item 内心到三角形三边距离相等. 
\end{enumerate}

\begin{figure}[!htb]
\begin{center}
\begin{tikzpicture}
  \coordinate (A) at (-2,0);
  \coordinate (B) at (4,0);
  \coordinate (C) at (-1,4);
  \path[incenter={A,B,C}] coordinate (I);
  \fill (I) [red] circle (2pt);
  \draw (I) node[below] {$I$};
  \node[draw,red] at (I) [circle through=($(B)!(I)!(C)$)]{};
  \draw[thick] (A) -- (B) -- (C) -- cycle;
  \draw (A) node[left] {$A$};
  \draw (B) node[right] {$B$};
  \draw (C) node[above] {$C$};
  \draw (A) -- (I) (B) -- (I) (C) -- (I);
  \pic [draw,angle radius=12pt] {angle=I--A--C};
  \pic [draw,angle radius=15pt] {angle=B--A--I};
  \pic [draw,double,angle radius=12pt] {angle=A--C--I};
  \pic [draw,double,angle radius=15pt] {angle=I--C--B};
  \pic [draw,pic text=.,angle radius=12pt, 
    angle eccentricity=1.2] {angle=C--B--I};
  \pic [draw,pic text=.,angle radius=15pt, 
    angle eccentricity=1.2] {angle=I--B--A};
\end{tikzpicture}
\end{center}
\caption{三角形的内心}
\end{figure}


% ------------------------------------ 旁心 -------------------------------------
\section{三角形的旁心 Excenter}

\begin{definition}[三角形的旁心 Excenter]
  三角形一条内角平分线与两条外角平分线的交点, 共 3 个.
\end{definition}

{\bfseries 旁心的性质}

\begin{enumerate}
  \item 旁心到三边的距离相等. 
\end{enumerate}

\begin{figure}[!htp]
\begin{center}
\begin{tikzpicture}[scale=.5]
  \coordinate (A) at (-2,0);
  \coordinate (B) at (4,0);
  \coordinate (C) at (-1,4);
  \path[incenter={A,B,C}] coordinate (I);
  \path[excenter={A,B,C}] coordinate (JA);
  \path[excenter={B,A,C}] coordinate (JB);
  \path[excenter={C,A,B}] coordinate (JC);
  \foreach \point in {I,JA,JB,JC}
    \fill (\point) [red] circle (2pt);
  %\draw (\point) node[below] {$E$};
  \node[draw,red] at (I) [circle through=($(B)!(I)!(C)$)]{};
  \node[draw,red] at (JA) [circle through=($(B)!(JA)!(C)$)]{};
  \node[draw,red] at (JB) [circle through=($(B)!(JB)!(C)$)]{};
  \node[draw,red] at (JC) [circle through=($(B)!(JC)!(C)$)]{};
  \draw[thick] (A) -- (B) -- (C) -- cycle;
  \draw (A) node[left] {$A$};
  \draw (B) node[right] {$B$};
  \draw (C) node[above] {$C$};
  \draw[blue] ($(A)!-1!(B)$) -- ($(A)!2!(B)$);
  \draw[blue] ($(B)!-1!(C)$) -- ($(B)!2!(C)$);
  \draw[blue] ($(C)!-1!(A)$) -- ($(C)!2!(A)$);
  \draw (I) node[right] {$I$};
  \draw (JA) node[right] {$J_A$};
  \draw (JB) node[right] {$J_B$};
  \draw (JC) node[right] {$J_C$};
\end{tikzpicture}
\end{center}
\caption{三角形的内切圆与旁切圆}
\end{figure}

\section{欧拉定理}

\begin{theorem}[欧拉定理,Euler's theorem in geometry]
  三角形的外心与内心之间的距离$d$ 可表示为
  \[d^2=R(R-2r)\]
  或
  \[\cfrac{1}{R-d} + \cfrac{1}{R+d}=\cfrac{1}{r}\]
  其中:$R$为外接圆半径,$r$为内切圆半径.
\end{theorem}

从欧拉定理可推出欧拉不等式(Euler's inequality)(仅当三角形为等边三角形时,等号成立):
\[R \geqslant 2r\]

\section{欧拉线 Euler line}

\begin{theorem}
  三角形的垂心,外心,重心和九点圆圆心共线.
\end{theorem}
这条直线称为欧拉线(Euler line). 莱昂哈德·欧拉(Leonhard Euler)证明了在任意三角形中,
以上四点共线. 欧拉线上的四点中,
九点圆圆心到垂心和外心的距离相等,
而且重心到外心的距离是重心到垂心距离的一半.
注意内心一般不在欧拉线上,除了等腰三角形外.

\begin{figure}[!htp]
\begin{center}
\begin{tikzpicture}
  \coordinate (A) at (-2,0);
  \coordinate (B) at (4,0);
  \coordinate (C) at (-1,4);
  \path[orthocenter={A,B,C}] coordinate (H);
  \path[circumcenter={A,B,C}] coordinate (O);
  \path[centroid={A,B,C}] coordinate (G);
  \path[incenter={A,B,C}] coordinate (I);
  \path[nine-point center={A,B,C}] coordinate (N);
  \foreach \point in {A,B,C,H,O,G,I,N}
    \fill (\point) [red] circle (2pt);;
  \draw[thick] (A) -- (B) -- (C) -- cycle;
  \draw[thick,cyan] (H) -- (O) -- (G);
  \draw (A) node[left] {$A$};
  \draw (B) node[right] {$B$};
  \draw (C) node[above] {$C$};
  \draw (H) node[below] {$H$};
  \draw (O) node[below] {$O$};
  \draw (G) node[below] {$G$};
  \draw (I) node[above] {$I$};
  \draw (N) node[below] {$N$};
\end{tikzpicture}
\end{center}
\caption{欧拉线}
\end{figure}

\section{九点圆定理 Nine-point circle theorem}

\begin{definition}[欧拉点,Euler points]
  在三角形中,顶点到垂心的三条线段的中点.
\end{definition}
三个欧拉点构成的三角形称为欧拉三角形.

\begin{theorem}[九点圆定理,Nine-point circle theorem]
  对任意三角形,其三边的中点,三高的垂足,三个欧拉点九点共圆.
\end{theorem}

这个圆被称为九点圆(Nine-point circle), 又称欧拉圆(Euler's circle), 或费尔巴哈圆(Feuerbach's circle).

九点圆具有以下性质:

\begin{enumerate}
  \item 九点圆的半径是外接圆的一半,且九点圆平分垂心与外接圆上的任一点的连线.
  \item 圆心在欧拉线上,且在垂心到外心的线段的中点.
  \item 九点圆和三角形的内切圆和旁切圆相切(\hyperref[feuerbach]{费尔巴哈定理}).
  \item 圆周上四点任取三点做三角形,四个三角形的九点圆圆心共圆(库利奇-大上定理).
\end{enumerate}

\begin{figure}[!htp]
\begin{center}
\begin{tikzpicture}
  \coordinate (A) at (-2,0);
  \coordinate (B) at (4,0);
  \coordinate (C) at (-1,4);
  \path[orthocenter={A,B,C}] coordinate (H);
  
  \coordinate (E1) at ($(A)!.5!(H)$);
  \coordinate (E2) at ($(B)!.5!(H)$);
  \coordinate (E3) at ($(C)!.5!(H)$);
  \coordinate (F1) at ($(B)!(A)!(C)$);
  \coordinate (F2) at ($(C)!(B)!(A)$);
  \coordinate (F3) at ($(A)!(C)!(B)$);
  \coordinate (M1) at ($(A)!.5!(B)$);
  \coordinate (M2) at ($(B)!.5!(C)$);
  \coordinate (M3) at ($(C)!.5!(A)$);
  \coordinate[nine-point center={A,B,C}] (N);
  \foreach \point in {A,B,C,E1,E2,E3,F1,F2,F3,M1,M2,M3,N}
    \fill (\point) [red] circle (2pt);
  \draw[thick] (A) -- (B) -- (C) -- cycle;
  \draw[thick,magenta,circle={N,E1}];
  \draw[blue] (A) -- (F1) (B) -- (F2) (C) -- (F3);
  \draw (A) node[left] {$A$};
  \draw (B) node[right] {$B$};
  \draw (C) node[above] {$C$};
  \draw (N) node[right] {$N$};
\end{tikzpicture}
\end{center}
\caption{九点圆}
\end{figure}

\section{费尔巴哈定理 Feuerbach's theorem}

\begin{theorem}[费尔巴哈定理,Feuerbach's theorem\label{feuerbach}]
  九点圆和三角形的内切圆和旁切圆相切.
\end{theorem}

\begin{figure}[!htp]
\begin{center}
\begin{tikzpicture}[scale=.5]
  \coordinate (A) at (-2,0);
  \coordinate (B) at (4,0);
  \coordinate (C) at (-1,4);
  \path[incenter={A,B,C}] coordinate (I);
  \path[excenter={A,B,C}] coordinate (JA);
  \path[excenter={B,A,C}] coordinate (JB);
  \path[excenter={C,A,B}] coordinate (JC);
  \path[nine-point center={A,B,C}] coordinate(N);
  \foreach \point in {I,JA,JB,JC}
    \fill (\point) [red] circle (2pt);
  \node[draw,red] at (I) [circle through=($(B)!(I)!(C)$)]{};
  \node[draw,red] at (JA) [circle through=($(B)!(JA)!(C)$)]{};
  \node[draw,red] at (JB) [circle through=($(B)!(JB)!(C)$)]{};
  \node[draw,red] at (JC) [circle through=($(B)!(JC)!(C)$)]{};
  \node[draw,cyan,thick] at (N) [circle through=($(A)!.5!(B)$)]{};
  \draw[thick] (A) -- (B) -- (C) -- cycle;
  \draw (A) node[left] {$A$};
  \draw (B) node[right] {$B$};
  \draw (C) node[above] {$C$};
  \draw[blue] ($(A)!-1!(B)$) -- ($(A)!2!(B)$);
  \draw[blue] ($(B)!-1!(C)$) -- ($(B)!2!(C)$);
  \draw[blue] ($(C)!-1!(A)$) -- ($(C)!2!(A)$);
  \draw (I) node[right] {$I$};
  \draw (JA) node[right] {$J_A$};
  \draw (JB) node[right] {$J_B$};
  \draw (JC) node[right] {$J_C$};
  \draw (N) node[below] {$N$};
\end{tikzpicture}
\end{center}
\caption{费尔巴哈定理}
\end{figure}

\section{费马点 Fermat point}

\begin{definition}[费马点,Fermat point]
  三角形内部满足到三个顶点距离之和最小的点.
\end{definition}

费马点问题最早是由法国数学家皮埃尔·德·费马(Pierre de Fermat)在一封写给意大利数学家埃万杰利斯塔·托里拆利(Evangelista Torricelli 气压计的发明者)的信中提出的. 
托里拆利最早解决了这个问题, 而19世纪的数学家斯坦纳(Jakob Steiner)重新发现了这个问题,
并系统地进行了推广,因此这个点也称为托里拆利点或斯坦纳点,
相关的问题也被称作费马-托里拆利-斯坦纳问题.

下面是三角形的费马点的作法:

\begin{enumerate}
  \item 当有一个内角不小于120°时, 费马点为此角对应顶点.
  \item 当三角形的内角都小于120°时
  \begin{itemize}
    \item 以三角形的每一边为底边, 向外做三个正三角形$\triangle ABC', \triangle BCA', \triangle CAB'$.
    \item 连接$AA',BB',CC'$, 则三条线段的交点就是所求的点.
  \end{itemize}
\end{enumerate}

\begin{figure}[!htp]
\begin{center}
\begin{tikzpicture}
  \coordinate (A) at (-1,4);
  \coordinate (B) at (-2,0);
  \coordinate (C) at (4,0);
  \coordinate[revolve/angle=60,revolve={C,B}] (A'); % B 绕 C 旋转 60 度
  \coordinate[revolve/angle=60,revolve={A,C}] (B');
  \coordinate[revolve/angle=60,revolve={B,A}] (C');
  \coordinate[intersect={A,A',B,B'}] (P);
  \coordinate[revolve/angle=-60,revolve={P,C}] (Q);
  \coordinate (P') at (2,1); % 任意一点
  \coordinate[revolve/angle=60,revolve={C,P'}] (Q');
  \foreach \point in {A,B,C,A',B',C',P,Q,P',Q'}
    \fill (\point) [red] circle (2pt);
  \draw[thick,red] (A) -- (A') (B) -- (B') (C) -- (C');
  \draw[magenta] (Q) -- (C);
  \draw[thick] (A) -- (B) -- (C') -- cycle;
  \draw[thick] (B) -- (C) -- (A') -- cycle;
  \draw[thick] (C) -- (A) -- (B') -- cycle;
  \draw[blue] (A) -- (P') -- (Q') -- (A') 
    (P') -- (C) (Q') -- (C) (P') -- (B);
  \draw (A) node[above] {$A$};
  \draw (B) node[below left] {$B$};
  \draw (C) node[below right] {$C$};
  \draw (A') node[below] {$A'$};
  \draw (B') node[above right] {$B'$};
  \draw (C') node[above left] {$C'$};
  \draw (P) node[left] {$P$};
  \draw (Q) node[left] {$Q$};
  \draw (P') node[right] {$P'$};
  \draw (Q') node[left] {$Q'$};
\end{tikzpicture}
\end{center}
\caption{费马点(三个内角均小于$120^\circ$)}
\label{fermat:case1}
\end{figure}

\begin{figure}[!htp]
\begin{center}
\begin{tikzpicture}
  \coordinate (A) at (3,1);
  \coordinate (B) at (-1,0);
  \coordinate (C) at ($(A)!0.5!120:(B)$);
  \coordinate (D) at ($(A)!0.75!-120:(B)$);
  \coordinate (P) at (2,0.5); % 任意一点
  \foreach \point in {A,B,C,D,P}
    \fill (\point) [red] circle (2pt);
  \draw[thick] (A) -- (B) -- (C) -- cycle;
  \draw[blue] (B) -- (D) (C) -- (D) (P) -- (A);
  \draw[red] (A) -- (D);
  \draw[teal] (P) -- (C) (P) -- (B) (P) -- (D);
  \draw (A) node[above] {$A$};
  \draw (B) node[below left] {$B$};
  \draw (C) node[below right] {$C$};
  \draw (D) node[above] {$D$};
  \draw (P) node[below] {$P$};
\end{tikzpicture}
\end{center}
\caption{费马点($\angle BAC = 120^\circ$)}
\label{fermat:case2}
\end{figure}

\begin{figure}[!htp]
\begin{center}
\begin{tikzpicture}
  \coordinate (A) at (3,1);
  \coordinate (B) at (-1,0);
  \coordinate (C) at ($(A)!0.5!140:(B)$);
  \coordinate (D) at ($(A)!0.5!120:(B)$);
  \coordinate (P) at (2,0.5); % 任意一点
  \coordinate[intersect={A,D,P,C}] (D);
  \foreach \point in {A,B,C,D,P}
    \fill (\point) [red] circle (2pt);
  \draw[thick] (A) -- (B) -- (C) -- cycle;
  \draw[red] (A) -- (D);
  \draw[teal] (P) -- (C) (P) -- (B) (P) -- (A);
  \draw (A) node[above] {$A$};
  \draw (B) node[below left] {$B$};
  \draw (C) node[below right] {$C$};
  \draw (D) node[above] {$D$};
  \draw (P) node[below] {$P$};
\end{tikzpicture}
\end{center}
\caption{费马点($\angle BAC > 120^\circ$)}
\label{fermat:case3}
\end{figure}

\begin{proof}
  分三种情况证明.
\begin{enumerate}
  \item 三角形的内角都小于120°的情况:
  
  如果\ref{fermat:case1}, 首先证明$AA',BB',CC'$三条线交于一点.
  设$P$为线段$BB'$和$CC'$的交点,
  注意到 $\triangle C'AC \cong \triangle BAB'$, 
  $\triangle C'AC$可以看做是$\triangle B'AB$ 
  以$A$点为轴心顺时针旋转$60^\circ$得到的,
  所以$\angle AC'P = \angle ABP$, $A,C',B,P$四点共圆, 
  $\angle APC' = \angle C'PB = 60^\circ$.
  同样地,可以证明其它以$P$为顶点的角也为$60^\circ$. 
  因此, $AA',BB',CC'$三条线交于一点.
  
  接下来证明交点P就是到三个顶点距离之和最小的点.
  在线段$AA'$上选择一点$Q$,
  使得$QP = PC$,
  易知等腰三角形$PQC$是正三角形,
  $\triangle BPC \cong \triangle A'QC$.
  所以$QA' = PB$.
  综上可得出: $PA + PB + PC = AA'$

  对于平面上另外一个点$P'$,
  以$P'C$为底边,
  向下作正三角形$P'Q'C$,
  运用类似以上的推理可以证明$\triangle BP'C \cong \triangle A'Q'C$.
  因此也有 $P'A + P'B + P'C = AP' + P'Q' + Q'A$.
  平面上两点之间以直线长度最短, 因此
  \[P'A + P'B + P'C = AP' + P'Q' + Q'A' ≥ AA' = PA + PB + PC\]
  也就是说, 点$P$是平面上到点$A,B,C$距离的和最短的一点.
  
  最后证明唯一性. 按照前面的论述, $P$点在$AA'$上,
  同理可以证明$P$也在$BB',CC'$上.
  因此$P$也是$AA',BB',CC'$三条线的交点.
  因此点P是唯一的.
  
  \item 有一内角等于120°的情况:
  
  如图\ref{fermat:case2}, $\angle BAC = 120^\circ$,
  在 $\triangle ABC$ 外上作$\angle BAD = 120^\circ$,
  连接$BD,CD$,则$A$为 $\triangle BCD$的费马点.于是:
  $AD+AB+AC \leqslant PD+PB+PC$, 即$AB+AC \leqslant PD-AD+PB+PC \leqslant PA+PB+PC $. 
  
  \item 有一内角大于120°的情况:
  \item 
  如图\ref{fermat:case3}, $\angle BAC > 120^\circ$, 在 $\angle BAC$内做$\angle BAD = 120^\circ$ 交 $PC$于 $D$.
  有(2)结论, 
  \begin{align*}
    PA+PB+PD &> AB + AD \\
    PA+PB+PC &= PA+PB+PD+DC \\
             &> AB + AD + DC \\
             &> AB + AC
  \end{align*}
\end{enumerate}
\end{proof}

\section{莱斯特定理 Lester's's theorem}

\begin{definition}[第一费马点或正费马点, the first Fermat point or the positive Fermat point]
  以$AB,BC,AC$为边向外作等边三角形,对应点连线三线交于一点$P$,即为第一费马点.
\end{definition}

\begin{definition}[第二费马点或负费马点, the second Fermat point or the negative Fermat point]
  以$AB,BC,AC$为边向内作等边三角形,对应点连线三线交于一点$P$,即为第二费马点.
\end{definition}

\begin{theorem}[莱斯特定理,Lester's's theorem\label{lester}]
  在任意不等边三角形中(scalene triangle), 第一费马点,第二费马点,九点圆圆心和外心四点共圆.
\end{theorem}

\begin{figure}[!htp]
\begin{center}
\begin{tikzpicture}
  \coordinate (A) at (-1,4);
  \coordinate (B) at (-2,0);
  \coordinate (C) at (4,0);
  \coordinate[revolve/angle=60,revolve={A,C}] (B1); 
  \coordinate[revolve/angle=60,revolve={B,A}] (C1);
  \coordinate[intersect={B,B1,C,C1}] (F1);
  \coordinate[revolve/angle=-60,revolve={A,C}] (B2);
  \coordinate[revolve/angle=-60,revolve={B,A}] (C2);
  \coordinate[intersect={B,B2,C,C2}] (F2);
  \coordinate[circumcenter={A,B,C}] (O);
  \coordinate[nine-point center={A,B,C}] (N);
  \coordinate[circumcenter={F1,F2,O}] (P);
  \draw[purple,thick,circle={P,O}];
  \foreach \p in {A,B,C,F1,F2,O,N}
    \fill[red] (\p) circle (2pt);
  \draw[thick] (A) -- (B) -- (C) -- cycle;
  \draw (A) node[above] {$A$};
  \draw (B) node[below left] {$B$};
  \draw (C) node[below right] {$C$};
  \draw (F1) node[left] {$F_1$};
  \draw (F2) node[below] {$F_2$};
  \draw (O) node[below] {$O$};
  \draw (N) node[below] {$N$};
\end{tikzpicture}
\end{center}
\caption{莱斯特定理}
\end{figure}
