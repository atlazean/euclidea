\chapter{通径 Latus Rectum}\label{ch:latus-rectum}

\textbf{通径 (latus rectum)} 亦称"正通径","首通径","直焦弦","主焦弦","正焦弦". 
过圆锥曲线的焦点且与过焦点的轴垂直的弦称为通径, 清代明安图《割环密率捷法》中, 称圆的直径为通径. 

The distance $p$ from the focus to the conic section directrix of a conic section. 
The following table gives the focal parameter for the different types of conics, 
where $a$ is the semimajor axis, $b$ is the semiminor axis, $c$ is the distances from the origin to the focus, 
and $e$ is the eccentricity.

\begin{table}[H]
  \centering
  \caption{Focal Parameters of Conics}
  \rowcolors{2}{cyan!25}{white} % even odd
  \begin{tabular}{p{.2\textwidth}|m{.2\textwidth}|m{.5\textwidth}}
    \toprule
    \rowcolor{cyan!40}
     Conic & $e$ & $p$ \\
    \midrule
    Ellipse & $0 < e <1$ & $p=\dfrac{b^2}{c}$ \\
    Parabola & $e = 1$ & $p=2a$ \\
    Hyperbola & $e > 1$ & $p=\dfrac{b^2}{c}$ \\
    \bottomrule
  \end{tabular}
\end{table}

\section{椭圆的通径与焦准距}

连接椭圆上任意两点的线段叫作这个\textbf{椭圆的弦}, 
通过焦点的弦叫作这个椭圆的\textbf{焦点弦}(所以椭圆的长轴也是焦点弦), 
和长轴垂直的焦点弦叫作这个椭圆的\textbf{通径(正焦弦)}. 
连接椭圆上任意一点与一个焦点的线段(或这线段的长)叫作椭圆在这点的\textbf{焦半径}, 
椭圆上任意一点有两条焦半径. 

设椭圆的方程为
\[
  \dfrac{x^2}{a^2}+\dfrac{y^2}{b^2}=1,(a > b > 0)
\]

其通径的长为$\dfrac{2b^2}{a}$, 或$2ep$,
其中: $a$为\textbf{半长轴长}, $b$为\textbf{半短轴长}, $e$为椭圆的\textbf{离心率}, $p$为椭圆的\textbf{焦准距}. 

焦点与相应准线的距离称为椭圆的\textbf{焦准距}, 也叫\textbf{焦参数}. 

\[
  p=\dfrac{a^2}{c}-c=\dfrac{a^2-c^2}{c}=\dfrac{b^2}{c}
\]

\section{抛物线的通径与焦准距}

经过抛物线的焦点, 作一条垂直于它的对称轴的直线, 这直线与抛物线有两个交点, 这两个交点之间的线段叫做\textbf{抛物线的通径}. 

抛物线的方程:

\[
  y^2=2px
\]

抛物线的通径长为$2p$, 其中: 抛物线的\textbf{焦准距}长为$p$. 

\section{双曲线的通径与焦准距}

过双曲线的焦点与双曲线的实轴垂直的直线被双曲线截得的线段的长, 称为\textbf{双曲线的通径}. 

\[
  \dfrac{x^2}{a^2}+\dfrac{y^2}{b^2}=1,(a >0, b > 0)
\]

其通径长 $\dfrac{2b^2}{a}$, 或 $2ep$, 
其中: $a$为\textbf{半实轴长}, $b$为\textbf{半虚轴长}, $e$为双曲线的\textbf{离心率}, $p$为双曲线的\textbf{焦准距}. 
