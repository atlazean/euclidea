% Braikenridge–Maclaurin Construction
% the converse to Pascal's theorem
\begin{tikzpicture}
  \clip (-6,-6) rectangle (6,6);
  \axes(-5:5,-5:5);

  % 为方便起见, 使用了椭圆参数方程来生成 5 个点
  \tikzmath{
    \a = 4;
    \b = 3;
  }
  \foreach \i/\angle in {1/110,2/240,3/330,4/170,5/45}{
    \coordinate (A\i) at ({\a*cos(\angle)},{\b*sin(\angle)});
  }
  
  % 根据 Pascal 定理求得一对对边的交点 L
  \coordinate[intersect={A1,A2,A4,A5}] (L);

  % 过 L 作任意直线 a, 这里使用了单位向量来构造任意直线
  % 求直线 a 与 A2A3 的交点 M
  % 求直线 a 与 A3A4 的交点 N
  % 第 6 点 A6 是 A5M 与 A1N 的交点
  \foreach \angle in {0,5,...,360}{
    \coordinate (U) at ($(L)+(\angle:1)$);
    \coordinate[intersect={A2,A3,L,U}] (M);
    \coordinate[intersect={A3,A4,L,U}] (N);
    \coordinate[intersect={A5,M,A1,N}] (A6);
    \fill[teal] (A6) circle (2pt);

    % 绘制中间过程
    \ifthenelse{\angle=315}{%if-part
      \draw[thick,purple,start modifier=-1cm,end modifier=5.5cm,extend={L,M}];
      \draw (A1) -- (A2) -- (A3) -- (A4) -- (A5) --(A6) -- cycle;
      \draw (A6) node[below] {$A6$};
      \foreach \p/\placement in {L/below left,M/above,N/below left}{
        \fill[green] (\p) circle (2pt);
        \draw (\p) node[\placement] {$\p$};
      }
    }{%else-part
    }
  }

  \tikzset {
    conic/five points={A1,A2,A3,A4,A5},
  }
  \draw [purple,thick,conic];

  \foreach \i/\placement in {1/above,2/below,3/below right,4/left,5/above right}{
    \fill[red] (A\i) circle (2pt);
    \draw (A\i) node[\placement] {$A\i$};
  }
\end{tikzpicture}