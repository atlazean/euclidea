\begin{tikzpicture}
  \axes(-5:5, -5:5);

  \coordinate (F) at (0,1);% focus
  \coordinate (V) at (0.5,0);% vertex
  \coordinate (P) at (2,-2);

  \tikzset{
    % parabola/doamin=-1.25:1.25,
    parabola/define={F,V},
  }

  \draw [purple, thick, parabola] node [above] {parabola};
  \draw [teal, thick, 
  % parabola/directrix/scale=1.5,
  parabola/directrix] node [below] {directrix};
  \draw [magenta, thick, dashdotted, 
  % parabola/axis/scale=2.5,
  parabola/axis] node [right] {axis};

  % 求准线上的两点 A, B
  \coordinate[affine={F,V,2}] (A);
  \coordinate[revolve/angle=90,revolve={A,F}] (B);

  % 作以 P 为圆心, PF 为半径的圆
  \draw[red,circle={P,F}];
  
  % 求圆与直线的交点(存储在 cl1 和 cl2 中)
  \tikzset{circle-line={P,F,A,B},}
  \coordinate (Q) at (cl1);
  \coordinate (R) at (cl2);

  % 作出 F2Q, F2R 的中垂线即为椭圆的切线
  \draw[violet,thick,end modifier=7.5cm,perpendicular bisector={F, Q}];
  \draw[violet,thick,end modifier=6.5cm,perpendicular bisector={F, R}];

  \foreach \p/\placement in {F/left,
  P/above right,Q/above,R/above left}{
    \fill[red] (\p) circle (2pt);
    \draw (\p) node[\placement] {$\p$};
  }

\end{tikzpicture}