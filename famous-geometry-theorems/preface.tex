\chapter{序言}

在铉儿约四岁的时候,在一次剪纸游戏过程中问我:“剪刀是直的,为什么能剪出圆?”
我思考了良久,不得不用微积分的知识向他解释:因为曲线可以用小的线段来近似。
看着他似懂非懂的表情,我心中惊讶小孩子能有这么敏锐的观察力。
在铉儿上小学时候,我们又谈论起这个问题,他解释到,剪刀和纸只有一点接触,
而点是可以按曲线运动的。至此,这个问题才有一个比较通俗易懂的解释。

几何学(geometry)最开始来源于人类生产活动的需要(如土地测量),逐渐发展成数学的重要分支。
古希腊柏拉图学院门口挂着“不懂几何者,不得入内”。 
对古希腊尺规作图三大难题(化圆为方、三等分角和倍立方问题)的探索一直是推动数学发展的重要动力之一。
有的研究已经远远超越当时的需要,
如 Apollonius 对圆锥曲线的研究。
本书列举了一些欧几里得几何(Euclidean Geometry)的重要定理(主要来源\cite{wikipedia,mathworld,mathsgreat}),
当我们看到这些几何定理时,一定会为其所表现出来的简单与和谐而感到震撼。

在学习的道路上,要保持兴趣,保持观察力。几何学,其实和其它的学科一样,
会为你开启一扇窗户,看到不同的风景。