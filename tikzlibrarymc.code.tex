\ProvidesFile{tikzlibrarymc.code.tex}[2025/03/28 v1.2.1 A tikz library for matrix computations and linear algebra]

% 提高数值计算精度
% https://tex.stackexchange.com/questions/521857/tikz-fpu-seems-to-be-inaccurate
\RequirePackage{xfp}

\usetikzlibrary{math,calc,quotes}

% ===============================================
% Caveats
% ===============================================
% 1. Tikzmath is mainly used for handling numerical computations and variable 
% assignments, supporting basic mathematical operations and conditional 
% statements, but it does not support arrays or lists as function parameters.
% 2. Tikzmath internally requires semicolons for separation, allows comments, 
% but cannot have blank lines.
% 3. The for statement does not have loop interruption commands like 'break' or 'continue'.
% 4. The first iteration of a for loop is always executed without evaluating the 
% loop variable; to restrict the range of the loop variable, it must be 
% explicitly limited.
% 5. Integer "subscripts" (not true subscripts) must be defined as int; otherwise, 
% they default to floating-point numbers, and defined macros cannot be found.
% 6. Changes to variables inside a function do not propagate to the outside.
% 7. When defining functions, there should be no spaces between parameters.
% 8. The return statement does not interrupt the execution of subsequent 
% statements within the function.
% 9. The maximum number of function parameters is 9.
% 10.Function parameters cannot directly use arrays.
% 11.Extremely difficult to debug.
% 12.Given the limitations of tikzmath’s function, you can use tikzset 
% with handlers(.code) to execute some code.
% 13.Call tikzmath within the code, but note that this may generate 
% some global variables; to avoid conflicts with user-defined variables, 
% add the prefix mc@.
% 14.Modifications of variables inside a foreach statement will vanish outside.
% 15.Special attention should be paid to variables in tikzmath, as these 
% variables are global. If they are used directly without declaration, 
% it may pose risks. For example, \mc@c might be declared as an int in one 
% place, but if it is assigned a floating-point number elsewhere without 
% declaration, it will automatically be converted to an integer.
%
% 1. tikzmath 主要用于处理数值计算和变量赋值, 支持基本的数学运算和条件语句,
% 但不支持数组或列表作为函数参数.
% 2. tikzmath 内部要以分号分隔, 可以有注释, 但不能有空行.
% 3. for 语句没有类似 break, continue 之类的中断循环的语句.
% 4. for 的第一次总是执行的, 不对循环变量作判断, 要限制循环变量的范围.
% 5. 整数"下标"(非真正下标), 必须定义为 int, 否则默认为浮点数, 无法找到定义的宏.
% 6. 函数内部对变量的修改不会传导至外部.
% 7. 函数定义时, 参数之间不要有空格.
% 8. return 语句不会中断函数内部下面语句的执行.
% 9. 函数参数个数最大为 9.
% 10.函数参数不能直接使用数组.
% 11.极难调试.
% 12.鉴于 tikzmath 的 function 存在一些缺陷, 可以使用 
% tikzset 的 /.code 来执行一些代码
% 13.在代码中调用 tikzmath, 但是注意这里会产生一些全局的变量, 
% 为了避免与用户的变量冲突, 添加前缀 mc@
% 14.foreach 中的修改在循环外面是失效的
% https://tex.stackexchange.com/questions/196065/pgfkeys-computed-i-e-dynamically-defined-key
% 15.要特别注意 tikzmath 中的变量, 这些变量是全局的, 如果不加声明就直接使用
% 可能会带来风险, 如 \mc@c 可能在某处被声明为 int, 在另一处不声明就赋值浮点数时则会
% 自动转换为整数, 
% 16.此外重名可能导致一些计算结果被误修改
% 17.当.code args 或 .style args 中的参数本身是一个算式时, 必须加括号以保证计算的正确性.

\makeatletter

\tikzmath {
  % threshold, numbers not more than epsilon will be treated as zero.
  \EPSILON = 0.00001;
  % Jacobi maximum number of iterations, default: 3*(n-1)^2
  % wherein, n is size of the symetric matrix, 
  % User can override it with a positive integer
  % -1 means using the default value
  \JacobiMaxIter = -1; 
  % Jacobi precision, too small a value can cause iteration oscillation
  \JacobiEpsilon = 0.00005;
  % status of solving linear systems, inversing matrices, and
  % Jacobi eigenvalue algorithm
  % 1 for success and 0 for failure
  \status = 0;
  % Determinant calculation using Gaussian elimination
  % In order to calculate determinant, user should initialize the matrix 
  % \mc@mat, and then call this function. This function is reused in 
  % calcuating minors of a matrix.
  function det(\n) {
    \result = 1.0;
    int \i, \j, \k;
    int \stop;
    \stop = 0;
    % for 的第一项总是执行的,要限制范围
    for \k in {1,...,\n-1} {
      if \stop == 0 then {
        \pivotRow = \k;
        \pivot = abs(\mc@mat{\k,\k});
        for \i in {\k+1,...,\n} {
          if \pivot < abs(\mc@mat{\i,\k}) then {
            \pivotRow = \i;
            \pivot = abs(\mc@mat{\i,\k});
          };
        };
        if \pivot < \EPSILON then { 
          \stop = 1;
          \result = 0.0;
        };
        if \stop == 0 then {
          % swap
          if \pivotRow != \k then {
            \result = -\result;
            for \j in {\k,...,\n}{
              \temp = \mc@mat{\k,\j};
              \mc@mat{\k,\j} = \mc@mat{\pivotRow,\j};
              \mc@mat{\pivotRow,\j} = \temp;
            };
          };
          % eliminate
          for \i in {\k+1,...,\n}{
            \factor = \fpeval{\mc@mat{\i,\k} / \mc@mat{\k,\k}};
            for \j in {\k,...,\n}{
              \mc@mat{\i,\j} = \fpeval{\mc@mat{\i,\j} - \factor * \mc@mat{\k,\j}};
            };
          };
          \result = \fpeval{\result * \mc@mat{\k,\k}};
        };% stop
      };% stop
    };
    \result = \fpeval{\result * \mc@mat{\n,\n}};
    return \result;
  };
}

\tikzset {
  % display matrix: show={\a,m,n}
  show/.code args={#1,#2,#3} {
    \tikzmath{
      print{\ \\\detokenize{#1}\\};% convert to string
      int \mc@i, \mc@j;
      for \mc@i in {1,...,(#2)} {
        for \mc@j in {1,...,(#3)} {
          print{[\mc@i,\mc@j]:\ #1{\mc@i,\mc@j}, \ \ };
        };
        print{\\};
      };
    }%tikzmath
  },
  % display matrix: stdout={\a,m,n}
  stdout/.code args={#1,#2,#3} {
    \message{matrix \detokenize{#1}:^^J}% convert to string
    \foreach \i in {1,...,#2} {
      \foreach \j in {1,...,#3} {
        \message{[\i,\j]: #1{\i,\j} }
      }
      \message{^^J}%line break
    }
  },
  % scalar multiplication: show={\a,m,n,k,\b}
  % b = k * a
  scalar/.code args={#1,#2,#3,#4,#5} {
    \tikzmath{
      int \mc@i, \mc@j;
      for \mc@i in {1,...,(#2)} {
        for \mc@j in {1,...,(#3)} {
          #5{\mc@i,\mc@j} = (#4)*(#1{\mc@i,\mc@j});
        };
      };
    }%tikzmath
  },
  % transpose matrix: transpose={\a,m,m,\b}
  % b = transpose(a)
  transpose/.code args={#1,#2,#3,#4} {
    \tikzmath{
      int \mc@i, \mc@j;
      for \mc@i in {1,...,(#2)} {
        for \mc@j in {1,...,(#3)} {
          #4{\mc@j,\mc@i} = #1{\mc@i,\mc@j};
        };
      };
    }%tikzmath
  },
  % matrix multiplication: product={\a,\b,row1,col1,col2,\c}
  % returns the matrix product of two matrices: c = a * b
  % size(a) = row1 x col1, size(b) = row2 x col2
  % col1 = row2
  product/.code args={#1,#2,#3,#4,#5,#6} {
    \tikzmath{
      int \mc@i, \mc@j, \mc@k;
      for \mc@i in {1,...,(#3)} {
        for \mc@j in {1,...,(#5)} {
          #6{\mc@i,\mc@j} = 0.0;
          for \mc@k in {1,...,(#4)} {
            #6{\mc@i,\mc@j} = \fpeval{#6{\mc@i,\mc@j} + 
                              #1{\mc@i,\mc@k} * #2{\mc@k,\mc@j}};
          };
        };
      };
    }%tikzmath
  },
  % Determinant calculation using Gaussian elimination with partial pivoting
  % det={\a,n,d}
  % the result is stored in d, the original matrix is not modified.
  det/.code args={#1,#2,#3} {
    \tikzmath{
      int \mc@i, \mc@j;
      % keep the original matrix
      for \mc@i in {1,...,(#2)}{
        for \mc@j in {1,...,(#2)}{
          \mc@mat{\mc@i,\mc@j} = #1{\mc@i,\mc@j};
        };
      };
      #3 = det(#2);
    }%tikzmath
  },
  % The (i, j) minor is the det of the submatrix formed 
  % by deleting the i-th row and j-th column. 
  % The (i, j) cofactor is obtained by multiplying 
  % the minor by (−1)^(i + j).
  % minors={\a,\n,\b}
  % the result is stored in \minor, 
  % the original matrix is not modified.
  minors/.code args={#1,#2,#3}{
    \tikzmath{
      int \mc@r, \mc@c, \mc@n;
      int \mc@i, \mc@j;
      \mc@n = #2;
      for \mc@r in {1,...,\mc@n}{
        for \mc@c in {1,...,\mc@n}{
          % initalize \mc@mat
          int \mc@nexti, \mc@nextj;
          for \mc@i in {1,...,\mc@n} {
            for \mc@j in {1,...,\mc@n} {
              if \mc@i != \mc@r && \mc@j != \mc@c then{
                if \mc@i < \mc@r then {
                  \mc@nexti = \mc@i;
                } else {
                  \mc@nexti = \mc@i - 1;
                };
                if \mc@j < \mc@c then {
                  \mc@nextj = \mc@j;
                } else {
                  \mc@nextj = \mc@j - 1;
                };
                \mc@mat{\mc@nexti,\mc@nextj} = #1{\mc@i,\mc@j};
              };%if
            };
          };%for
          % calculate det, we cannot directly use \mc@n-1 as bellow
          % #2{\mc@r,\mc@c} = det(\mc@n-1);
          int \mc@dim;
          \mc@dim = \mc@n - 1;
          #3{\mc@r,\mc@c} = det(\mc@dim);
        };
      };
    }% No commas after it
  },
  cofactors/.code args={#1,#2,#3}{
    \tikzset{
      minors={#1,#2,#3},
    }
    \tikzmath{
      int \mc@r, \mc@c, \mc@n;
      \mc@n = #2;
      for \mc@r in {1,...,\mc@n}{
        for \mc@c in {1,...,\mc@n}{
          #3{\mc@r,\mc@c} = (-1)^(\mc@r+\mc@c) * #3{\mc@r,\mc@c};
        };
      };
    }%tikzmath
  },
  % Solving linear systems using Gaussian-Jordan elimination with partial pivoting
  % solve={\a,\b,row1,col2,\c}
  % solve ax=b, size(a) = row1 x row1, size(b) = row1 x col2, size(c) = row1 x col2
  % if the system is solvable, \status is 1; otherwise 0.
  solve/.code args={#1,#2,#3,#4,#5}{
    \tikzmath{
      % print{Solve linear system...\\};
      \status = 1; % succeed
      int \mc@i, \mc@j, \mc@k;
      int \mc@row1, \mc@col2;
      \mc@row1 = #3;
      \mc@col2 = #4;
      % keep the original matrices
      for \mc@i in {1,...,\mc@row1} {
        for \mc@j in {1,...,\mc@row1} {
          \mc@a{\mc@i,\mc@j} = #1{\mc@i,\mc@j};
        };
        for \mc@j in {1,...,\mc@col2} {
          #5{\mc@i,\mc@j} = #2{\mc@i,\mc@j};
        };
      };
      int \mc@stop;
      \mc@stop = 0;
      % for 的第一项总是执行的,要限制范围
      for \mc@k in {1,...,\mc@row1} {
        if \mc@stop == 0 then {
          \mc@pivotRow = \mc@k;
          \mc@pivot = abs(\mc@a{\mc@k,\mc@k});
          % pivoting
          if \mc@k < \mc@row1 then {
            for \mc@i in {\mc@k+1,...,\mc@row1} {
              if \mc@pivot < abs(\mc@a{\mc@i,\mc@k}) then {
                \mc@pivotRow = \mc@i;
                \mc@pivot = abs(\mc@a{\mc@i,\mc@k});
              };
            };
          };
          if \mc@pivot < \EPSILON then { 
            \mc@stop = 1;
            \status = 0; % failed
          };
          if \mc@stop == 0 then {
            % swap
            if \mc@pivotRow != \mc@k then {
              % swap A
              for \mc@j in {\mc@k,...,\mc@row1}{
                \mc@temp = \mc@a{\mc@k,\mc@j};
                \mc@a{\mc@k,\mc@j} = \mc@a{\mc@pivotRow,\mc@j};
                \mc@a{\mc@pivotRow,\mc@j} = \mc@temp;
              };
              % swap B
              for \mc@j in {1,...,\mc@col2}{
                \mc@temp = #5{\mc@k,\mc@j};
                #5{\mc@k,\mc@j} = #5{\mc@pivotRow,\mc@j};
                #5{\mc@pivotRow,\mc@j} = \mc@temp;
              };
            };
            % eliminate
            int \mc@next;
            for \mc@next in {0,...,\mc@row1-1}{
              %先处理第 k 行(i.e. \mc@next=0), 将主元归一化, 然后消去其它行
              \mc@i =int(mod(\mc@k + \mc@next, \mc@row1)) ;
              if \mc@i == 0 then {
                \mc@i = \mc@row1;
              };
              \mc@temp = \mc@a{\mc@i,\mc@k};
              % eliminate A
              for \mc@j in {\mc@k,...,\mc@row1}{
                if \mc@i == \mc@k then {
                  \mc@a{\mc@i,\mc@j} = \fpeval{\mc@a{\mc@i,\mc@j} / \mc@temp};
                } else {%注意不能少else
                  \mc@a{\mc@i,\mc@j} = \fpeval{\mc@a{\mc@i,\mc@j} 
                                       - \mc@temp * \mc@a{\mc@k,\mc@j}};
                };
              };
              % eliminate B
              for \mc@j in {1,...,\mc@col2}{
                if \mc@i == \mc@k then {
                  #5{\mc@i,\mc@j} = \fpeval{#5{\mc@i,\mc@j} / \mc@temp};
                } else {%注意不能少else
                #5{\mc@i,\mc@j} = \fpeval{#5{\mc@i,\mc@j} 
                                  - \mc@temp * #5{\mc@k,\mc@j}};
                };
              };
            };
          };% stop
        };% stop
      };%for
    }%tikzmath
  },
  % inverse matrix: inverse={\a,n,\b}
  % b = inverse(a)
  inverse/.code args={#1,#2,#3}{
    \tikzmath{
      int \mc@n;
      \mc@n = #2;
      % initialize
      for \mc@i in {1,...,\mc@n} {
        for \mc@j in {1,...,\mc@n} {
          if \mc@i == \mc@j then {
            \mc@temp{\mc@i,\mc@j} = 1.0;
          } else {
            \mc@temp{\mc@i,\mc@j} = 0.0;
          };
        };
      };
    }%tikzmath
    \tikzset {
      solve={#1,\mc@temp,\mc@n,\mc@n,#3},
    }
  },
  % Egienvalues and Egienvectors of a symetric matrix
  % jacobi={\a,n,\d,\v}
  % \a: a symetric matrix
  % \d: returns a diagnoalized matrix with eigenvalues on diagonal
  % \v: returns the eigenvector matrix
  % n: size of matrix 'a'
  % change \status to 1 if maximum number of iterations 
  % hasn't been exceeded, otherwise 0;
  jacobi/.code args = {#1,#2,#3,#4} {
    \tikzmath{
      \status = 1;
      int \mc@i, \mc@j, \mc@k, \mc@n;
      \mc@n = #2;
      % max iterations
      int \JacobiMaxIter;
      if \JacobiMaxIter < 0 then {% not overriden by user
        \JacobiMaxIter = 3 * (\mc@n-1)^2;
      };
      % Initialize eigenvector matrix as identity
      for \mc@i in {1,...,\mc@n} {
        for \mc@j in {1,...,\mc@n} {
          #3{\mc@i,\mc@j} = #1{\mc@i,\mc@j};
          if \mc@i == \mc@j then {
            #4{\mc@i,\mc@j} = 1.0;
          } else {
            #4{\mc@i,\mc@j} = 0.0;
          };
        };
      };
      % print{JacobiMaxIter: \JacobiMaxIter\\};
      int \mc@stop;
      \mc@stop = 0;
      for \mc@k in {1,...,\JacobiMaxIter} {
        % Performs a single Jacobi rotation to eliminate an off-diagonal element
        if \mc@stop == 0 then {
          % find the maximum off-diagnoal element
          int \mc@p, \mc@q; % position p < q
          real \mc@temp;
          \mc@p = 1; \mc@q = 2; \mc@temp = 0.0;
          for \mc@i in {1,...,\mc@n-1} {
            for \mc@j in {\mc@i+1,...,\mc@n} {
              % print{[\mc@i,\mc@j]:\ #2{\mc@i,\mc@j}\\};
              if abs(#3{\mc@i,\mc@j}) > \mc@temp then {
                \mc@temp = abs(#3{\mc@i,\mc@j});
                \mc@p = \mc@i;
                \mc@q = \mc@j;
              };
            };
          };
          % print{abs max:[\mc@p,\mc@q]:\ #2{\mc@p,\mc@q}\\};
          if \mc@temp < \JacobiEpsilon then {
            % we cannot break for-loop in tikzmath
            \mc@stop = 1;
          }; 
          if \mc@stop == 0 then {
            % rotation angle: -pi/4 <= θ <= pi/4
            % assuming tan(θ) = t, sin(θ) = s, cos(θ) = c
            % tau = (a[pp]-a[qq])/(2a[pq]) = cot(2θ) = (1-tan(θ)^2)/2tan(θ)
            % then t satisfies the quadratic equation:
            % t^2 + 2 tau t - 1 = 0
            % Choosing the root that is smaller in absolute value,
            % we can compute all relevant quantities without 
            % using trigonometric functions,
            % t = sign(tau)/(abs(tau)+sqrt(1+tau^2)) 
            % i.e. t = -tau + sqrt(1+tau^2), if tau >=0; 
            %      t = -tau - sqrt(1+tau^2), otherwise.
            % c = 1/sqrt(1+t^2), s = c*t
            real \mc@c, \mc@s, \mc@t, \mc@tau;
            \mc@tau = \fpeval{(#3{\mc@p,\mc@p}-#3{\mc@q,\mc@q})/(2*#3{\mc@p,\mc@q})};
            if \mc@tau >= 0 then {
              \mc@t = \fpeval{- \mc@tau + sqrt(1+(\mc@tau)^2)};
            } else {
              \mc@t = \fpeval{- \mc@tau - sqrt(1+(\mc@tau)^2)};
            };
            \mc@c = \fpeval{1 / sqrt(1+(\mc@t)^2)};
            \mc@s = \fpeval{\mc@t * \mc@c};
            % print{$\tau:\mc@tau,\\ \tan\theta:\mc@t,\\ \cos\theta:\mc@c,\\ sin\theta:\mc@s$\\};
            % rotation matrix:
            % R = [
            %   {...  .  ...  .  ...},
            %   {...  c  ... -s  ...},
            %   {...  s  ...  c  ...},
            %   {...  .  ...  .  ...}
            % ], 
            % perform Jacobi rotation
            for \mc@i in {1,...,\mc@n}{
              \mc@ap{\mc@i} = \fpeval{  #3{\mc@i,\mc@p} * \mc@c + #3{\mc@i,\mc@q} * \mc@s};
              \mc@aq{\mc@i} = \fpeval{- #3{\mc@i,\mc@p} * \mc@s + #3{\mc@i,\mc@q} * \mc@c};
              \mc@vp{\mc@i} = \fpeval{  #4{\mc@i,\mc@p} * \mc@c + #4{\mc@i,\mc@q} * \mc@s};
              \mc@vq{\mc@i} = \fpeval{- #4{\mc@i,\mc@p} * \mc@s + #4{\mc@i,\mc@q} * \mc@c};
            };
            \mc@ap{\mc@p} = \fpeval{#3{\mc@p,\mc@p}*(\mc@c)^2 + 
                            #3{\mc@q,\mc@q}*(\mc@s)^2 + 2*#3{\mc@p,\mc@q}*\mc@s*\mc@c};
            \mc@aq{\mc@q} = \fpeval{#3{\mc@p,\mc@p}*(\mc@s)^2 + 
                            #3{\mc@q,\mc@q}*(\mc@c)^2 - 2*#3{\mc@p,\mc@q}*\mc@s*\mc@c};
            \mc@ap{\mc@q} = 0.0;
            \mc@aq{\mc@p} = 0.0;
            for \mc@i in {1,...,\mc@n}{
              % Update matrix
              #3{\mc@i,\mc@p} = \mc@ap{\mc@i};
              #3{\mc@i,\mc@q} = \mc@aq{\mc@i};
              #3{\mc@p,\mc@i} = \mc@ap{\mc@i};
              #3{\mc@q,\mc@i} = \mc@aq{\mc@i};
              % Update eigenvectors
              #4{\mc@i,\mc@p} = \mc@vp{\mc@i};
              #4{\mc@i,\mc@q} = \mc@vq{\mc@i};
            };
            #3{\mc@q,\mc@p} = 0.0;
            % for \mc@i in {1,...,\mc@n} {
            %   for \mc@j in {1,...,\mc@n} {
            %     print{[\mc@i,\mc@j]:\ #2{\mc@i,\mc@j}, \ };
            %   };
            %   print{\\};
            % };
          };%if
        };%if
      };%for
      if \mc@stop == 0 then {
        \status = 0;
      };
    }%tikzmath
  },%jacobi
  % maximum in absolute value
  % maxabs={\a,m,n,\max}
  maxabs/.code args = {#1,#2,#3,#4} {
    \tikzmath {  
      int \mc@i, \mc@j;
      \mc@max = 0.0;
      for \mc@i in {1,...,(#2)} {
        for \mc@j in {1,...,(#3)} {
          if \mc@max < abs(#1{\mc@i,\mc@j}) then {
            \mc@max = abs(#1{\mc@i,\mc@j});
          };
        };
      };
      #4 = \mc@max;
    }
  },
  % normalize={\a,m,n,\b}
  normalize/.code args = {#1,#2,#3,#4} {
    \tikzset{maxabs={#1,#2,#3,\mc@max},}
    \tikzmath{
      int \mc@i, \mc@j;
      for \mc@i in {1,...,(#2)} {
        for \mc@j in {1,...,(#3)} {
          #4{\mc@i,\mc@j} = \fpeval{(#1{\mc@i,\mc@j})/\mc@max};
        };
      };
    }%tikzmath
  },
}

\makeatother
