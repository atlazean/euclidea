\chapter{三角形 Triangles}

\section{勾股定理 Pythagorean theorem}

\begin{theorem}[勾股定理,毕达哥拉斯定理,Pythagorean theorem\label{pythagorean}]
直角三角形中两直角边的平方和等于斜边的平方.
\end{theorem}

毕达哥拉斯\footnote{毕达哥拉斯(前570年-前495年): 是一名古希腊哲学家,数学家和音乐理论家,毕达哥拉斯主义的创立者.}
学派的证明没有流传下来,流传下来书面证明最早见于《几何原本》第一册的第47个命题.
在中国,东汉末年吴国的赵爽最早给出勾股定理的证明.
勾股定理是人类早期发现并证明的重要数学定理之一, 也是联系几何和代数的纽带, 如解析几何中两点的距离就可以由勾股定理计算, 
Stillwell甚至将勾股定理作为数学历史的开篇\cite{stillwell}.

\begin{proof}
采用面积法证明, 正方形的面积由 4 个三角形面积加上中间小正方形面积, 如图\ref{pythagoras-fig}:
\begin{align*}
  S &= 4(\cfrac{1}{2}ab)+c^2 \\
    &= 2ab+c^2 \\
  \intertext{而:}
  S &= (a+b)^2\\
  \intertext{所以:}
  c^2 &= a^2+b^2
\end{align*}
\end{proof}

\begin{figure}[!htb]
\begin{center}
\begin{tikzpicture}[scale=.75]
  \tikzmath{
    \a = 2.5;
    \b = 4.5;
  }
  \coordinate (A) at (0,0);
  \coordinate (B) at (\a+\b,0);
  \coordinate (C) at (\a+\b,\a+\b);
  \coordinate (D) at (0,\a+\b);
  \coordinate (E) at (\a,0);
  \coordinate (F) at (\a+\b,\a);
  \coordinate (G) at (\b,\a+\b);
  \coordinate (H) at (0,\b);
  \draw[thick,fill=cyan,opacity=0.75] (A) -- (E) -- (H) -- cycle;
  \draw[thick,fill=cyan,opacity=0.75] (B) -- (F) -- (E) -- cycle;
  \draw[thick,fill=cyan,opacity=0.75] (C) -- (G) -- (F) -- cycle;
  \draw[thick,fill=cyan,opacity=0.75] (D) -- (H) -- (G) -- cycle;
  \draw ($(A)!.5!(E)$) node[below] {$a$};
  \draw ($(E)!.5!(B)$) node[below] {$b$};
  \draw ($(B)!.5!(F)$) node[right] {$a$};
  \draw ($(F)!.5!(C)$) node[right] {$b$};
  \draw ($(C)!.5!(G)$) node[above] {$a$};
  \draw ($(G)!.5!(D)$) node[above] {$b$};
  \draw ($(D)!.5!(H)$) node[left] {$a$};
  \draw ($(H)!.5!(A)$) node[left] {$b$};
  \draw ($(E)!.5!(F)$) node[above] {$c$};
  \draw ($(F)!.5!(G)$) node[below left] {$c$};
  \draw ($(G)!.5!(H)$) node[below] {$c$};
  \draw ($(H)!.5!(E)$) node[above right] {$c$};
\end{tikzpicture}
\end{center}
\caption{勾股定理的证明}
\label{pythagoras-fig}
\end{figure}

\begin{theorem}[勾股定理的逆定理]
  勾股定理的逆定理是判断三角形为钝角、锐角或直角的一个简单的方法, 其中$AB=c$为最长边:
  \begin{itemize}
    \item 如果$a^2 + b^2 = c^2$,则$\triangle ABC$是直角三角形,其中$\angle C$是直角.
    \item 如果$a^2 + b^2 > c^2$,则$\triangle ABC$是锐角三角形(若无先前条件$AB=c$为最长边,则该式的成立仅满足$\angle C$是锐角).
    \item 如果$a^2 + b^2 < c^2$,则$\triangle ABC$是钝角三角形,其中$\angle C$是钝角.
  \end{itemize}
\end{theorem}

这个逆定理实际上是\hyperref[cosines]{余弦定理}的一个推论.

\section{射影定理 Geometric mean theorem}

\begin{theorem}[射影定理,Geometric mean theorem,或Euclid's theorem of altitude]
在 $\triangle{ABC}$ 中 $\angle C = 90^\circ$. 设 $CD$ 在 $AB$ 的上的高, 则有:
\begin{align*}
  {AC}^{2} &= AD \cdot AB \\
  {BC}^{2} &= BD \cdot AB \\
  {CD}^{2} &= AD \cdot DB
\end{align*}
\end{theorem}

这个定理出现在欧几里得所著《几何原本》第一卷当中,作为\hyperref[pythagorean]{勾股定理}证明过程的一部分.
在这里, $AD$ 及 $BD$ 分别是 $AC$ 及 $BC$ 在底边 $AB$ 的正投影, 故定理以此为名.

\begin{figure}[!htb]
\begin{center}
\begin{tikzpicture}
  \tikzmath{
    \r = 4;  % radius of circumcircle
    \a = 35; % degree of angle a
  }
  \coordinate (A) at (180:\r);
  \coordinate (B) at (0:\r);
  \coordinate (C) at (2*\a:\r);
  \coordinate (D) at ($(A)!(C)!(B)$);
  \draw[thick] (A) -- (B) -- (C) -- cycle;
  \draw[thick,red] (C) -- (D);
  \foreach \p in {A,B,C,D}
    \fill[red] (\p) circle (2pt);
  \pic[draw,red,angle radius=6pt]{right angle=A--C--B};
  \pic[draw,red,angle radius=6pt]{right angle=C--D--B};
  \draw (A) node[below left] {$A$};
  \draw (B) node[below right] {$B$};
  \draw (C) node[above] {$C$};
  \draw (D) node[below] {$D$};
\end{tikzpicture}
\end{center}
\caption{射影定理}
\end{figure}

\begin{proof}
利用相似三角形的比例关系容易证明
\begin{align*}
  \because \triangle ABC &\sim \triangle ACD \\
  \therefore \cfrac{AB}{AC} &= \cfrac{AC}{AD} \\
  \intertext{也即:} AC^2 &= AD \cdot AB \\
  \intertext{同理可证:} {BC}^{2} &= BD \cdot AB \\
  {CD}^{2} &= AD \cdot DB
\end{align*}
\end{proof}

\section{中线定理 Apollonius's theorem}

\begin{theorem}[中线定理,阿波罗尼斯定理,Apollonius's theorem]
在$\triangle{ABC}$中, 点 $D$ 是 $BC$ 的中点, 则 \[AB^2+AC^2=2(AD^2+BD^2)\]
\end{theorem}

\begin{figure}[!htb]
\begin{center}
\begin{tikzpicture}[scale=.75]
  \coordinate (A) at (2,5);
  \coordinate (B) at (0,0);
  \coordinate (C) at (7,0);
  \coordinate (D) at ($(B)!.5!(C)$); 
  \coordinate (E) at ($(B)!(A)!(C)$);
  \pic[draw,red,angle radius=6pt]{right angle=A--E--C};
  \draw[thick] (A) -- (B) -- (C) -- cycle;
  \draw[thick,red] (A) -- (D);
  \draw[thick,blue] (A) -- (E);
  \foreach \p in {A,B,C,D,E}
    \fill[red] (\p) circle (2pt);
  \draw (A) node[above] {$A$};
  \draw (B) node[below left] {$B$};
  \draw (C) node[below right] {$C$};
  \draw (D) node[below] {$D$};
  \draw (E) node[below] {$E$};
\end{tikzpicture}
\end{center}
\caption{中线定理}
\end{figure}

\begin{proof}
利用勾股定理
\begin{align*}
AB^2+AC^2 &= (AE^2+BE^2)+(AE^2+CE^2) \\
          &= 2AE^2+BE^2+CE^2 \\
          &= 2AE^2+(BD-DE)^2+(CD+DE)^2 \\
          &= 2AE^2 + (BD^2+DE^2-2BD \cdot DE) + (CD^2+DE^2-2CD \cdot DE)
\intertext{因为 $BD=CD$, 所以:}
AB^2+AC^2 &= 2AE^2 + 2ED^2 + 2BD^2 \\
          &= 2(AE^2+ED^2) + 2BD^2 \\
          &= 2AD^2 + 2BD^2 \\
          &= 2(AD^2+BD^2)
\end{align*}

\end{proof}

\section{角平分线定理 Angle bisector theorem}

\begin{theorem}[角平分线定理,Angle bisector theorem]
在三角形$\triangle{ABC}$中, 由$A$点作一角平分线与$BC$交于$D$, 则\[\cfrac{AB}{AC}=\cfrac{BD}{DC}\]
\end{theorem}

\begin{figure}[!htb]
\begin{center}
\begin{tikzpicture}[scale=.75]
  \coordinate (A) at (2,5);
  \coordinate (B) at (0,0);
  \coordinate (C) at (7,0);
  \coordinate[revolve/angle={A,B,C},
    revolve/scale=1/2,revolve={A,B}] (D');
  \coordinate[intersect={A,D',B,C}] (D);
  \draw[thick] (A) -- (B) -- (C) -- cycle;
  \draw[thick,red] (A) -- (D);
  \foreach \p in {A,B,C,D}
    \fill[red] (\p) circle (2pt);
  \pic[draw,blue,angle radius=12pt] {angle=B--A--D};
  \pic[draw,blue,angle radius=14pt] {angle=D--A--C};
  \draw (A) node[above] {$A$};
  \draw (B) node[below left] {$B$};
  \draw (C) node[below right] {$C$};
  \draw (D) node[below] {$D$};
\end{tikzpicture}
\end{center}
\caption{角平分线定理}
\end{figure}

\begin{proof}
利用角平分线的性质和面积法
\begin{align*}
  \intertext{一方面:}
  \cfrac{S_{\triangle ABD}}{S_{\triangle ADC}} &= \cfrac{AB \cdot AD \cdot \sin {\angle ABD}}{AC \cdot AD \cdot \sin {\angle ADC}} \\
  &= \cfrac{AB}{AC} \\
  \intertext{另一方面:}
  \cfrac{S_{\triangle ABD}}{S_{\triangle ADC}} &=\cfrac{BD}{DC} \\
  \intertext{所以:}
  \cfrac{AB}{AC} &= \cfrac{BD}{DC}
\end{align*}
\end{proof}

\section{正弦定理 Law of sines}

\begin{theorem}[正弦定理,Law of sines\label{sines}]
对于任意 $\triangle{ABC}$, $a,b,c$分别为$\angle A,\angle B,\angle C$的对边, $R$为$\triangle ABC$
的外接圆半径, 则:
\[\cfrac{a}{\sin A} = \cfrac{b}{\sin B} = \cfrac{c}{\sin C} = 2R\]
\end{theorem}

\begin{figure}[!htb]
\begin{center}
\begin{tikzpicture}[scale=.75]
  \coordinate (A) at (2,5);
  \coordinate (B) at (0,0);
  \coordinate (C) at (7,0);
  \coordinate[circumcenter={A,B,C}] (O);
  \coordinate (D) at ($(B)!2!(O)$);
  \draw[thick] (A) -- (B) -- (C) -- cycle;
  \draw[circle={O,A}];
  \foreach \p in {O,A,B,C,D}
    \fill[red] (\p) circle (2pt);
  \draw[red,thick,dashed] (B) -- (D);
  \draw[red,thick,dashed] (C) -- (D);
  \pic[draw,red,angle radius=6pt]{right angle=B--C--D};
  \pic[draw,purple,double,angle radius=12pt] {angle=B--A--C};
  \pic[draw,purple,double,angle radius=12pt] {angle=B--D--C};
  \draw (A) node[above] {$A$};
  \draw (B) node[below left] {$B$};
  \draw (C) node[below right] {$C$};
  \draw (D) node[above right] {$D$};
  \draw (O) node[below] {$O$};
\end{tikzpicture}
\end{center}
\caption{正弦定理}
\end{figure}

\begin{proof}
利用\hyperref[inscribed]{圆周角定理}
\end{proof}

\section{余弦定理 Law of cosines}

\begin{theorem}[余弦定理,Law of cosines\label{cosines}]
对于任意 $\triangle{ABC}$, $a,b,c$分别为$\angle A,\angle B,\angle C$的对边, 则:
\begin{align*}
  a^2 &= b^2 + c^2 -2bc \cos A \\
  b^2 &= c^2 + a^2 -2ca \cos B \\
  c^2 &= a^2 + b^2 -2ab \cos C
\end{align*}
\end{theorem}

\begin{figure}[!htb]
\begin{center}
\begin{tikzpicture}[scale=.75]
  \coordinate (A) at (2,5);
  \coordinate (B) at (0,0);
  \coordinate (C) at (7,0);
  \coordinate (D) at ($(A)!(B)!(C)$);
  \draw[thick] (A) -- (B) -- (C) -- cycle;
  \foreach \p in {A,B,C,D}
    \fill[red] (\p) circle (2pt);
  \draw[red,dashed] (B) -- (D);
  \pic[draw,red,angle radius=6pt]{right angle=B--D--A};
  \draw (A) node[above] {$A$};
  \draw (B) node[below left] {$B$};
  \draw (C) node[below right] {$C$};
  \draw (D) node[above right] {$D$};
\end{tikzpicture}
\end{center}
\caption{余弦定理}
\end{figure}

\begin{proof}
利用勾股定理
\begin{align*}
  a^2 &= {BD}^2+{CD}^2 \\
      &= (c\sin C)^2 + (b-c\cos C)^2 \\
      &= b^2+c^2-2bc \cos C
\end{align*}
\end{proof}

\section{梅涅劳斯定理 Menelaus' theorem}

\begin{theorem}[梅涅劳斯定理,Menelaus' theorem\label{menelaus}]
如果一直线与$\triangle ABC$的边$BC,CA,AB$或其延长线分别交于$D,E,F$,
则有:
\[\cfrac{AF}{FB} \cdot \cfrac{BD}{DC} \cdot \cfrac{CE}{EA} = 1\]
\end{theorem}

该定理以古希腊数学家和天文学家梅涅劳斯(Menelaus of Alexandria)\footnote{\url{https://en.wikipedia.org/wiki/Menelaus_of_Alexandria}}命名.

\begin{theorem}[梅涅劳斯定理的逆定理]
如果有三点$D,E,F$分别在$\triangle ABC$的边$BC,CA,AB$或其延长线上, 且满足
\[\cfrac{AF}{FB} \cdot \cfrac{BD}{DC} \cdot \cfrac{CE}{EA} = 1\]
则$D,E,F$三点共线.
\end{theorem}

利用这个逆定理,可以判断三点共线.
如果在上式中线段用有向线段表示,那么右面的结果为-1.
该定理与\hyperref[ceva]{塞瓦定理}的等式仅在条件上有所不同,二者互为对偶定理.

\begin{figure}[!htb]
\begin{center}
\begin{tikzpicture}[scale=.75]
  \coordinate (A) at (2,5);
  \coordinate (B) at (0,0);
  \coordinate (C) at (7,0);
  \coordinate (M) at (0,3);
  \coordinate (N) at (12,-1);
  \draw[thick] (A) -- (B) -- (C) -- cycle;
  \coordinate[intersect={B,C,M,N}] (D);
  \coordinate[intersect={C,A,M,N}] (E);
  \coordinate[intersect={A,B,M,N}] (F);
  \coordinate[translate={N,M,C}] (G);
  \coordinate[intersect={A,B,C,G}] (G);
  \foreach \p in {A,B,C,D,E,F,G}
    \fill[red] (\p) circle (2pt);
  \draw[red] (M) -- (N);
  \draw[thick] (C) -- (D);
  \draw[blue,dashed] (C) -- (G);
  \draw (A) node[above] {$A$};
  \draw (B) node[below left] {$B$};
  \draw (C) node[below right] {$C$};
  \draw (D) node[below] {$D$};
  \draw (E) node[above right] {$E$};
  \draw (F) node[above left] {$F$};
  \draw (G) node[left] {$G$};
\end{tikzpicture}
\end{center}
\caption{梅涅劳斯定理}
\end{figure}

\begin{proof}
作$CG \parallel DF$交 $AB$ 与点$G$,将所有比例关系转换到$AB$上,
\begin{align*}
  \cfrac{BD}{DC} &= \cfrac{FB}{FG} \\
  \cfrac{CE}{EA} &= \cfrac{FG}{AF} \\
  \intertext{所以:}
  \cfrac{AF}{FB} \cdot \cfrac{BD}{DC} \cdot \cfrac{CE}{EA} 
  &= \cfrac{AF}{FB} \cdot \cfrac{FB}{FG} \cdot \cfrac{FG}{AF} = 1
\end{align*}
\end{proof}


\section{塞瓦定理 Ceva's theorem}

\begin{theorem}[塞瓦定理,或西瓦定理,Ceva's theorem\label{ceva}]
如果有三点$D,E,F$分别在$\triangle ABC$的边$BC,CA,AB$或其延长线上, 且$AD,BE,CF$通过同一点O,则:
\[\cfrac{AF}{FB} \cdot \cfrac{BD}{DC} \cdot \cfrac{CE}{EA} = 1\]
\end{theorem}

该定理最先由意大利数学家乔瓦尼·塞瓦(Giovanni Ceva,1647年12月7日-1734年6月15日)证明.

\begin{theorem}[塞瓦定理的逆定理]
如果有三点$D,E,F$分别在$\triangle ABC$的边$BC,CA,AB$或其延长线上, 且满足
\[\cfrac{AF}{FB} \cdot \cfrac{BD}{DC} \cdot \cfrac{CE}{EA} = 1\]
则直线$AD,BE,CF$共点或彼此平行(于无限远处共点).
\end{theorem}

\begin{figure}[!htb]
\begin{center}
\begin{tikzpicture}[scale=.75]
  \coordinate (A) at (2,5);
  \coordinate (B) at (0,0);
  \coordinate (C) at (7,0);
  \coordinate (O) at (3,3);
  \draw[thick] (A) -- (B) -- (C) -- cycle;
  \coordinate[intersect={B,C,A,O}] (D);
  \coordinate[intersect={C,A,B,O}] (E);
  \coordinate[intersect={A,B,C,O}] (F);
  \foreach \p in {A,B,C,D,E,F,O}
    \fill[red] (\p) circle (2pt);
  \draw[red] (A) -- (D);
  \draw[red] (B) -- (E);
  \draw[red] (C) -- (F);
  \draw (A) node[above] {$A$};
  \draw (B) node[below left] {$B$};
  \draw (C) node[below right] {$C$};
  \draw (D) node[below] {$D$};
  \draw (E) node[above right] {$E$};
  \draw (F) node[above left] {$F$};
  \draw (O) node[below] {$O$};
\end{tikzpicture}
\end{center}
\caption{塞瓦定理}
\end{figure}

\begin{proof}
面积法

\begin{align*}
  \cfrac{AF}{FB} = \cfrac{S_{\triangle{CAF}}}{S_{\triangle{CFB}}}
  = \cfrac{S_{\triangle{OAF}}}{S_{\triangle{OFB}}} \\
  \intertext{利用等比性质:}
  \cfrac{AF}{FB} = \cfrac{S_{\triangle{CAF}}-S_{\triangle{OAF}}}{S_{\triangle{CFB}}-S_{\triangle{OFB}}}
  = \cfrac{S_{\triangle{OCA}}}{S_{\triangle{OBC}}} \\
  \intertext{同理:}
  \cfrac{BD}{DC} = \cfrac{S_{\triangle{OAB}}}{S_{\triangle{OCA}}} \\
  \cfrac{CE}{EA} = \cfrac{S_{\triangle{OBC}}}{S_{\triangle{OAB}}} \\
  \cfrac{AF}{FB} \cdot \cfrac{BD}{DC} \cdot \cfrac{CE}{EA} 
  = \cfrac{S_{\triangle{OCA}}}{S_{\triangle{OBC}}} 
  \cdot \cfrac{S_{\triangle{OAB}}}{S_{\triangle{OCA}}}
  \cdot \cfrac{S_{\triangle{OBC}}}{S_{\triangle{OAB}}}
  =1
\end{align*}
\end{proof}

\section{海伦公式 Heron's formula}

\begin{theorem}[海伦公式,Heron's formula,Hero's formula\label{heron}]
如果三角形的边长分别为$a,b,c$, 则三角形的面积$A$可由以下公式求得:
\[A= \sqrt{s(s-a)(s-b)(s-c)}\]
其中$s=\cfrac {a+b+c}{2}$.
\end{theorem}
  
该公式由古希腊数学家亚历山大港的海伦(Hero of Alexandria)发现, 并在其于公元60年所著的 {\itshape Metrica} 中载有数学证明,
原理是利用三角形的三条边长求取三角形面积.
亦有认为更早的阿基米德已经了解这条公式, 因为 {\itshape Metrica} 是一部古代数学知识的结集,
该公式的发现时间很有可能先于海伦的著作.

中国南宋末年数学家秦九韶发现或知道等价的公式,
其著作《数书九章》卷五第二题即三斜求积.
\[A=\cfrac{1}{2} \sqrt{a^2c^2-\left(\cfrac{a^2+c^2-b^2}{2}\right)^2}\]

\begin{remark*}
海伦公式在角度很小时可能存在数值不稳定
\urldef\heron\url{https://en.wikipedia.org/wiki/Heron%27s_formula}
\footnote{\heron}
,可以变形为:
\[A=\cfrac{1}{4}\sqrt{[a+(b+c)][c-(a-b)][c+(a-b)][a+(b-c)]}\]
其中$a \geqslant b \geqslant c$.
\end{remark*}

\begin{proof}
需要用到\hyperref[cosines]{余弦定理}

\begin{align*}
\cos C &= \cfrac{a^2+b^2-c^2}{2ab} \\
\sin C &= \sqrt{1- \cos^2 A} = \cfrac{\sqrt{4a^2b^2-(a^2+b^2-c^2)^2}}{2ab} \\
A &= \cfrac{1}{2} ab \sin C \\
  &= \cfrac{1}{4} \sqrt{4a^2b^2-(a^2+b^2-c^2)^2} \\
  &= \cfrac{1}{4} \sqrt{-a^4-b^4-c^4+2a^2b^2++2b^2c^2+2c^2a^2} \\
  &= \cfrac{1}{4} \sqrt{(a+b+c)(-a+b+c)(a-b+c)(a+b-c)} \\
  &= \sqrt{\cfrac{a+b+c}{2} \cdot \cfrac{-a+b+c}{2} \cdot \cfrac{a-b+c}{2} \cdot \cfrac{a+b-c}{2}} \\
  &= \sqrt{s(s-a)(s-b)(s-c)} 
\end{align*}
\end{proof}

\section{西姆松定理 Simson line theorem}

\begin{theorem}[西姆松定理,Simson line theorem]
平面上有一点$P$, $P$在$\triangle ABC$三边上的投影(即由$P$到边上的垂足)共线(此线称为西姆松线或译“西摩松线”, Simson line)当且仅当$P$在三角形的外接圆上.
\end{theorem}

\begin{figure}[!htb]
\begin{center}
\begin{tikzpicture}[scale=.75]
  \coordinate (A) at (2,5);
  \coordinate (B) at (0,0);
  \coordinate (C) at (7,0);
  \coordinate (P) at (7,5);
  \coordinate[circumcenter={A,B,C}] (O);
  \coordinate (D) at ($(B)!2!(O)$);
  \draw[thick] (A) -- (B) -- (C) -- cycle;
  \draw[circle={O,A},name path=circumcircle];
  \path[name path=l] (O) -- (P);
  \path[name intersections={of=l and circumcircle,by={P}}];
  \coordinate (D) at ($(B)!(P)!(C)$);
  \coordinate (E) at ($(C)!(P)!(A)$);
  \coordinate (F) at ($(A)!(P)!(B)$);
  \draw[red] (P) -- (D);
  \draw[red] (P) -- (E);
  \draw[red] (P) -- (F);
  \draw[blue] (D) -- (F);
  \draw[thick] (A) -- (F);
  \foreach \p in {O,A,B,C,D,E,F,P}
    \fill[red] (\p) circle (2pt);
  \pic[draw,red,angle radius=6pt]{right angle=P--D--C};
  \pic[draw,red,angle radius=6pt]{right angle=P--E--A};
  \pic[draw,red,angle radius=6pt]{right angle=P--F--B};
  \draw (A) node[above left] {$A$};
  \draw (B) node[below left] {$B$};
  \draw (C) node[below right] {$C$};
  \draw (P) node[above right] {$P$};
  \draw (D) node[below] {$L$};
  \draw (E) node[below] {$M$};
  \draw (F) node[above left] {$N$};
  \draw (O) node[below] {$O$};
\end{tikzpicture}
\end{center}
\caption{西姆松定理}
\end{figure}

\begin{proof} 利用圆内接四边形(cyclic quadrilateral)的性质.
因 $PMCL$ 是圆内接四边形, 故 $\angle PML = \angle PCL$.
因 $PBNM$ 是圆内接四边形, 故 $\angle PMN + \angle PBN = 180^\circ$.
因 $PBAC$ 是圆内接四边形, 故 $\angle PCL = \angle PBN$.
所以, $\angle PMN + \angle PCL = 180^\circ$, $L,M,N$ 共线.
\end{proof}

\section{莫利定理 Morley's trisector theorem}

\begin{theorem}[莫利定理,莫利角三分线定理,Morley's theorem,Morley's trisector theorem]
将三角形的三个内角三等分, 靠近某边的两条三分角线相交得到一个交点, 则这样的三个交点可以构成一个正三角形. 
\end{theorem}

此定理由法兰克·莫雷(Frank Morley)在1899年发现.这个三角形常被称作莫利正三角形. 
对外角作外角三分线, 也会有类似的性质,可以再作出4个等边三角形.

此定理有趣的地方是我们没办法用尺规作图作出其等边三角形,
因为已经证明出尺规作图无法作出三等分角.

\begin{figure}[!htb]
\begin{center}
\begin{tikzpicture}
  \coordinate (A) at (1,6);
  \coordinate (B) at (0,0);
  \coordinate (C) at (8,0);
  \coordinate[revolve/angle={A,B,C},
    revolve/scale=1/3,revolve={A,B}] (A1);
  \coordinate[revolve/angle={A,B,C},
    revolve/scale=2/3,revolve={A,B}] (A2);
  \coordinate[revolve/angle={B,C,A},
    revolve/scale=1/3,revolve={B,C}] (B1);
  \coordinate[revolve/angle={B,C,A},
    revolve/scale=2/3,revolve={B,C}] (B2);
  \coordinate[revolve/angle={C,A,B},
    revolve/scale=1/3,revolve={C,A}] (C1);
  \coordinate[revolve/angle={C,A,B},
    revolve/scale=2/3,revolve={C,A}] (C2);
  \coordinate[intersect={A,A1,B,B2}] (D);
  \coordinate[intersect={B,B1,C,C2}] (E);
  \coordinate[intersect={C,C1,A,A2}] (F);
  \draw[fill=green] (A) -- (B) -- (D) -- cycle;
  \draw[fill=cyan] (B) -- (C) -- (E) -- cycle;
  \draw[fill=yellow] (C) -- (A) -- (F) -- cycle;
  \draw[fill=purple] (D) -- (E) -- (F) -- cycle;
  \foreach \p/\placement in {A/above,B/below left,C/below right,
  D/left,E/below,F/above}{
    \fill[red] (\p) circle (2pt);
    \draw (\p) node[\placement] {$\p$};
  }
\end{tikzpicture}
\end{center}
\caption{莫利定理}
\end{figure}

\section{外森比克不等式 Weitzenböck's inequality}

\begin{theorem}[外森比克不等式,Weitzenböck's inequality]
设三角形的边长为$a,b,c$,面积为$S$, 则
\[a^2+b^2+c^2 \geqslant 4\sqrt{3}S\]
当且仅当三角形为等边三角形,等号成立.
\end{theorem}

\begin{proof}
\begin{align*}
  & a^2+b^2+c^2 \geqslant ab+ bc + ca \\
  &\Leftrightarrow 3(a^2+b^2+c^2) \geqslant (a+b+c)^2 \\
  &\Leftrightarrow a^2+b^2+c^2 \geqslant \sqrt{3(a+b+c)\left(\cfrac{a+b+c}{3}\right)^3} \\
  &\Leftrightarrow a^2+b^2+c^2 \geqslant \sqrt{3(a+b+c)(-a+b+c)(a-b+c)(a+b-c)} \\
  &\Leftrightarrow a^2+b^2+c^2 \geqslant 4\sqrt{3}S
\end{align*}
\end{proof}

\section{斯图尔特定理 Steward's theorem}

\begin{theorem}[斯图尔特定理,斯图沃特定理,Steward's theorem\label{steward}]
  在$\triangle ABC$中, 点$D$是$BC$上任意一点,则有:
  \[AD^2 \cdot BC + BD \cdot DC \cdot BC =
    AB^2 \cdot DC + AC^2 \cdot BD \]
\end{theorem}

证明很简单, 使用余弦定理.

\section{斯霍滕定理 Schouten's theorem}

\begin{theorem}[斯霍滕定理,Schouten's theorem]
  在$\triangle ABC$中, $\angle BAC$的角平分线交$BC$于点$D$,则有:
  \[AD^2  = AB \cdot AC - BD \cdot CD \]
\end{theorem}

斯霍滕(Schouten),荷兰数学家. 由\hyperref[steward]{斯图尔特定理}可证明.