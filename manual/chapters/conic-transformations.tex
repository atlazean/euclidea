\chapter{圆锥曲线的变换 Conic Transformations}

\section{坐标变换 Coordinate Transformations}

\subsection{Transformations on Axes}

下面讨论坐标轴平移和旋转后坐标的变化.

\subsubsection{Translation of Axes}

\begin{center}
\begin{tikzpicture}
  \tikzmath{
    \xmin = 0;
    \xmax = 3;
    \ymin = 0;
    \ymax = 3;
    \dx = 1.5;
    \dy = 1;
  }
  % original coordinate system
  \draw[-latex] ({(\xmin)-0.5},0) -- ({(\xmax)+0.5},0) node[below] {$x$};
  \draw[-latex] (0,{(\ymin)-0.5}) -- (0,{(\ymax)+0.5}) node[left] {$y$};
  \draw (0,0) node[below left] {$O$};
  % the transformed coordinate system
  \draw[red,-latex] ({(\xmin)-0.5},\dy) -- ({(\xmax)+0.5},\dy) node[below] {$x'$};
  \draw[red,-latex] (\dx,{(\ymin)-0.5}) -- (\dx,{(\ymax)+0.5}) node[left] {$y'$};
  \draw[red] (\dx,\dy) node[below left] {$O'$};
  \coordinate (P) at (2.5,2.5);
  \foreach \p/\placement in {P/above}{
    \fill[teal] (\p) circle (2pt);
    \draw (\p) node[\placement] {$\p$};
  }
\end{tikzpicture}
\end{center}

\[
  \begin{cases}
    x' = x - d_x\\
    y' = y - d_y
  \end{cases}
\]

\[
  \begin{pmatrix}
    x'\\
    y'\\
    1
  \end{pmatrix}
  =
  \begin{pmatrix}
    1 & 0 & -d_x\\
    0 & 1 & -d_y\\
    0 & 0 & 1
  \end{pmatrix}
  \begin{pmatrix}
    x\\
    y\\
    1
  \end{pmatrix}
\]

\[
  \begin{pmatrix}
    x\\
    y\\
    1
  \end{pmatrix}
  =
  \begin{pmatrix}
    1 & 0 & d_x\\
    0 & 1 & d_y\\
    0 & 0 & 1
  \end{pmatrix}
  \begin{pmatrix}
    x'\\
    y'\\
    1
  \end{pmatrix}
\]

\subsubsection{Rotation of Axes}

\begin{center}
\begin{tikzpicture}
  \tikzmath{
    \xmin = 0;
    \xmax = 3;
    \ymin = 0;
    \ymax = 3;
    \a = 15;
  }
  % original coordinate system
  \draw[-latex] ({(\xmin)-0.5},0) -- ({(\xmax)+0.5},0) node[below] {$x$};
  \draw[-latex] (0,{(\ymin)-0.5}) -- (0,{(\ymax)+0.5}) node[left] {$y$};
  \draw (0,0) node[below left] {$O$};
  % the transformed coordinate system
  \draw[rotate=\a,red,-latex] ({(\xmin)-0.5},0) -- ({(\xmax)+0.5},0) node[below] {$x'$};
  \draw[rotate=\a,red,-latex] (0,{(\ymin)-0.5}) -- (0,{(\ymax)+0.5}) node[left] {$y'$};
  % \draw (0,0) node[below left] {$O$};

  \coordinate (O) at (0,0);
  \coordinate (P) at (2.5,2.5);
  \coordinate (A) at (1,0);
  \coordinate (B) at (\a:1);

  \pic[draw,teal,"$\phi$",angle eccentricity=1.25,angle radius=8mm] {angle=A--O--P};
  \pic[draw,red,"$\theta$",angle eccentricity=1.25,angle radius=12mm] {angle=A--O--B};

  \draw[teal] (O) -- (P);
  \foreach \p/\placement in {P/above}{
    \fill[teal] (\p) circle (2pt);
    \draw (\p) node[\placement] {$\p$};
  }
\end{tikzpicture}
\end{center}

\[
  \begin{cases}
    x = r\cos\phi\\
    y = r\sin\phi
  \end{cases}
\]

\[
  \begin{cases}
    x' = r\cos(\phi-\theta)\\
    y' = r\sin(\phi-\theta)
  \end{cases}
\]

\[
  \begin{cases}
    x' = x\cos\theta+y\sin\theta\\
    y' = y\cos\theta-x\sin\theta
  \end{cases}
\]

\[
  \begin{pmatrix}
    x'\\
    y'\\
    1
  \end{pmatrix}
  =
  \begin{pmatrix}
    \cos\theta & \sin\theta & 0\\
    -\sin\theta & \cos\theta & 0\\
    0 & 0 & 1
  \end{pmatrix}
  \begin{pmatrix}
    x\\
    y\\
    1
  \end{pmatrix}
\]

\[
  \begin{pmatrix}
    x\\
    y\\
    1
  \end{pmatrix}
  =
  \begin{pmatrix}
    \cos\theta & -\sin\theta & 0\\
    \sin\theta & \cos\theta & 0\\
    0 & 0 & 1
  \end{pmatrix}
  \begin{pmatrix}
    x'\\
    y'\\
    1
  \end{pmatrix}
\]

\subsection{Transformations on Coordinates}

\subsubsection{Translation of Coordinates}

\begin{center}
\begin{tikzpicture}
  \tikzmath{
    \xmin = 0;
    \xmax = 3;
    \ymin = 0;
    \ymax = 3;
    \dx = 1.5;
    \dy = 1;
  }
  % original coordinate system
  \draw[-latex] ({(\xmin)-0.5},0) -- ({(\xmax)+0.5},0) node[below] {$x$};
  \draw[-latex] (0,{(\ymin)-0.5}) -- (0,{(\ymax)+0.5}) node[left] {$y$};
  \draw (0,0) node[below left] {$O$};

  \coordinate (P) at (1.5,0.5);
  \coordinate (P') at ($(P)+(\dx,\dy)$);

  \fill[teal] (P) circle (2pt);
  \draw (P) node[above] {$P(x,y)$};
  \fill[red] (P') circle (2pt);
  \draw (P') node[above] {$P'(x',y')$};
\end{tikzpicture}
\end{center}

\[
  \begin{cases}
    x' = x + d_x\\
    y' = y + d_y
  \end{cases}
\]

\[
  \begin{pmatrix}
    x'\\
    y'\\
    1
  \end{pmatrix}
  =
  \begin{pmatrix}
    1 & 0 & d_x\\
    0 & 1 & d_y\\
    0 & 0 & 1
  \end{pmatrix}
  \begin{pmatrix}
    x\\
    y\\
    1
  \end{pmatrix}
\]

\[
  \begin{pmatrix}
    x\\
    y\\
    1
  \end{pmatrix}
  =
  \begin{pmatrix}
    1 & 0 & -d_x\\
    0 & 1 & -d_y\\
    0 & 0 & 1
  \end{pmatrix}
  \begin{pmatrix}
    x'\\
    y'\\
    1
  \end{pmatrix}
\]

\subsubsection{Rotation of Coordinates}

\begin{center}
\begin{tikzpicture}
  \tikzmath{
    \xmin = 0;
    \xmax = 3;
    \ymin = 0;
    \ymax = 3;
    \a = 15;
  }
  % original coordinate system
  \draw[-latex] ({(\xmin)-0.5},0) -- ({(\xmax)+0.5},0) node[below] {$x$};
  \draw[-latex] (0,{(\ymin)-0.5}) -- (0,{(\ymax)+0.5}) node[left] {$y$};
  \draw (0,0) node[below left] {$O$};

  \coordinate (O) at (0,0);
  \coordinate (X) at (1,0);
  \coordinate (P) at (2.5,1.5);
  \coordinate (P') at($(O)!1.0!\a:(P)$); % rotate P around O by \a degrees

  \pic[draw,teal,"$\phi$",angle eccentricity=1.25,angle radius=8mm] {angle=X--O--P};
  \pic[draw,red,"$\theta$",angle eccentricity=1.25,angle radius=10mm] {angle=P--O--P'};

  \draw[teal] (O) -- (P) (O) -- (P');
  \fill[teal] (P) circle (2pt);
  \draw (P) node[right] {$P(x,y)$};
  \fill[red] (P') circle (2pt);
  \draw (P') node[above] {$P'(x',y')$};
\end{tikzpicture}
\end{center}

\[
  \begin{cases}
    x = r\cos\phi\\
    y = r\sin\phi
  \end{cases}
\]

\[
  \begin{cases}
    x' = r\cos(\phi+\theta)\\
    y' = r\sin(\phi+\theta)
  \end{cases}
\]

\[
  \begin{cases}
    x' = x\cos\theta-y\sin\theta\\
    y' = y\cos\theta+x\sin\theta
  \end{cases}
\]

\[
  \begin{pmatrix}
    x'\\
    y'\\
    1
  \end{pmatrix}
  =
  \begin{pmatrix}
    \cos\theta & -\sin\theta & 0\\
    \sin\theta & \cos\theta & 0\\
    0 & 0 & 1
  \end{pmatrix}
  \begin{pmatrix}
    x\\
    y\\
    1
  \end{pmatrix}
\]

\[
  \begin{pmatrix}
    x\\
    y\\
    1
  \end{pmatrix}
  =
  \begin{pmatrix}
    \cos\theta & \sin\theta & 0\\
    -\sin\theta & \cos\theta & 0\\
    0 & 0 & 1
  \end{pmatrix}
  \begin{pmatrix}
    x'\\
    y'\\
    1
  \end{pmatrix}
\]

\section{圆锥曲线的变换}

\subsection{圆锥曲线方程的一般形式}

设二次曲线的方程为:

\[
  ax^2 + bxy + cy^2 + dx + ey + f = 0
\]

记

\[
  Q = \begin{pmatrix}
    a & b/2 & d/2\\
    b/2 & c & e/2\\
    d/2 & e/2 & f
  \end{pmatrix}
\]

\[
  \mathbf{x} = (x,y,1)^T
\]

则:

\[
  \mathbf{x}^T Q \mathbf{x} = 0
\]

\subsection{圆锥曲线的切线方程}

设$\mathbf{x}_0$ 是圆锥曲线上的一点, 则过该点的切线方程是:

\[
  \mathbf{x}_0^T Q \mathbf{x} = 0
\]

\subsection{圆锥曲线的中心}

设 $A$ 是 矩阵 $Q$ 的代数余子式, 如果 $A_{33} = 0$, 则曲线是抛物线, 其中心在无穷远处;
否则曲线的中心是$\left(\dfrac{A_{31}}{A_{33}}, \dfrac{A_{32}}{A_{33}}\right)$.


\subsection{将圆锥曲线方程从一般形式化为标准形式}

因为抛物线是没有中心的, 对于任意圆锥曲线, 可以先进行旋转, 消除交叉项 $xy$, 然后再处理.

\subsubsection{旋转坐标系}

首先是旋转, 消除交叉项 $xy$, 旋转角度为 $\theta(-\pi/4 \leq \theta \leq \pi/4)$, 且:

\[
  \cot2\theta = \dfrac{1-\tan^2\theta}{2\tan\theta} = \dfrac{a-c}{b}
\]

令 $\tau = \cot2\theta$, 则:

\[
  \tan^2\theta + 2\tau \tan\theta -1 = 0
\]

取绝对值最小的根, 这样保证 $-\pi/4 \leq \theta \leq \pi/4$, 即:

\[
  \tan\theta = \begin{cases}
    - \tau + \sqrt{\tau^2+1}, & \text{if } \tau \geq 0;\\
    - \tau - \sqrt{\tau^2+1}, & \text{otherwise}.
  \end{cases}
\]

\begin{gather*}
  t = \tan\theta\\
  c = \cos\theta = \dfrac{1}{\sqrt{t^2+1}}\\
  s = \sin\theta = \tan\theta \cos\theta = tc
\end{gather*}

这样无需进行三角函数运算.

构造旋转矩阵:

\[
  R =
  \begin{pmatrix}
    c & -s & 0\\
    s & c & 0\\
    0 & 0 & 1
  \end{pmatrix}
\]

计算 $\tilde{Q}=R^TQR$ 就得到新的系数矩阵.

\subsubsection{平移坐标系}

然后根据系数矩阵 $\tilde{Q}$判断是否为抛物线, 即 $\tilde{Q}_{11} = 0$ 或 $\tilde{Q}_{22} = 0$. 如果是抛物线,
则将坐标系平移至顶点; 否则将坐标系平移至中心.

若$\tilde{Q}_{11} = 0$ 且 $\tilde{Q}_{22} \neq 0$, 则抛物线简化为:

\[
  cy^2+dx+ey+f=0
\] 

其中:

\begin{gather*}
  c = \tilde{Q}_{22}\\
  d = 2\tilde{Q}_{13}\\
  e = 2\tilde{Q}_{23}\\
  f = \tilde{Q}_{33}
\end{gather*}

则抛物线的顶点为: $\left(-\dfrac{e}{2c},\dfrac{e^2-4cf}{4cd}\right)$.

对于椭圆和双曲线, 计算其代数余子式 $A$,
中心为: $\left(\dfrac{A_{31}}{A_{33}},\dfrac{A_{32}}{A_{33}}\right)$.

构造平移矩阵:

\[
  T =
  \begin{pmatrix}
    1 & 0 & d_x\\
    0 & 1 & d_y\\
    0 & 0 & 1
  \end{pmatrix}
\]

其中,

\begin{gather*}
  d_x = \dfrac{A_{31}}{A_{33}},\\
  d_y = \dfrac{A_{32}}{A_{33}}.
\end{gather*}

计算 $\hat{Q}=T^T\tilde{Q}T$ 就得到化为标准形式的系数矩阵.

