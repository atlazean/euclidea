\begin{tikzpicture}
  \axes(-5:5, -5:5);

  \coordinate (F1) at (-3,-1);
  \coordinate (F2) at (3,1);
  \coordinate (P1) at (2,2);
  \coordinate (P) at (0,4);

  \tikzset{
    hyperbola/define={F1,F2,P1},
    hyperbola/domain=-1.25:1.25,
  }
  
  % 绘制双曲线
  \draw [purple, thick,hyperbola] node [above right] {hyperbola};

  % 获取存储的坐标(中心)
  \coordinate (C) at (\pgfkeysvalueof{/tikz/hyperbola/xcenter} cm, 
    \pgfkeysvalueof{/tikz/hyperbola/ycenter} cm);
  
  % 双曲线的顶点
  \coordinate (A) at 
    ($(C)+(\pgfkeysvalueof{/tikz/hyperbola/angle}:\pgfkeysvalueof{/tikz/hyperbola/a} cm)$);

  % 作辅助圆
  \draw[teal,circle={C,A}];
  
  % 作以 PF1 为直径的圆
  \coordinate (M) at ($(P)!.5!(F1)$);
  \draw[red,circle={M,P}];
  
  % 求两圆的交点(存储在 cc1 和 cc2 中)
  \tikzset{circle-circle={C,A,M,P},}
  \coordinate[affine={P,cc1,1.75}] (Q);
  \coordinate[affine={P,cc2,2}] (R);

  % 作出切线
  \draw[violet,thick] (P) -- (Q) (P) -- (R);

  \foreach \p/\placement in {F1/left,F2/right,
  P/above right,Q/left,R/below right}{
    \fill[red] (\p) circle (2pt);
    \draw (\p) node[\placement] {$\p$};
  }

\end{tikzpicture}