\documentclass[tikz,border=10pt]{article}
\usepackage{tikz}

\ProvidesFile{tikzlibrarymc.code.tex}[2025/03/05 v1.0.0 A tikz library for numerical linear algebra]

% 提高数值计算精度
% https://tex.stackexchange.com/questions/521857/tikz-fpu-seems-to-be-inaccurate
\RequirePackage{xfp}

\usetikzlibrary{math,calc,quotes}

% ===============================================
% Caveats
% ===============================================
% tikzmath 主要用于处理数值计算和变量赋值,支持基本的数学运算和条件语句,但不支持数组或列表作为函数参数。
% tikzmath 内部要以分号分隔, 可以有注释, 但不能有空行
% for 语句没有类似  stop, continue 之类的中断循环的语句
% for 的第一次总是执行的, 不对循环变量作判断, 要限制循环变量的范围
% 整数"下标"(非真正下标), 必须定义为 int, 否则默认为浮点数, 无法找到定义的宏 
% 函数内部对变量的修改不会传导至外部
% 函数定义时, 参数之间不要有空格
% return 语句不会中断函数内部下面语句的执行
% 函数参数个数最大为 9
% 函数参数不能直接使用数组。
% 极难调试
\tikzmath{
  function show(\m,\n) {
    int \i, \j;
    for \i in {1,...,\m} {
      for \j in {1,...,\n} {
        print{a[\i,\j]=\a{\i,\j}\ \ };
      };
      print{\\};
    };
  };
  % Determinant calculation using Gaussian elimination
  function det(\n) {
    \result = 1.0;
    int \i, \j, \k;
    int \stop;
    \stop = 0;
    % for 的第一项总是执行的,要限制范围
    for \k in {1,...,\n-1} {
      if \stop == 0 then {
        \pivotRow = \k;
        \pivot = abs(\a{\k,\k});
        for \i in {\k+1,...,\n} {
          if \pivot < abs(\a{\i,\k}) then {
            \pivotRow = \i;
            \pivot = abs(\a{\i,\k});
          };
        };
        if \pivot < 0.00001 then { 
          \stop = 1;
          \result = 0.0;
        };
        if \stop == 0 then {
          % swap
          if \pivotRow != \k then {
            \result = -\result;
            for \j in {\k,...,\n}{
              \temp = \a{\k,\j};
              \a{\k,\j} = \a{\pivotRow,\j};
              \a{\pivotRow,\j} = \temp;
            };
          };
          % eliminate
          for \i in {\k+1,...,\n}{
            \factor = \fpeval{\a{\i,\k} / \a{\k,\k}};
            for \j in {\k,...,\n}{
              \a{\i,\j} = \fpeval{\a{\i,\j} - \factor * \a{\k,\j}};
            };
          };
          \result = \fpeval{\result * \a{\k,\k}};
        };% stop
      };% stop
    };
    \result = \fpeval{\result * \a{\n,\n}};
    return \result;
  };
  function det3(\a1,\a2,\a3,\b1,\b2,\b3,\c1,\c2,\c3) {
    \a{1,1} = \a1; \a{1,2} = \a2; \a{1,3} = \a3;  
    \a{2,1} = \b1; \a{2,2} = \b2; \a{2,3} = \b3; 
    \a{3,1} = \c1; \a{3,2} = \c2; \a{3,3} = \c3; 
    % show(3,3);
    return det(3);
  };
  % The (i, j) minor is the determinant of the submatrix formed 
  % by deleting the i-th row and j-th column. 
  % The (i, j) cofactor is obtained by multiplying 
  % the minor by (−1)^(i + j).
  function minor(\n,\r,\c) {
    int \i, \j, \nexti, \nextj;
    for \i in {1,...,\n} {
      for \j in {1,...,\n} {
        if \i != \r && \j != \c then{
          if \i < \r then {
            \nexti = \i;
          } else {
            \nexti = \i - 1;
          };
          if \j < \c then {
            \nextj = \j;
          } else {
            \nextj = \j - 1;
          };
          \a{\nexti,\nextj} = \a{\i,\j};
        };%if
      };
    };%for
    % print{minor:\\};
    % show(\n-1,\n-1);
    int \m;
    \m = \n - 1;% 必须使用临时变量
    return det(\m);% 不能写成 return det(\n-1);
  };
  function cofactor(\n,\r,\c) {
    return (-1)^(\r+\c)*minor(\n,\r,\c);
  };
  % Solving linear systems with Gaussian-Jordan elimination
  % 由于 tikzmath 函数不能返回数组 也不能修改外部变量
  % 每次只返回第 [row,col] 元素, 求解 n * m 矩阵, 则需调用 n * m 次, 效率很低!
  % 初始时 矩阵为增广矩阵, Size: n * (n + m)
  function solve(\n,\m,\r,\c) {
    % show(\n,\n+\m);
    int \i, \j, \k;
    int \stop;
    \stop = 0;
    \result = 0.0;
    % for 的第一项总是执行的,要限制范围
    for \k in {1,...,\n} {
      if \stop == 0 then {
        \pivotRow = \k;
        \pivot = abs(\a{\k,\k});
        if \k < \n then {
          for \i in {\k+1,...,\n} {
            if \pivot < abs(\a{\i,\k}) then {
              \pivotRow = \i;
              \pivot = abs(\a{\i,\k});
            };
          };
        };
        if \pivot < 0.00001 then { 
          \stop = 1;
        };
        if \stop == 0 then {
          % swap
          if \pivotRow != \k then {
            for \j in {\k,...,\n+\m}{
              \temp = \a{\k,\j};
              \a{\k,\j} = \a{\pivotRow,\j};
              \a{\pivotRow,\j} = \temp;
            };
          };
          % eliminate
          int \next;
          for \next in {0,...,\n-1}{
            %先处理第 k 行(i.e. \next=0), 将主元归一化, 然后消去其它行
            \i =int(mod(\k + \next, \n)) ;
            if \i == 0 then {
              \i = \n;
            };
            \temp = \a{\i,\k};
            for \j in {\k,...,\n+\m}{
              if \i == \k then {
                \a{\i,\j} = \fpeval{\a{\i,\j} / \temp};
              } else {%注意不能少else
                \a{\i,\j} = \fpeval{\a{\i,\j} - \temp * \a{\k,\j}};
              };
            };
          };
        };% stop
      };% stop
    };%for
    % show(\n,\n+\m);
    int \cumn;
    % 默认相加结果为浮点数, \cumn 定义为 int, 自动转整数
    \cumn = \c + \n;
    return \a{\r,\cumn};
  };%function solve
  % 逆矩阵元素
  function inverse(\n,\r,\c){
    % create augumented matrix
    int \i,\j;
    for \i in {1,...,\n}{
      for \j in {\n+1,...,2*\n}{
        if \i == \j - \n then {
          \a{\i,\j} = 1.0;
        } else {
          \a{\i,\j} = 0.0;
        };
      };
    };
    return solve(\n,\n,\r,\c);
  };
}


\begin{document}

\tikzmath{
  print{Determinant calculation using Gaussian elimination:\\};
  \d = det3(7,5,3,
            2,1,6,
            1,2,3);
  print{n=3: det = \d \\};
  % n = 4
  \a{1,1} = 2; \a{1,2} = 1; \a{1,3} = 7; \a{1,4} = -1; 
  \a{2,1} = -1; \a{2,2} = 2; \a{2,3} = 4; \a{2,4} = 3; 
  \a{3,1} = 2; \a{3,2} = 1; \a{3,3} = 0; \a{3,4} = -1; 
  \a{4,1} = 3; \a{4,2} = 2; \a{4,3} = 2; \a{4,4} = 1;
  show(4,4);
  \d = det(4);
  print{n=4: det = \d \\};%-70
  % n = 4
  \a{1,1} = 1; \a{1,2} = 7; \a{1,3} = 2; \a{1,4} = 4; 
  \a{2,1} = 1; \a{2,2} = 5; \a{2,3} = 2; \a{2,4} = 4; 
  \a{3,1} = 3; \a{3,2} = 0; \a{3,3} = 1; \a{3,4} = 0; 
  \a{4,1} = 2; \a{4,2} = 1; \a{4,3} = 5; \a{4,4} = -3;
  show(4,4);
  \d = det(4);
  print{n=4: det = \d \\};%-134
  % n = 5
  \a{1,1} = 4; \a{1,2} = 0; \a{1,3} = -7; \a{1,4} = 3; \a{1,5} = -5;
  \a{2,1} = 0; \a{2,2} = 0; \a{2,3} = 2; \a{2,4} = 0; \a{2,5} = 0; 
  \a{3,1} = 7; \a{3,2} = 3; \a{3,3} = -6; \a{3,4} = 4; \a{3,5} = -8; 
  \a{4,1} = 5; \a{4,2} = 0; \a{4,3} = 5; \a{4,4} = 2; \a{4,5} = -3;
  \a{5,1} = 0; \a{5,2} = 0; \a{5,3} = 9; \a{5,4} = -1; \a{5,5} = 2;
  show(5,5);
  \d = det(5);
  print{n=5: det = \d \\};%6
  print{Compute the minor and cofactor:\\};
  \a{1,1} = 7; \a{1,2} = 5; \a{1,3} = 3;
  \a{2,1} = 2; \a{2,2} = 1; \a{2,3} = 6; 
  \a{3,1} = 1; \a{3,2} = 2; \a{3,3} = 3;
  show(3,3);
  int \i,\j;
  \i = 2; \j = 1;
  \d = det(3);
  print{n=3: det = \d \\};
  % for \i in {1,2,3} {
    \m = minor(3,\i,\j);
    \c = cofactor(3,\i,\j);
    print{minor[\i,\j] = \m \\cofator[\i,\j] = \c \\};
  % };
  % 计算经过三点的圆
  % A(x^2+y^2)+Bx+Cy+D=0
  % (1,0), (0,1), (-1,0)
  function squaresum(\x,\y) {
    return (\x)^2+(\y)^2;%可能为负 注意加括号
  };
  \x1 = 1;  \y1 = 0;
  \x2 = 0;  \y2 = 1;
  \x3 = 2;  \y3 = 1;
  % 第一行是 placeholder
  \a{1,1} = 0.0;                \a{1,2} = 0.0; \a{1,3} = 0.0; \a{1,4} = 0.0;
  \a{2,1} = squaresum(\x1,\y1); \a{2,2} = \x1; \a{2,3} = \y1; \a{2,4} = 1.0;
  \a{3,1} = squaresum(\x2,\y2); \a{3,2} = \x2; \a{3,3} = \y2; \a{3,4} = 1.0;
  \a{4,1} = squaresum(\x3,\y3); \a{4,2} = \x3; \a{4,3} = \y3; \a{4,4} = 1.0;
  \A = cofactor(4,1,1);
  \B = cofactor(4,1,2);
  \C = cofactor(4,1,3);
  \D = cofactor(4,1,4);
  \B = \B / \A;
  \C = \C / \A;
  \D = \D / \A;
  \A = 1.0;
  show(4,4);
  print{the circle through (\x1,\y1),(\x2,\y2),(\x3,\y3) is:\\$A(x^2+y^2)+Bx+Cy+D=0$, \\wherein\\ A = \A, B = \B, C = \C, D = \D\\};
  % 计算经过五点的圆锥曲线
  % Ax^2+Bxy+Cy^2+Dx+Ey+F=0
  \x1 = 0;  \y1 = 2;
  \x2 = 3;  \y2 = 1;
  \x3 = 2;  \y3 = -2;
  \x4 = -2;  \y4 = -1;
  \x5 = -2;  \y5 = 1;
  % 第一行是 placeholder
  % \a{1,1} = 0.0;     \a{1,2} = 0.0;     \a{1,3} = 0.0;     \a{1,4} = 0.0; \a{1,5} = 1.0; \a{1,6} = 1.0;
  % \a{2,1} = (\x1)^2; \a{2,2} = \x1*\y1; \a{2,3} = (\y1)^2; \a{2,4} = \x1; \a{2,5} = \y1; \a{2,6} = 1.0; 
  % \a{3,1} = (\x2)^2; \a{3,2} = \x2*\y2; \a{3,3} = (\y2)^2; \a{3,4} = \x2; \a{3,5} = \y2; \a{3,6} = 1.0; 
  % \a{4,1} = (\x3)^2; \a{4,2} = \x3*\y3; \a{4,3} = (\y3)^2; \a{4,4} = \x3; \a{4,5} = \y3; \a{4,6} = 1.0; 
  % \a{5,1} = (\x4)^2; \a{5,2} = \x4*\y4; \a{5,3} = (\y4)^2; \a{5,4} = \x4; \a{5,5} = \y4; \a{5,6} = 1.0; 
  % \a{6,1} = (\x5)^2; \a{6,2} = \x5*\y5; \a{6,3} = (\y5)^2; \a{6,4} = \x5; \a{6,5} = \y5; \a{6,6} = 1.0; 
  int \i;
  for \i in {1,...,5}{
    \a{\i,1} = (\x{\i})^2;
    \a{\i,2} = \x{\i} * \y{\i};
    \a{\i,3} = (\y{\i})^2;
    \a{\i,4} = \x{\i};
    \a{\i,5} = \y{\i};
    \a{\i,6} = 1.0;
  };
  % 第 6 行是占位
  for \i in {1,...,6}{
    \a{6,\i} = 0.0;
  };
  \A = cofactor(6,6,1);
  \B = cofactor(6,6,2);
  \C = cofactor(6,6,3);
  \D = cofactor(6,6,4);
  \E = cofactor(6,6,5);
  \F = cofactor(6,6,6);
  show(6,6);
  print{the conic through (\x1,\y1),(\x2,\y2),(\x3,\y3),(\x4,\y4),(\x5,\y5) is:\\$Ax^2+Bxy+Cy^2+Dx+Ey+F=0$, \\wherein\\ A = \A, B = \B, C = \C, D = \D, E = \E, F = \F};
}

\end{document}