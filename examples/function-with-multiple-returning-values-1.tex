% https://tex.stackexchange.com/questions/510418/tikz-declare-function-with-multiple-outputs

% 在 TikZ 中,函数本身并没有直接的“返回多个值”的机制,因为 TikZ 是一种基于 TeX 的绘图语言,主要用于描述图形,
% 而不是像编程语言那样处理多返回值逻辑。不过,通过一些技巧,你可以在 TikZ 中模拟类似“返回多个值”的效果。

% \pgfmathsmuggle 是 PGF/TikZ 数学引擎中的一个命令,
% 它主要用于在 \pgfmathparse 或 \pgfmathdeclarfunction 等命令内部“偷运”计算结果到外部环境。
% 理解这个命令的关键在于理解 PGF/TikZ 数学引擎的工作方式和作用域。

\documentclass[tikz,border=3mm]{standalone}
\usepackage{pgfplots}

\pgfplotsset{compat=1.16}

\begin{document}

\pgfmathdeclarefunction{myfun}{2}{% 
\begingroup% 
% \typeout{==========================}
% \typeout{#1,#2}
% \typeout{==========================}
\pgfmathsetmacro{\myx}{#1+#2}% 
\pgfmathsetmacro{\myy}{#1-#2}% 
\edef\pgfmathresult{{\myx}{\myy}}%
\pgfmathsmuggle\pgfmathresult\endgroup} 

\pgfmathdeclarefunction{xcomp2}{2}{% x component of a 2-vector 
\begingroup% 
\pgfmathparse{#1}% 
\pgfmathsmuggle\pgfmathresult\endgroup} 

\pgfmathdeclarefunction{ycomp2}{2}{% y component of a 2-vector 
\begingroup% 
\pgfmathparse{#2}% 
\pgfmathsmuggle\pgfmathresult\endgroup} 

\begin{tikzpicture}
\begin{axis}
\addplot [only marks, samples=50] ({xcomp2(myfun(rnd,rnd))},{ycomp2(myfun(rnd,rnd))});
\end{axis}
\end{tikzpicture}

\end{document}