\chapter{圆锥曲线 Conics}

尽管 tikz 内置的 \verbum{ellipse} 和 \verbum{parabola} 绘制椭圆和抛物线,
这里定义了 \mintinline{latex}|\ellipse| 和 \mintinline{latex}|\parabola|.

% ---------------------------------
\section{椭圆 Ellipse}

\subsection{已知半长轴长 a 和半短轴长 b 绘图}

\emph{调用方式}

\begin{tcolorbox}{}
\mint{latex}{\ellipse [options] (a,b)}
\end{tcolorbox}

\emph{参数说明}

\begin{description}
  \item[a, b] 半长轴长 (semi-major axis) 和半短轴长 (semi-minor axis), 默认单位为 cm, 可指定单位, 如 \verbum{(4cm, 3cm)}
\end{description}

返回中心为原点的椭圆曲线: $\dfrac{x^2}{a^2}+\dfrac{y^2}{b^2}=1$.

\begin{remark*}
当指定椭圆曲线 (ellipse) 的 \verbum{domain} (default: \verbum{domain=-180:180})时, \verbum{domain} 是下列参数方程中 $t$ 的取值范围:
\begin{align*}
  \begin{cases}
  x = a \cos t,\\y = b \sin t
  \end{cases}
\end{align*}
\end{remark*}

\emph{示例}

使用 tikz 内置曲线:

\showcode{snippets/conics/ellipse1.tex}

使用 \mintinline{latex}|\ellipse| 命令:

\showcode{snippets/conics/ellipse2.tex}

% ---------------------------------
\subsection{由两焦点和一点确定椭圆}

\emph{调用方式}

\begin{minted}{latex}
  `ellipse/define = {F1,F2,P}`: define an ellipse with two foci and a point.
  `ellipse`: creates the ellipse path.
  `ellipse/directrix/scale=k`: set the scale of directrices.
  `ellipse/directrix`: create the directrices.
  `ellipse/axis/scale=k`: set the scale of axes.
  `ellipse/axis`: create the axes.
  `ellipse/a`, `ellipse/b`, `ellipse/c`, `ellipse/e`: ellipse semimajor-axis, semiminor-axis, linear eccentricity, eccentricity.
\end{minted}

注意: \mint{latex}{ellipse/define = {F1,F2,P}} 必须单独使用 tikzset 定义才能访问 /tikz/ellipse 下面的变量, 这和作用域有关.

\emph{示例}

\showcode{snippets/conics/ellipse3.tex}

% ---------------------------------
\section{椭圆弧 Arc}

\emph{基本用法}

在 TikZ 中绘制圆弧可以通过 arc 操作 来实现, 将椭圆的一部分添加到当前路径中.

\showcode{snippets/conics/arc1.tex}

圆弧从点 (2,0) 开始, 以 0 度(由起始角度指定)结束, 以 300 度(由结束角度指定)结束.

为了绘制圆的一部分, 我们使用与前一个相同的语法, 并且不提供参数 x radius 和 y radius, 而是仅提供参数radius.

\showcode{snippets/conics/arc2.tex}

\begin{itemize}

\item 黄色扇区是通过 arc 操作绘制的, 为了得到圆的一部分, 我们只定义 radius 而不是  x radius 和 y radius 参数. 

\item 蓝色扇区的绘制方式与黄色扇区相同, 但在本例中我们指定了 delta angle 而不是 end angle, 后者等于start angle + delta angle.

\end{itemize}

\emph{简短语法}

上图中的弯曲箭头是使用 arc 操作的较短语法绘制的, 对应于圆的一部分:

arc(start angle:end angle: radius)

对于椭圆, 我们使用以下语法: 

arc(start angle:end angle: x radius and y radius)

然而, 这种语法不直观且难以阅读, 因此一般情况下应首选普通语法(见 PGF 手册).

% ---------------------------------
\section{抛物线 Parabola}

\subsection{已知方程系数 a, b 和 c 绘图}

\emph{调用方式}

\begin{tcolorbox}{}
\mint{latex}{\parabola [options] (a,b,c)}
\end{tcolorbox}

\emph{参数说明}

\begin{description}
  \item[a,b,c] 二次函数的系数
\end{description}

返回抛物线: $y=ax^2+bx+c$.

\emph{示例}

使用 tikz 内置曲线:

\showcode{snippets/conics/parabola1.tex}

使用 \mintinline{latex}|\parabola| 命令:

\showcode{snippets/conics/parabola2.tex}

% ---------------------------------
\subsection{由焦点和顶点确定抛物线}

\emph{调用方式}

\begin{minted}{latex}
  `parabola/define = {F,P}`: define a parabola with the focus and vertex.
  `parabola/domain=t1:t2`: set the domain for parametric equation: x = a*t^2, y = 2a*t, where `a` is the distance from the focus to the vertex.
  `parabola`: creates the parabola path.
  `parabola/directrix/scale=k`: set the scale of directrices.
  `parabola/directrix`: create the directrix.
  `parabola/axis/scale=k`: set the scale of axis.
  `parabola/axis`: create the axis.
  `parabola/a`, `parabola/e`: 1/4 parabola latus rectum, eccentricity (1.0).
\end{minted}

\emph{示例}

\showcode{snippets/conics/parabola3.tex}

% ---------------------------------
\section{双曲线 Hyperbola 与渐近线 Asymptote}

\subsection{已知半实轴长 a 和半虚轴长 b 绘图}

\emph{调用方式}

\begin{tcolorbox}{}
\mint{latex}{\hyperbola [options] (a,b);}
\end{tcolorbox}

或

\begin{tcolorbox}{}
\mint{latex}{\asymptote [options] (a,b);}
\end{tcolorbox}

\emph{参数说明}

\begin{description}
  \item[a, b] 半实轴长 (semi-major axis) 和半虚轴长 (semi-minor axis), 默认单位为 cm, 可指定单位, 如 \verbum{(4cm, 3cm)}
\end{description}

分别返回中心在原点的双曲线: $\dfrac{x^2}{a^2}-\dfrac{y^2}{b^2}=1$ 和渐近线: $y = \pm \dfrac{b}{a}x$.

\emph{示例}

\showcode{snippets/conics/hyperbola1.tex}

\begin{remark*}
当指定绘制双曲线 (hyperbola) 的 \verbum{domain} (default: \verbum{domain=-1.5:1.5})时, \verbum{domain} 是下列双曲线参数方程中 $t$ 的取值范围:
\begin{align*}
  \begin{cases}
  x = \pm a \cosh t,\\y = b \sinh t
  \end{cases}
\end{align*}

$t$ 的几何意义: 射线出原点交单位双曲线 $x^2-y^2=1$ 于 $(x = \cosh t, y = \sinh t)$,
$t$ 是射线,双曲线和 $x$ 轴围成的面积的二倍. 对于双曲线上位于 $x$ 轴下方的点, 这个面积被认为是负值.

当指定绘制渐进线 (asymptote) 的 \verbum{domain} (default: \verbum{domain=-2:2})时, \verbum{domain} 是下列直线方程中 $x$ 的取值范围:
\begin{align*}
  y = \pm \dfrac{b}{a} x
\end{align*}
\end{remark*}

\emph{示例}

\showcode{snippets/conics/hyperbola2.tex}

也可使用 pgfplots 底层函数来绘制双曲线:

\emph{示例}

\showcode{snippets/conics/hyperbola3.tex}

% ---------------------------------
\subsection{由两焦点和一点确定双曲线}

\emph{调用方式}

\begin{minted}{latex}
  `hyperbola/define = {F1,F2,P}`: define a hyperbola with two foci and a point.
  `hyperbola/domain=t1:t2`: set the domain for parametric equation: x = a*cosh(t), y = b*sinh(t).
  `hyperbola`: creates the hyperbola path.
  `hyperbola/directrix/scale=k`: set the scale of directrices.
  `hyperbola/directrix`: create the directrices.
  `hyperbola/axis/scale=k`: set the scale of axes.
  `hyperbola/axis`: create the axes.
  `hyperbola/a`, `hyperbola/b`, `hyperbola/c`, `hyperbola/e`: hyperbola semimajor-axis, semiminor-axis, linear eccentricity, eccentricity.
\end{minted}

\emph{示例}

\showcode{snippets/conics/hyperbola4.tex}

% ---------------------------------
\section{过椭圆外一点作椭圆的切线 Tangents to an Ellipse}

过椭圆外一点作椭圆的切线的方法有: 利用椭圆的光学性质作图; 
或者根据射影几何理论, 采用极点极线法.此处采用前者.

方法 1: 

\begin{enumerate}
  \item 作出椭圆的辅助圆 (以椭圆中心为圆心, 以长半轴长 $a$ 为半径的圆);
  \item 作出一个以 $PF_1$ 或 $PF_2$ 为直径的圆;
  \item 两圆交于 $Q,R$, 则直线 $PQ,PR$ 即为所求.
\end{enumerate}

\showcode{snippets/conics/tangents-of-ellipse-1.tex}

方法 2: 

\begin{enumerate}
  \item 作出一个以 $F_1$ 为圆心, $2a$ 为半径的圆;
  \item 作出一个以 $F_2$ 为圆心, $PF_2$ 为半径的圆;
  \item 两圆交于 $Q,R$, 则 $F_2Q,F_2R$ 的中垂线即为所求.
\end{enumerate}

\showcode{snippets/conics/tangents-of-ellipse-2.tex}

% ---------------------------------
\section{过双曲线外一点作双曲线的切线 Tangents to a Hyperbola}

类似椭圆的切线作法, 对于双曲线同样有下面的方法:

\showcode{snippets/conics/tangents-of-Hyperbola-1.tex}

\showcode{snippets/conics/tangents-of-Hyperbola-2.tex}

% ---------------------------------
\section{过抛物线外一点作抛物线的切线 Tangents to a Parabola}

抛物线的切线作法如下:

\begin{enumerate}
  \item 以点 $P$ 为圆心, $PF$ 为半径作圆, 交准线将于 $Q,R$;
  \item $FQ,FR$ 的中垂线即为所求.
\end{enumerate}

% ---------------------------------
\section{过五点作圆锥曲线 Five Points Determine a Conic}

Braikenridge-Maclaurin 定理指出,如果通过六边形的相对侧的三对线的三个交点位于一条直线上,则六边形的六个顶点位于圆锥上. 这实际上是 Pascal 定理的逆定理.

\emph{示例}

\showcode{snippets/conics/five-points-determine-a-conic.tex}

% ---------------------------------
\section{与五条直线相切的圆锥曲线 Five Tangents Determine a Conic}

下面的作图采用 Brianchon 定理中退化的外切六边形, 求出各边的切点, 然后根据 5 点确定一个
圆锥曲线 (略).

\emph{示例}

\showcode{snippets/conics/five-tangents-determine-a-conic.tex}


