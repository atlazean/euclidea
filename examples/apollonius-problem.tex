\documentclass{standalone}
\usepackage{tikz}
\usetikzlibrary{euclidea,intersections}

\begin{document}
\begin{tikzpicture}
  % 1. 任意给定三个圆 O1, O2, O3
  \coordinate (O1) at (-3,3);
  \coordinate (A1) at ($(O1)+(30:3)$);
  \coordinate (O2) at (-.5,0);
  \coordinate (A2) at ($(O2)+(30:.75)$);
  \coordinate (O3) at (1,3);
  \coordinate (A3) at ($(O3)+(30:1.2)$);
  \draw[name path=circle1, circle={O1,A1}];
  \draw[name path=circle2, circle={O2,A2}];
  \draw[name path=circle3, circle={O3,A3}];
  % 2. 作出三个圆的根心 G
  \path[name path=axis1,radical axis={O2,A2,O3,A3}];
  \path[name path=axis2,radical axis={O3,A3,O1,A1}];
  \path[name intersections={of=axis1 and axis2,by=G}];
  % 3. 作出外位似中心和外位似轴 RS
  \coordinate[external center={O2,A2,O3,A3}] (R);
  \coordinate[external center={O3,A3,O1,A1}] (S);
  % 4. 以外位似轴 RS 为极线, 分别以圆 O1, O2, O3 为反演圆, 
  % 做出三个极点 P1, P2, P3
  \coordinate (U) at ($(R)!(O1)!(S)$);
  \coordinate (V) at ($(R)!(O2)!(S)$);
  \coordinate (W) at ($(R)!(O3)!(S)$);
  \coordinate[circle inverse={O1,A1,U}] (P1);
  \coordinate[circle inverse={O2,A2,V}] (P2);
  \coordinate[circle inverse={O3,A3,W}] (P3);
  % 5. 作直线 GP1 交圆 O1 于 U1, U2; 作直线 GP2 交圆 O2 于 V1, V2; 
  % 作直线 GP3 交圆于 W1, W2; 
  \draw[name path=l1,end modifier=4.5,parallel={G,P1,G}];
  \draw[name path=l2,end modifier=1.5,parallel={G,P2,G}];
  \draw[name path=l3,end modifier=1.75,parallel={G,P3,G}];
  \path[name intersections={of=circle1 and l1,sort by=l1,by={U1,U2}}];
  \path[name intersections={of=circle2 and l2,sort by=l2,by={V1,V2}}];
  \path[name intersections={of=circle3 and l3,sort by=l3,by={W1,W2}}];
  % 6. 作 U1 -- V1 -- W1 和 U2 -- V2 -- W2 的外接圆
  \coordinate[circumcenter={U1,V1,W1}] (O4);
  \coordinate[circumcenter={U2,V2,W2}] (O5);
  \draw[thick,purple,circle={O4,U1}];
  \draw[thick,purple,circle={O5,U2}];
  \foreach \p/\placement in {O1/above, O2/above, O3/above, 
  P1/above, P2/above, P3/above, 
  R/above, S/above, 
  G/above} {
    \fill[red] (\p) circle (2pt);
    \draw (\p) node[\placement] {$\p$};
  }

\end{tikzpicture}
\end{document}