\chapter{圆 Circles}

\section{圆周角定理 Inscribed angle theorem}

\begin{theorem}[圆周角定理,Inscribed angle theorem\label{inscribed}]
  在同圆或等圆中,同弧或等弧所对圆周角等于其所对的圆心角的一半.
\end{theorem}

推论:
\setlist{nolistsep}
\begin{enumerate}
  \item 若两圆周角定点在弦所在直线的同一边, 则两圆周角相等.
  \item 若两圆周角定点分别在弦所在直线的两边, 则圆周角互补.
  \item 若两圆周角定点在一条直径上, 则圆周角恒等于$90^\circ$.
\end{enumerate}

\section{泰勒斯定理 Thales' theorem}

\begin{theorem}[泰勒斯定理,Thales' theorem\label{thales}]
  若$A,B,C$是圆周上的三点,且$AC$是该圆的直径,那么$\angle ABC$为直角.或者说,直径所对的圆周角是直角.
\end{theorem}

泰勒斯定理以古希腊思想家,科学家,哲学家泰勒斯(Thales of Miletus)的名字命名. 该定理在欧几里得《几何原本》(Euclid's {\itshape Elements})第三卷中被提到并证明.

泰勒斯定理的逆定理同样成立,即:

\begin{theorem}[泰勒斯定理的逆定理]
  直角三角形中,直角的顶点在以斜边为直径的圆上.
\end{theorem}

\section{弦切角定理 Alternate segment theorem}

\begin{theorem}[弦切角定理,Alternate segment theorem]
  在任何圆中, 弦切角的度数等于通过一个弦和切线之间的端点的夹角, 亦是所夹弧所对的圆周角(圆心角的一半).
\end{theorem}

\begin{figure}[!htb]
\begin{center}
\begin{tikzpicture}[scale=.55]
  \tikzmath{
    \r = 5;
  }
  \coordinate (O) at (0,0);
  \coordinate (A) at (-90:\r);
  \coordinate (B) at (2*\r,-\r);
  \coordinate (C) at (55:\r);
  \coordinate (D) at (-10:\r);
  \draw[thick,red] (O) circle (\r);
  \draw[thick] (A) -- (B);
  \draw[thick] (A) -- (C);
  \draw[thick] (A) -- (D);
  \draw[thick] (C) -- (D);
  \foreach \p in {O,A,B,C,D}
    \fill[red] (\p) circle (2pt);
  \pic[draw,blue,double,angle radius=12pt] {angle=A--C--D};
  \pic[draw,blue,double,angle radius=12pt] {angle=B--A--D};
  \draw (A) node[below] {$A$};
  \draw (B) node[below] {$B$};
  \draw (C) node[above] {$C$};
  \draw (D) node[above right] {$D$};
  \draw (O) node[below] {$O$};
\end{tikzpicture}
\end{center}
\caption{弦切角定理}
\end{figure}

\section{圆幂定理 Circle power theorem}

\begin{definition}[圆幂,Circle Power,Power of a point]
给点半径为$r$,圆心为$O$的圆,平面上任意一点$P$的圆幂$\Pi(P)={OP}^2-r^2$.
\end{definition}
这个定义是 Jakob Steiner
\footnote{\url{https://mathworld.wolfram.com/CirclePower.html}} 
\footnote{\url{https://en.wikipedia.org/wiki/Power_of_a_point}}
在 1826 年引入的, 根据定义:
\begin{enumerate}
  \item 如果$P$在圆外, 则$\prod(P)>0$.
  \item 如果$P$在圆上, 则$\prod(P)=0$.
  \item 如果$P$在圆内, 则$\prod(P)<0$.
\end{enumerate}

\begin{theorem}[圆幂定理,Circle power theorem]
给定一个圆$\Gamma$以及一点$P$, 从该点引出两条割线, 
分别与$\Gamma$相交于$A,B$以及$C,D$, 则有
\[PA \cdot PB = PC \cdot PD\]
\end{theorem}
这个乘积, 是$P$对于$\Gamma$的圆幂(Circle power), 故定理以此为名.
圆幂定理有三个变体, 分别是“相交弦定理 intersecting chords theorem”, “割线定理Intersecting secants theorem”及“切割线定理”.

\begin{figure}[!htb]
\begin{center}
\begin{tikzpicture}[scale=.55]
  \tikzmath{
    \r = 5;
  }
  \coordinate (O) at (0,0);
  \coordinate (A) at (40:\r);
  \coordinate (B) at (190:\r);
  \coordinate (C) at (-20:\r);
  \coordinate (D) at (100:\r);
  \coordinate[intersect={A,B,C,D}] (P);
  \draw[thick,red] (O) circle (\r);
  \draw[thick] (A) -- (B); 
  \draw[thick] (C) -- (D); 
  \draw[dashed] (B) -- (D);
  \draw[dashed] (A) -- (C);
  \foreach \p in {O,A,B,C,D,P}
    \fill[red] (\p) circle (2pt);
  \draw (A) node[above right] {$A$};
  \draw (B) node[below left] {$B$};
  \draw (C) node[below right] {$C$};
  \draw (D) node[above] {$D$};
  \draw (P) node[below] {$P$};
  \draw (O) node[below] {$O$};
\end{tikzpicture}
\end{center}
\caption{相交弦定理}
\end{figure}

\begin{figure}[!htb]
\begin{center}
\begin{tikzpicture}[scale=.55]
  \tikzmath{
    \r = 5;
  }
  \coordinate (O) at (0,0);
  \coordinate (A) at (40:\r);
  \coordinate (B) at (-20:\r);
  \coordinate (C) at (100:\r);
  \coordinate (D) at (190:\r);
  \coordinate[intersect={A,B,C,D}] (P);
  \draw[thick,red] (O) circle (\r);
  \draw[thick] (P) -- (B); 
  \draw[thick] (P) -- (D); 
  \draw[dashed] (B) -- (D);
  \draw[dashed] (A) -- (C);
  \foreach \p in {O,A,B,C,D,P}
    \fill[red] (\p) circle (2pt);
  \draw (A) node[above right] {$A$};
  \draw (B) node[below right] {$B$};
  \draw (C) node[above left] {$C$};
  \draw (D) node[below left] {$D$};
  \draw (P) node[above] {$P$};
  \draw (O) node[below] {$O$};
\end{tikzpicture}
\end{center}
\caption{割线定理}
\end{figure}

\begin{figure}[!htb]
\begin{center}
\begin{tikzpicture}[scale=.55]
  \tikzmath{
    \r = 5;
  }
  \coordinate (O) at (0,0);
  \coordinate (A) at (40:\r);
  \coordinate (B) at (-20:\r);
  \coordinate (P) at ($(B)!2!(A)$);
  \draw[thick,red,name path=c1] (O) circle (\r);
  \coordinate (Q) at ($(O)!.5!(P)$);
  \path[circle={Q,P},name path=c2];
  \path[name intersections={of=c1 and c2,by={S,T}}]; % discard S
  \draw[thick] (P) -- (B); 
  \draw[thick] (P) -- (T); 
  \draw[dashed] (T) -- (A);
  \draw[dashed] (T) -- (B);
  \foreach \p in {O,A,B,P,T}
    \fill[red] (\p) circle (2pt);
  \draw (A) node[above right] {$A$};
  \draw (B) node[below right] {$B$};
  \draw (P) node[above] {$P$};
  \draw (T) node[above left] {$T$};
  \draw (O) node[below] {$O$};
\end{tikzpicture}
\end{center}
\caption{切割线定理}
\end{figure}

\begin{proof}
通过相似三角形可以证明
\end{proof}

\section{阿波罗尼斯圆}

\begin{theorem}
已知平面上两点$A,B$,则所有满足$\cfrac{PA}{PB}=\lambda(\lambda \neq 1)$
则点$P$的轨迹,是以定比$\lambda$内分和外分定线段的两个分点的连线为直径的圆.
\end{theorem}
这个轨迹最先由古希腊数学家阿波罗尼斯(Apollonius)发现, 故称作阿波罗尼斯(或阿氏圆).

\begin{figure}[!htb]
\begin{center}
\begin{tikzpicture}[scale=.55]
  \tikzmath{
    \k = 2;
    function inner(\k) {
      return \k/(\k+1);
    };
    function outer(\k) {
      return \k/(\k-1);
    };
  }
  \coordinate (A) at (0,0);
  \coordinate (B) at (6,0);
  \coordinate (M) at ($(A)!inner(\k)!(B)$);
  \coordinate (N) at ($(A)!outer(\k)!(B)$);
  \coordinate (O) at ($(M)!.5!(N)$);
  \coordinate[revolve/angle=-55,revolve={O,M}] (P);
  \draw[thick] (A) -- (N);
  \draw[red,thick,circle={O,M}];
  \draw[thick,blue] (P) -- (A) (P) -- (B);
  \draw[dashed,red] (P) -- (M) (O) -- (P);
  \foreach \p in {O,A,B,M,N,P}
    \fill[red] (\p) circle (2pt);
  \draw (A) node[below] {$A$};
  \draw (B) node[below] {$B$};
  \draw (M) node[below left] {$M$};
  \draw (N) node[below right] {$N$};
  \draw (O) node[below] {$O$};
  \draw (P) node[above] {$P$};
\end{tikzpicture}
\end{center}
\caption{阿波罗尼斯圆}
\end{figure}

\begin{proof}
如图, 不妨设$\lambda>1$, 在直线 $AB$ 上取点 $M,N$,使得
\[\cfrac{MA}{MB}=\cfrac{NA}{NB}=1\]
连接$PM,PN$,若$P$ 不在直线 $AB$ 上, 由角平分线定理可得, 
$PM,PN$分别为$\angle APB$的角分线和外角平分线, 故:
\[\angle MPN = 90^\circ\]
所以, $P$在以$MN$为直径的圆上.

设圆心为$O$, 下面证明圆$\bigodot O$上的点满足$\cfrac{PA}{PB}=\lambda$.
设$AB=d$,则
\begin{align*}
MB &= \cfrac{d}{\lambda+1} \\
NB &= \cfrac{d}{\lambda-1} \\
MN &= MB+NB=\cfrac{2\lambda d}{{\lambda}^2-1} \\
OP &= OM=\cfrac{1}{2}MN=\cfrac{\lambda d}{{\lambda}^2-1} \\
OB &= OM-MB=\cfrac{d}{{\lambda}^2-1} \\
OA &= AB+OB=\cfrac{{\lambda}^2d}{{\lambda}^2-1} \\
\intertext{对于圆$O$上出 $M,N$之外的点$P$, 有}
\cfrac{OP}{OB}=\cfrac{OA}{OP}=\lambda \\
\intertext{因此,} 
\triangle OPB \sim \triangle OPA \\
\cfrac{PA}{PB} = \lambda
\end{align*}
\end{proof}


\section{托勒密定理 Ptolemy's theorem}

\begin{theorem}[托勒密定理,Ptolemy's theorem\label{ptolemy}]
圆内接四边形两对对边乘积的和等于两条对角线的乘积.
\[AB \cdot CD + AD \cdot BC = AC \cdot BD\]
\end{theorem}

托勒密定理实际上也可以看做一种判定圆内接四边形的方法. 

\begin{figure}[!htb]
\begin{center}
\begin{tikzpicture}[scale=.55]
  \tikzmath{
    \r = 5;
    \a = 190;
    \b = -20;
    \c = 40;
    \d = 100;
    \e = \a-(\d-\c);
  }
  \coordinate (O) at (0,0);
  \coordinate (A) at (\a:\r);
  \coordinate (B) at (\b:\r);
  \coordinate (C) at (\c:\r);
  \coordinate (D) at (\d:\r);
  \coordinate (E) at (\e:\r);
  \draw[thick,red] (O) circle (\r);
  \draw[thick] (A) -- (B) -- (C) -- (D) -- cycle; 
  \draw[thick] (A) -- (C);
  \draw[thick] (B) -- (D);
  \draw[thick,blue,dashed] (A) -- (E) (C) -- (E);
  \foreach \p in {O,A,B,C,D,E}
    \fill[red] (\p) circle (2pt);
  \draw (A) node[below left] {$A$};
  \draw (B) node[below right] {$B$};
  \draw (C) node[above right] {$C$};
  \draw (D) node[above] {$D$};
  \draw (E) node[above left] {$E$};
  \draw (O) node[below] {$O$};
\end{tikzpicture}
\end{center}
\caption{托勒密定理}
\end{figure}

证明方法很多, 采用构造相似三角形的方法
\urldef\ptolemy\url{https://en.wikipedia.org/wiki/Ptolemy%27s_theorem}
\footnote{\ptolemy}, 本文采用面积法.

\begin{proof} 
  不妨设$\widehat{AD}>\widehat{CD}$,
  在$\widehat{AD}$上找到一点$E$使得$\widehat{AE}=\widehat{CD}$.
  设$AC,BD$的夹角为$\alpha$.
  \begin{align*}
    S_{ABCE} &= S_{\triangle ABE} + S_{\triangle BCE} \\
             &= AB \cdot AE \cdot \sin {\angle BAE} + BC \cdot CE \cdot \sin {\angle BCE}  \\
             &= AB \cdot CD \cdot \sin {\angle BAE} + BC \cdot AD \cdot \sin {\angle BCE} \\
    S_{ABCD} &= AC \cdot BD \cdot \sin \alpha \\
    \alpha &= \angle CBD + \angle ACB \\
           &= \angle ACE + \angle ACB \\
           &= \angle BCE \\
    \angle BCE + \angle BAE = 180^\circ \\
    \sin \alpha = \sin {\angle BAE} = \sin {\angle BCE} \\
    S_{ABCE} &= S_{ABCD} \\
    AB \cdot CD + AD \cdot BC = AC \cdot BD
  \end{align*}
\end{proof}

\section{布拉美古塔定理 Brahmagupta's theorem}

\begin{theorem}[布拉美古塔定理,Brahmagupta's theorem]
若圆内接四边形的对角线相互垂直, 则垂直于一边且过对角线交点的直线将平分对边.
\end{theorem}

\begin{figure}[!htb]
\begin{center}
\begin{tikzpicture}[scale=.75]
  \coordinate (A) at (1,5);
  \coordinate (B) at (0,0);
  \coordinate (C) at (7,0);
  \coordinate (M) at ($(A)!(B)!(C)$);
  \coordinate[circumcenter={A,B,C}] (O);
  \draw[red,circle={O,A},name path=circumcircle];
  \path[name path=l] (B) -- ($(B)!2!(M)$);
  \path[name intersections={of=l and circumcircle,sort by=l,by={B,D}}];
  \draw[thick] (A) -- (B) -- (C) -- (D) -- cycle; 
  \draw[thick] (A) -- (C);
  \draw[thick] (B) -- (D);
  \coordinate (E) at ($(B)!(M)!(C)$);
  \coordinate[intersect={A,D,M,E}] (F);
  \draw[thick,blue] (E) -- (F);
  \foreach \p in {A,B,C,D,E,F,M}
    \fill[red] (\p) circle (2pt);
  \pic[draw,red,angle radius=6pt]{right angle=C--M--D};
  \pic[draw,red,angle radius=6pt]{right angle=M--E--C};
  \draw (A) node[above left] {$A$};
  \draw (B) node[below left] {$B$};
  \draw (C) node[below right] {$C$};
  \draw (D) node[above] {$D$};
  \draw (E) node[below] {$E$};
  \draw (F) node[above left] {$F$};
  \draw (M) node[left] {$M$};
\end{tikzpicture}
\end{center}
\caption{布拉美古塔定理}
\end{figure}

\section{密克定理 Miquel's theorem}

1838年,奥古斯特·密克(Auguste Miquel,法国数学家,曾发表数条圆和多边形的定理,称为密克定理(Miquel's theorem).

\begin{theorem}[三圆定理]
  设三个圆 $O_1,O_2,O_3$ 交于一点$O$, 而$D,E,F$分别是$O_2$和$O_3$, $O_3$和$O_1$,$O_1$和$O_2$
  的另一交点.
  设$A$为$O_1$的点,直线$AF$交$O_2$于$B$,
  直线$AE$交$O_3$于$C$,那么$B,C,D$三点共线.
\end{theorem}

\begin{theorem}[三圆定理的逆定理]
  对任意 $\triangle ABC$, $D,E,F$三点分别在边$AB,BC,CA$上(或延长线上),
  那么$\triangle AFE$,$\triangle BDF$,$\triangle CED$的外接圆交于一点.
\end{theorem}

\begin{figure}[!htb]
\begin{center}
\begin{tikzpicture}
  \tikzmath{
    \a = .45;
    \b = .65;
    \c = .55;
  }
  \coordinate (A) at (2,5);
  \coordinate (B) at (0,0);
  \coordinate (C) at (7,0);
  \coordinate (D) at ($(B)!\a!(C)$);
  \coordinate (E) at ($(C)!\b!(A)$);
  \coordinate (F) at ($(A)!\c!(B)$);
  \coordinate[circumcenter={A,F,E}] (O1);
  \coordinate[circumcenter={B,D,F}] (O2);
  \coordinate[circumcenter={C,E,D}] (O3);
  \draw[red,circle={O1,A}];
  \draw[blue,circle={O2,B}];
  \draw[magenta,circle={O3,C}];
  \draw[thick] (A) -- (B) -- (C) -- cycle; 
  \draw[thick] (D) -- (E) -- (F) -- cycle;
  \foreach \p in {A,B,C,D,E,F}
    \fill[red] (\p) circle (2pt);
  \draw (A) node[above] {$A$};
  \draw (B) node[below left] {$B$};
  \draw (C) node[below right] {$C$};
  \draw (D) node[below] {$D$};
  \draw (E) node[above right] {$E$};
  \draw (F) node[above left] {$F$};
\end{tikzpicture}
\end{center}
\caption{密克定理之三圆定理}
\end{figure}

\begin{definition}[完全四边形]
  完全四边形是平面上由任意四条直线组成的图形,其中任意三条都不共点,且相交于六个点.
\end{definition}

\begin{theorem}[完全四线形定理]
  如果 $ABCDEF$ 是完全四线形,那么 $\triangle EAD$, $\triangle EBC$, $\triangle FAB$, $\triangle FDC$ 的外接圆交于一点.
\end{theorem}

\begin{figure}[!htb]
\begin{center}
\begin{tikzpicture}[scale=.65]
  \coordinate (A) at (-1,0);
  \coordinate (B) at (3.5,0);
  \coordinate (C) at (3,2);
  \coordinate (D) at (1,3);
  \coordinate[intersect={A,B,C,D}] (E);
  \coordinate[intersect={A,D,B,C}] (F);
  \coordinate[circumcenter={E,A,D}] (O1);
  \coordinate[circumcenter={E,B,C}] (O2);
  \coordinate[circumcenter={F,A,B}] (O3);
  \coordinate[circumcenter={F,D,C}] (O4);
  \draw[circle={O1,A},thick,red];
  \draw[circle={O2,C},thick,teal];
  \draw[circle={O3,A},thick,cyan];
  \draw[circle={O4,C},thick,purple];
  \draw[thick] (F) -- (A) (F) -- (B) (E) -- (A) (E) -- (D); 
  \foreach \p in {A,B,C,D,E,F}
    \fill[red] (\p) circle (2pt);
  \draw (A) node[below left] {$A$};
  \draw (B) node[below] {$B$};
  \draw (C) node[above right] {$C$};
  \draw (D) node[above left] {$D$};
  \draw (E) node[below right] {$E$};
  \draw (F) node[above left] {$F$};
\end{tikzpicture}
\end{center}
\caption{密克定理之完全四线形定理}
\end{figure}

\begin{theorem}[四圆定理]
  设$O_1, O_2,O_3, O_4$为四个圆,
  $A_1$和$B_1$是$O_1$和$O_2$的交点,
  $A_2$和$B_2$是$O_2$和$O_3$的交点,
  $A_3$和$B_3$是$O_3$和$O_4$的交点,
  $A_4$和$B_4$是$O_4$和$O_1$的交点,
  那么$A_1,A_2,A_3,A_4$四点共圆当且仅当$B_1,B_2,B_3,B_4$四点共圆.
\end{theorem}

\begin{figure}[!htb]
\begin{center}
\begin{tikzpicture}[scale=.65]
  \tikzmath{
    \r = 3;
    \a1 = 1.2;
    \a2 = 1.45;
    \a3 = 1.6;
    \a4 = 1.7;
  }
  \coordinate (O) at (0,0);
  \coordinate (A1) at (55:\r);
  \coordinate (A2) at (130:\r);
  \coordinate (A3) at (190:\r);
  \coordinate (A4) at (-20:\r);
  \coordinate (M1) at ($(A1)!.5!(A2)$);
  \coordinate (M2) at ($(A2)!.5!(A3)$);
  \coordinate (M3) at ($(A3)!.5!(A4)$);
  \coordinate (M4) at ($(A4)!.5!(A1)$);
  \coordinate (O1) at ($(O)!\a1!(M1)$);
  \coordinate (O2) at ($(O)!\a2!(M2)$);
  \coordinate (O3) at ($(O)!\a3!(M3)$);
  \coordinate (O4) at ($(O)!\a4!(M4)$);
  \coordinate[reflect={O4,O1,A1}] (B1);
  \coordinate[reflect={O1,O2,A2}] (B2);
  \coordinate[reflect={O2,O3,A3}] (B3);
  \coordinate[reflect={O3,O4,A4}] (B4);
  \draw[circle={O1,A1},name path=circle1,red];
  \draw[circle={O2,A2},name path=circle2,teal];
  \draw[circle={O3,A3},name path=circle3,cyan];
  \draw[circle={O4,A4},name path=circle4,blue];
  \coordinate[circumcenter={B1,B2,B3}] (O5);
  \draw[magenta,thick] (O) circle (\r);
  \draw[dashed,thick,purple,circle={O5,B1}];
  \foreach \p in {O1,O2,O3,O4}
    \fill[red] (\p) circle (2pt);
  \foreach \p in {A1,A2,A3,A4}
    \fill[teal] (\p) circle (2pt);
  \foreach \p in {B1,B2,B3,B4}
    \fill[magenta] (\p) circle (2pt);
  
  \draw (O1) node[below] {$O_1$};
  \draw (O2) node[below] {$O_2$};
  \draw (O3) node[below] {$O_3$};
  \draw (O4) node[below] {$O_4$};
\end{tikzpicture}
\end{center}
\caption{密克定理之四圆定理}
\end{figure}

\begin{theorem}[五圆定理]
  设$A_1A_2A_3A_4A_5$为任意五边形,
  $A_1A_2 \cap A_4A_5 = B_1$,
  $A_2A_3 \cap A_5A_1 = B_2$,
  $A_3A_4 \cap A_1A_2 = B_3$,
  $A_4A_5 \cap A_2A_3 = B_4$,
  $A_5A_1 \cap A_3A_4 = B_5$,
  那么
  $\triangle B_1A_1A_5$,
  $\triangle B_2A_2A_1$,
  $\triangle B_3A_3A_2$,
  $\triangle B_4A_4A_3$,
  $\triangle B_5A_5A_4$,
  的外接圆的五个不在五边形上的交点共圆.
\end{theorem}

\begin{figure}[!htb]
\begin{center}
\begin{tikzpicture}[scale=.65]
  \coordinate (B1) at (.5,5);
  \coordinate (B2) at (-4,2);
  \coordinate (B3) at (-3,-2.5);
  \coordinate (B4) at (3,-3);
  \coordinate (B5) at (3.5,2.5);
  \coordinate[intersect={B2,B5,B1,B3}] (A1);
  \coordinate[intersect={B1,B3,B2,B4}] (A2);
  \coordinate[intersect={B2,B4,B3,B5}] (A3);
  \coordinate[intersect={B3,B5,B1,B4}] (A4);
  \coordinate[intersect={B1,B4,B2,B5}] (A5);
  \coordinate[circumcenter={B1,A1,A5}] (O1);
  \coordinate[circumcenter={B2,A2,A1}] (O2);
  \coordinate[circumcenter={B3,A3,A2}] (O3);
  \coordinate[circumcenter={B4,A4,A3}] (O4);
  \coordinate[circumcenter={B5,A5,A4}] (O5);
  \draw[circle={O1,A1},name path=circle1,red];
  \draw[circle={O2,A2},name path=circle2,teal];
  \draw[circle={O3,A3},name path=circle3,cyan];
  \draw[circle={O4,A4},name path=circle4,blue];
  \draw[circle={O5,A5},name path=circle5,magenta];
  \path[name intersections={of=circle1 and circle2,by={C1,D1}}];
  \path[name intersections={of=circle2 and circle3,by={C2,D2}}];
  \path[name intersections={of=circle3 and circle4,by={C3,D3}}];
  % \foreach \p in {C1,C2,D3}
  %   \fill[blue] (\p) circle (2pt);
  \coordinate[circumcenter={D1,C2,D3}] (O6);
  \draw[dashed,thick,purple,circle={O6,D1}];
  \draw[thick] (B1) -- (B3) -- (B5) -- (B2) -- (B4) -- cycle; 
  \foreach \p in {A1,A2,A3,A4,A5,B1,B2,B3,B4,B5,O1,O2,O3,O4,O5}
    \fill[red] (\p) circle (2pt);
\end{tikzpicture}
\end{center}
\caption{密克定理之五圆定理}
\end{figure}

\section{牛顿定理 Newton's theorem}

\begin{theorem}[牛顿定理,Newton's theorem\label{newton}]
圆外切四边形的两条对角线的中点,及该圆的圆心,三点共线.
\end{theorem}

如果圆外切四边形是菱形, 在这种情况下对角线的中点和内切圆的圆心重合.

\begin{figure}[!htb]
\begin{center}
\begin{tikzpicture}[scale=.75]
  \tikzmath{
    \r = 3;
  }
  % 定义切点
  \coordinate (P1) at (150:\r);
  \coordinate (P2) at (-90:\r);
  \coordinate (P3) at (30:\r);
  \coordinate (P4) at (100:\r);
  \coordinate (O) at (0,0);
  \coordinate[revolve/angle=90,revolve={P1,O}] (Q1); %at ($(P1)!1!90:(O)$);
  \coordinate[revolve/angle=90,revolve={P2,O}] (Q2); %at ($(P2)!1!90:(O)$);
  \coordinate[revolve/angle=90,revolve={P3,O}] (Q3); %at ($(P3)!1!90:(O)$);
  \coordinate[revolve/angle=90,revolve={P4,O}] (Q4); %at ($(P4)!1!90:(O)$);
  \coordinate[intersect={P1,Q1,P2,Q2}] (A);
  \coordinate[intersect={P2,Q2,P3,Q3}] (B);
  \coordinate[intersect={P3,Q3,P4,Q4}] (C);
  \coordinate[intersect={P4,Q4,P1,Q1}] (D);
  \coordinate (E) at ($(A)!.5!(C)$);
  \coordinate (F) at ($(B)!.5!(D)$);
  \draw[red,thick] (0,0) circle (\r);
  \draw[thick] (A) -- (B) -- (C) -- (D) -- cycle;
  \draw[purple,thick] (E) -- (F);
  \draw[teal,thick] (A) -- (C) (B) -- (D);
  \foreach \p in {P1,P2,P3,P4}
    \fill[green] (\p) circle (2pt);
  % \foreach \p in {Q1,Q2,Q3,Q4}
  %   \fill[blue] (\p) circle (2pt);
  \foreach \p in {O,A,B,C,D,E,F}
    \fill[red] (\p) circle (2pt);
  \draw (O) node[below] {$O$};
  \draw (A) node[below] {$A$};
  \draw (B) node[below] {$B$};
  \draw (C) node[above right] {$C$};
  \draw (D) node[above left] {$D$};
\end{tikzpicture}
\end{center}
\caption{牛顿定理}
\end{figure}

\section{根轴与根心 Radical axis and radical center}

\begin{definition}[根轴,Radical axis]
根轴亦称等幂轴, 指对于不同心两圆有相等幂的点的轨迹.
即向不同心两圆引相等切线的点的轨迹, 是垂直于两圆连心线的一条直线.
\end{definition}

\begin{theorem}根轴有以下性质:
\begin{enumerate}
  \item 平面上任意两圆的根轴垂直于它们的连心线;
  \item 若两圆相交, 则两圆的根轴为公共弦所在的直线;
  \item 若两圆相切, 则两圆的根轴为它们的内公切线;
  \item 若两圆外离, 则两圆的根轴上的点分别引两圆的切线, 则切线长相等. 从而, 根轴必过四条公切线的中点.
\end{enumerate}
\end{theorem}

在解析几何下, 两圆方程联立, 相减就可求得根轴的方程. 
同心圆的根轴是无穷远直线(射影几何).

\begin{theorem}[根心定理,Radical center theorem,Radical axis theorem]
  平面上任意三个圆, 若这三个圆圆心不共线, 则三条根轴相交于一点;
  若三圆圆心共线, 则三条根轴互相平行.
\end{theorem}
这个交点称为根心(radical center,或 power center)\footnote{\url{http://mathgardenblog.blogspot.com/2013/07/radical-axis.html}},
这个定理首先由蒙日(Monge)发现\footnote{\url{https://mathworld.wolfram.com/RadicalCenter.html}}.

\begin{remark*}
  可以使用代数方法来证明: 根据三个圆的方程得到三条根轴的方程, 根轴方程的系数组成一个 $3 \times 3$ 矩阵, 该矩阵行向量线性相关(第 1 行+第 2 行=-第 3 行), 
因此其行列式为 0, 三线共点.
\end{remark*}

% url 中含有 %, 不能在 footnote 中直接使用
\urldef\radicalaxis\url{https://ja.wikipedia.org/wiki/%E6%A0%B9%E8%BB%B8}
根轴的几何构造方法:
两圆相交时, 根轴为两圆交点的连线;两圆相切时, 根轴为两圆的公切线;
两圆内含或外离时,根据这个定理我们可以借助第三个圆作出两个不同心圆的根轴\footnote{\radicalaxis}.
\begin{enumerate}
  \item 作 $O_3$ 交 $O_1$ 于 $A_1A_2$ 交 $O_2$ 于 $B_1B_2$
  \item 作 $O_4$ 交 $O_1$ 于 $C_1C_2$ 交 $O_2$ 于 $D_1D_2$
  \item $A_1A_2$ 交 $B_1B_2$ 于 $P$
  \item $C_1C_2$ 交 $D_1D_2$ 于 $Q$
\end{enumerate}
$PQ$ 即是.

\begin{figure}[!htb]
\begin{center}
\begin{tikzpicture}[scale=.55]
  \tikzmath{
    \r1 = 3;
    \r2 = 4;
    \r3 = 3;
    \r4 = 7;
  }
  \coordinate (O1) at (-1,0);
  \coordinate (O2) at (7,0);
  \coordinate (O3) at (1,3);
  \coordinate (O4) at (1,4);
  \draw[thick,name path=circle1] (O1) circle (\r1);
  \draw[thick,name path=circle2] (O2) circle (\r2);
  \draw[name path=circle3,magenta] (O3) circle (\r3);
  \draw[name path=circle4,blue] (O4) circle (\r4);
  \path[name intersections={of= circle3 and circle1, by={A1,A2}}];
  \path[name intersections={of= circle3 and circle2, by={B1,B2}}];
  \path[name intersections={of= circle4 and circle1, by={C1,C2}}];
  \path[name intersections={of= circle4 and circle2, by={D1,D2}}];
  \coordinate[intersect={A1,A2,B1,B2}] (P);
  \coordinate[intersect={C1,C2,D1,D2}] (Q);
  \draw[red,thick] (P) -- (Q);
  \draw[magenta] (A1) -- (A2) (B1) -- (B2);
  \draw[blue] (C1) -- (C2) (D1) -- (D2);
  \foreach \p in {A1,A2,B1,B2,C1,C2,D1,D2,O1,O2,O3,O4}
    \fill[red] (\p) circle (2pt);
  \draw (O1) node[below] {$O_1$};
  \draw (O2) node[below] {$O_2$};
  \draw (O3) node[below] {$O_3$};
  \draw (O4) node[below] {$O_4$};
\end{tikzpicture}
\end{center}
\caption{根轴的几何构造方法}
\end{figure}

\section{蒙日定理 Monge's theorem}

\begin{theorem}[蒙日定理,Monge's theorem\label{monge}]
  对于平面上的任意三个圆,其中没有一个圆包含于另外的圆,则每两个圆的外公切线的交点在一直线上.
\end{theorem}

\begin{figure}[!htb]
\begin{center}
\begin{tikzpicture}
  \tikzmath{
    \a = 30;
    \r1 = .5;
    \r2 = 1.5;
    \r3 = 2.5;
  }
  \coordinate (O1) at (-1,-2);
  \coordinate (A1) at ($(O1)+(\a:\r1)$);
  \coordinate (O2) at (2,1);
  \coordinate (A2) at ($(O2)+(\a:\r2)$);
  \coordinate (O3) at (-3,4);
  \coordinate (A3) at ($(O3)+(\a:\r3)$);
  \coordinate[external center={O2,A2,O3,A3}] (X);
  \coordinate[external center={O3,A3,O1,A1}] (Y);
  \coordinate[external center={O1,A1,O2,A2}] (Z);
  \draw[thick] (O1) circle (\r1);
  \draw[thick] (O2) circle (\r2);
  \draw[thick] (O3) circle (\r3);
  \draw[thick,teal] (X) -- (Z);
  \foreach \p in {O1,O2,O3,X,Y,Z}
    \fill[red] (\p) circle (2pt);
  \draw (O1) node[below] {$O_1$};
  \draw (O2) node[below] {$O_2$};
  \draw (O3) node[below] {$O_3$};
  \draw (X) node[below] {$X$};
  \draw (Y) node[below] {$Y$};
  \draw (Z) node[below] {$Z$};
  % 作切线
  \coordinate[tangent point={O3,A3,X}] (R1);
  \coordinate[reflect={O3,X,R1}] (R2);
  \coordinate[tangent point={O3,A3,Y}] (S1);
  \coordinate[reflect={O3,Y,S1}] (S2);
  \coordinate[tangent point={O2,A2,Z}] (T1);
  \coordinate[reflect={O2,Z,T1}] (T2);
  \draw[thick,blue] (X) -- (R1) (X) -- (R2);
  \draw[thick,red] (Y) -- (S1) (Y) -- (S2);
  \draw[thick,cyan] (Z) -- (T1) (Z) -- (T2);
  \draw[dashed,teal] (O3) -- (X) (O2) -- (Z) (O3) -- (Y);
\end{tikzpicture}
\end{center}
\caption{蒙日定理}
\end{figure}

\begin{proof}
考虑$\triangle O_1O_2O_3$ 与 $X,Y,Z$, 利用\hyperref[menelaus]{梅涅劳斯定理}可以证明.
\begin{align*}
  \cfrac{O_1Z}{ZO_2} \cdot \cfrac{O_2X}{XO_3} \cdot \cfrac{O_3Y}{YO_1} = \cfrac{r_1}{r_2} \cdot \cfrac{r_2}{r_3} \cdot \cfrac{r_3}{r_1} = 1
\end{align*}
\end{proof}

这个定理还可以从三维空间来解释.
想象在这个平面的三个圆上放置着三个球,每个球的半径都等于它底下的那个圆的半径.
显然,这个平面是这三个球的一个公切面.
再把三组公切线想象成这三个球两两确定的圆锥在平面上的投影.
显然,三个圆锥的顶点都在这个平面上,我们要证明的就是,这三个顶点是共线的.

注意到这三个球还有另一个公切面,三个圆锥的顶点也都在这个公切面上.
而这两个公切面的公共部分就是它们的交线,因此这三个顶点必然都在这条交线上.

考虑到内公切线的交点, 我们可以得到一个对偶定理:
\begin{theorem}
内公切线交点与对应圆心的连线交于一点.
\end{theorem}

利用 \hyperref[ceva]{塞瓦定理} 或构造点的坐标通过行列式来证明.

\begin{figure}[!htb]
\begin{center}
\begin{tikzpicture}
  \tikzmath{
    \a = 30;
    \r1 = .5;
    \r2 = 1.5;
    \r3 = 2.5;
  }
  \coordinate (O1) at (-1,-2);
  \coordinate (A1) at ($(O1)+(\a:\r1)$);
  \coordinate (O2) at (2,1);
  \coordinate (A2) at ($(O2)+(\a:\r2)$);
  \coordinate (O3) at (-3,4);
  \coordinate (A3) at ($(O3)+(\a:\r3)$);
  \coordinate[internal center={O2,A2,O3,A3}] (U);
  \coordinate[internal center={O3,A3,O1,A1}] (V);
  \coordinate[internal center={O1,A1,O2,A2}] (W);
  \coordinate[external center={O2,A2,O3,A3}] (X);
  \coordinate[external center={O3,A3,O1,A1}] (Y);
  \coordinate[external center={O1,A1,O2,A2}] (Z);
  \draw[thick] (O1) circle (\r1);
  \draw[thick] (O2) circle (\r2);
  \draw[thick] (O3) circle (\r3);
  \foreach \p in {O1,O2,O3,U,V,W,X,Y,Z}
    \fill[red] (\p) circle (2pt);
  \draw (O1) node[below] {$O_1$};
  \draw (O2) node[below] {$O_2$};
  \draw (O3) node[below] {$O_3$};
  \draw[thick,red] (U) -- (Z) (U) -- (Y) (V) -- (X) (X) -- (Z);
  \draw[dashed,teal] (O1) -- (U) (O2) -- (V) (O3) -- (W);
\end{tikzpicture}
\end{center}
\caption{三个圆的位似中心}
\end{figure}

实际上, 三个圆的外位似中心(external homothetic centers)和内位似中心(internal homothetic centers)构成简单四线形\footnote{\url{https://en.wikipedia.org/wiki/Homothetic_center}}.

\section{蝴蝶定理 Butterfly theorem}

\begin{theorem}[蝴蝶定理,Butterfly theorem]
  设$M$为圆内弦$PQ$的中点,过$M$作弦$AB$和$CD$.
  设$AD$和$BC$各相交$PQ$于点$X$和$Y$,则$M$是$XY$的中点.
\end{theorem}

证明方法有多种,可分为初等几何,解析几何和射影几何\footnote{\url{http://www.cut-the-knot.org/pythagoras/Butterfly.shtml}}.

该定理实际上是射影几何中一个定理的特殊情况,有多种推广:
\begin{enumerate}
  \item M作为圆内弦中点是不必要的,可以移到圆外.
  \item 圆可以改为任意圆锥曲线.
  \item 将圆变为一个完全四角形,M为对角线交点.
  \item 去掉中点的条件,结论变为一个一般关于有向线段的比例式,称为坎迪定理(Candy's theorem),
  $M$不为中点时满足:
  \[\cfrac{1}{MY}-\cfrac{1}{MX}=\cfrac{1}{MQ}-\cfrac{1}{MP}\]
  这对2,3均成立.
\end{enumerate}

\begin{figure}[!htb]
\begin{center}
\begin{tikzpicture}
  \tikzmath{
    \r = 3;
  }
  \coordinate (O) at (0,0);
  \coordinate (P) at (150:\r);
  \coordinate (Q) at (40:\r);
  \coordinate (M) at ($(P)!.5!(Q)$);
  \coordinate (A) at (110:\r);
  \coordinate (C) at (60:\r);
  \coordinate (E) at ($(M)!(O)!(A)$);
  \coordinate (F) at ($(M)!(O)!(C)$);
  \coordinate (B) at ($(A)!2!(E)$);
  \coordinate (D) at ($(C)!2!(F)$);
  \coordinate[intersect={P,Q,A,D}] (X);
  \coordinate[intersect={P,Q,B,C}] (Y);
  \draw[thick,red] (O) circle (\r);
  \draw[thick] (P) -- (Q) (A) -- (B) -- (C) -- (D) -- cycle;
  \foreach \p in {A,B,C,D,M,P,Q,X,Y}
    \fill[red] (\p) circle (2pt);
  \draw (A) node[above left] {$A$};
  \draw (B) node[below right] {$B$};
  \draw (C) node[above right] {$C$};
  \draw (D) node[below left] {$D$};
  \draw (M) node[above] {$M$};
  \draw (P) node[left] {$P$};
  \draw (Q) node[right] {$Q$};
  \draw (X) node[above left] {$X$};
  \draw (Y) node[above right] {$Y$};
\end{tikzpicture}
\end{center}
\caption{蝴蝶定理}
\end{figure}



