% the converse to Brianchon's theorem
\begin{tikzpicture}
  \clip (-6,-6) rectangle (6,6);
  \axes(-5:5,-5:5);

  % 为方便起见, 使用了椭圆参数方程来生成 5 个点
  % 用这 5 个点产生 5 条切线
  \tikzmath{
    \a = 4;
    \b = 3;
  }
  \foreach \i/\angle in {1/45,2/110,3/170,4/240,5/330}{
    \coordinate (A\i) at ({\a*cos(\angle)},{\b*sin(\angle)});
  }

  % 利用 Brianchon 定理中退化的外接六边形(只有 5 边, 其中一边为切线), 
  % 求各直线上的切点
  \coordinate[intersect={A1,A3,A2,A5}] (B1);
  \coordinate[intersect={A1,A2,A4,B1}] (T1);

  \coordinate[intersect={A1,A3,A2,A4}] (B2);
  \coordinate[intersect={A2,A3,A5,B2}] (T2);

  \coordinate[intersect={A2,A4,A3,A5}] (B3);
  \coordinate[intersect={A3,A4,A1,B3}] (T3);

  \coordinate[intersect={A3,A5,A1,A4}] (B4);
  \coordinate[intersect={A4,A5,A2,B4}] (T4);

  \coordinate[intersect={A1,A4,A2,A5}] (B5);
  \coordinate[intersect={A5,A1,A3,B5}] (T5);

  \draw[red,thick] (A1) -- (A2) -- (A3) -- (A4) -- (A5) -- cycle;
  \draw[thick] (A1) -- (A3)  (A5) -- (A2) (A4) -- (T1);

  \foreach \i/\placement in {1/above right,2/above,3/left,4/below,5/below right}{
    \fill[red] (A\i) circle (2pt);
    \draw (A\i) node[\placement] {$A\i$};
  }

  \foreach \i/\placement in {1/above right,2/above left,3/left,4/below,5/below right}{
    \fill[teal] (T\i) circle (2pt);
    \draw (T\i) node[\placement] {$T\i$};
  }

\end{tikzpicture}