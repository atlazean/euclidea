\documentclass{standalone}
\usepackage{tikz}
\usetikzlibrary{calc, euclidea, quotes,angles}

\begin{document}
\begin{tikzpicture}
  \tikzmath{
    \r = 3;
    \anglea = 200;
    \angleb = -30;
    \anglec = 70;
    \angled = 130;
  }
  \coordinate (O) at (0,0);
  \coordinate (A) at (\anglea:\r);
  \coordinate (B) at (\angleb:\r);
  \coordinate (C) at (\anglec:\r);
  \coordinate (D) at (\angled:\r);
  \coordinate[revolve/angle={C,A,D},
    revolve={C,B}] (P);
  \coordinate[intersect={B,D,C,P}] (E);

  \fill[cyan,opacity=.5] (A) -- (C) -- (D) -- cycle;
  \fill[cyan,opacity=.5] (B) -- (C) -- (E) -- cycle;
  \draw[purple] (O) circle (\r);
  \draw[thick] (A) -- (B) -- (C) -- (D) -- cycle;
  \draw[thick] (A) -- (C)  (B) -- (D) (C) -- (E);
  
  \pic["$\alpha$", draw,angle eccentricity=1.5]{angle=D--C--A};
  \pic["$\alpha$", draw,angle eccentricity=1.5]{angle=E--C--B};

  \foreach \p/\placement in {A/below left, B/below right,
  C/above right, D/above left, E/below left}{
    \fill[red] (\p) circle (2pt);
    \draw (\p) node[\placement] {$\p$};
  }
\end{tikzpicture}
\end{document}