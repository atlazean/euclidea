\chapter{基本作图命令}

这里的命令都是通过 /tikz/insert path\cite{PATH} 在当前路径上插入新的路径, 或者是通过 \verbum{def} 组合一些 tikz 命令.

% ---------------------------------
\section{坐标轴 Axes}

\emph{调用方式}

\begin{tcolorbox}{}
\mint{latex}{\axes (xmin:xmain,ymin:ymax);}
\end{tcolorbox}

\emph{参数说明}

\begin{description}
  \item[xmin,xmax] $x$范围
  \item[ymin,ymax] $y$范围
\end{description}

\emph{示例}

见\ref{sec:transform}.

% ---------------------------------
\section{坐标变换 Coordinates Transformations}\label{sec:transform}

\emph{调用方式}

\begin{tcolorbox}{}
\mint{latex}{transform={angle:(xshift,yshift)}}
\end{tcolorbox}

\emph{参数说明}

绕原点旋转 angle, 然后水平方向和竖直方向分别平移 xshift和 yshift

\emph{示例}

\showcode{snippets/commands/transform.tex}

% ---------------------------------
\section{仿射组合 Affine Combination}

\emph{调用方式}

\begin{tcolorbox}{}
\mint{latex}{affine={A,B,k}}
\end{tcolorbox}

\emph{参数说明}

\begin{description}
  \item[A, B] 两点坐标
  \item[k] 系数
\end{description}

返回点 $A, B$ 的仿射组合: $A + k \cdot (B-A)$.

\emph{示例}

\showcode{snippets/commands/affine.tex}

% ---------------------------------
\section{中点 Midpoint}

\emph{调用方式}

\begin{tcolorbox}{}
\mint{latex}{midpoint={A,B}}
\end{tcolorbox}

\emph{参数说明}

\begin{description}
  \item[A, B] 两点坐标
\end{description}

返回点 $A, B$ 的中点坐标.

\emph{示例}

\showcode{snippets/commands/midpoint.tex}

% ---------------------------------
\section{平移 Translate}

\emph{调用方式}

\begin{tcolorbox}{}
\mint{latex}{translate={A,B,C}}
\end{tcolorbox}

\emph{参数说明}

\begin{description}
  \item[A,B,C] 三点坐标
\end{description}

返回 $C$ 按向量 $AB$ 移动所得的坐标: $C + (B-A)$.

\emph{示例}

\showcode{snippets/commands/translate.tex}

% ---------------------------------
\section{对称点 Reflect}

\emph{调用方式}

\begin{tcolorbox}{}
\mint{latex}{reflect={A,B,C}}
\end{tcolorbox}

\emph{参数说明}

\begin{description}
  \item[A,B,C] 三点坐标
\end{description}

返回 $C$ 关于直线 $AB$ 的对称点的坐标(设$D$ 为 $C$ 在 $AB$ 的投影): $C+2(D-C)$.

\emph{示例}

\showcode{snippets/commands/reflect.tex}

% ---------------------------------
\section{投影 Project}

\emph{调用方式}

\begin{tcolorbox}{}
\mint{latex}{project={A,B,C}}
\end{tcolorbox}

\emph{参数说明}

\begin{description}
  \item[A,B,C] 三点坐标
\end{description}

返回 $C$ 在直线 $AB$ 的投影的坐标.

\emph{示例}

\showcode{snippets/commands/project.tex}

% ---------------------------------
\section{反演 Inverse}

\emph{调用方式}

\begin{tcolorbox}{}
\mint{latex}{inverse={O,A,P}}
\end{tcolorbox}

\emph{参数说明}

\begin{description}
  \item[O] 圆心
  \item[A] 圆上一点
  \item[P] 平面上任一点 
\end{description}

返回 $P$ 关于圆 $(O,A)$ 的反演点.

\emph{示例}

\showcode{snippets/commands/inverse.tex}

% ---------------------------------
\section{旋转 Revolve}

\emph{调用方式}

\begin{tcolorbox}{}
\mint{latex}{revolve={A,B}}
\end{tcolorbox}

\emph{参数说明}

\begin{description}
  \item[A,B] 两点坐标
\end{description}

\begin{remark*}
  为了避免覆盖 tikz 的 rotate, 这里将旋转命令为 revolve.
\end{remark*}

返回 $B$ 绕 $A$ 旋转的点.

还需要指定 revolve/angle (default: 0) 和 revolve/angle scale(default: 1) 两个选项,可以通过下面的方式来指定 /revolve/angle:

\begin{enumerate}
  \item 直接指定角度: \mintinline{latex}|revolve/angle=60|
  \item 位置向量与 x 轴夹角: \mintinline{latex}|revolve/angle={P1}|
  \item 两位置向量的夹角: \mintinline{latex}|revolve/angle={P1,P2}|
  \item 由三点定义的角($P_1$ 为顶点, $P_2$ 为起点, $P_3$ 为终点): \mintinline[breaklines, breakafter=,]{latex}|revolve/angle={P1,P2,P3}|
  \item 两向量的夹角(逆时针方向): \mintinline{latex}|revolve/angle={P1,P2,P3,P4}|
\end{enumerate}

\emph{示例}

\showcode{snippets/commands/revolve1.tex}

\showcode{snippets/commands/revolve2.tex}

\showcode{snippets/commands/revolve3.tex}

% ---------------------------------
\section{构造角 Angle}

可以由 revolve 来构造一个角.

\emph{示例}

\showcode{snippets/commands/angle.tex}

% ---------------------------------
\section{角平分线 Angle Bisector}

\emph{调用方式}

\begin{tcolorbox}{}
\mint{latex}{angle bisector={A,B,C}}
\end{tcolorbox}

\emph{参数说明}

\begin{description}
  \item[A,B,C] 三点坐标, $A$为顶点(apex), $B$为起点, $C$为终点
\end{description}

返回 $\angle{BAC}$ 角平分线上的一点.实际上, 该操作等价于:

\begin{minted}{latex}
revolve/angle={A,B,C}, revolve/scale=.5, revolve={A,B}
\end{minted}

\emph{示例}

\showcode{snippets/commands/angle-bisector.tex}

% ---------------------------------
\section{等边三角形 Equilateral Triangle}

\emph{调用方式}

\begin{tcolorbox}{}
\mint{latex}{equilateral={A,B}}
\end{tcolorbox}

\emph{参数说明}

\begin{description}
  \item[A,B] 两点坐标
\end{description}

返回 以 $AB$ 为边长的等边三角形的第 3 点 (位于向量 $AB$ 的左侧).实际上, 该操作等价于:

\begin{minted}{latex}
revolve/angle=60, revolve={A,B}
\end{minted}

\emph{示例}

\showcode{snippets/commands/equilateral.tex}

% ---------------------------------
\section{旋转 $90^\circ$ Erect}

\emph{调用方式}

\begin{tcolorbox}{}
\mint{latex}{erect={A,B}}
\end{tcolorbox}

\emph{参数说明}

\begin{description}
  \item[A,B] 两点坐标
\end{description}

返回$B$绕 $A$ 旋转 $90^\circ$ 的坐标.实际上, 该操作等价于:

\begin{minted}{latex}
revolve/angle=90, revolve={A,B}
\end{minted}

\emph{示例}

\showcode{snippets/commands/erect.tex}

% ---------------------------------
\section{截取 Intercept}

\emph{调用方式}

\begin{tcolorbox}{}
\mint{latex}{intercept={A,B}}
\end{tcolorbox}

\emph{参数说明}

\begin{description}
  \item[A,B] 两点坐标
\end{description}

在直线 $AB$ 截取指定长度线段, $A$ 为新线段的起点, $AB$ 是方向.

需要指定 intercept/length (default: 1cm) 和 intercept/scale(default: 1) 两个选项.
其中 intercept/length 有两种形式:

\begin{enumerate}
  \item 直接指定长度: \mintinline{latex}|intercept/length=2cm|
  \item 指定线段长度: \mintinline{latex}|intercept/length={P1,P2}|
\end{enumerate}

\emph{示例}

\showcode{snippets/commands/intercept1.tex}

\showcode{snippets/commands/intercept2.tex}

% ---------------------------------
\section{直线与直线的交点 Line-Line Intersection}

\emph{调用方式}

\begin{tcolorbox}{}
\mint{latex}{intersect={A,B,C,D}}
\end{tcolorbox}

\emph{参数说明}

\begin{description}
  \item[A,B,C,D] 四点坐标 
\end{description}

返回 $AB$ 与 $CD$ 的交点 (可以是延长线相交点).

\emph{示例}

\showcode{snippets/commands/intersect.tex}

% ---------------------------------
\section{垂直平分线/中垂线 Perpendicular Bisector}

\emph{调用方式}

\begin{tcolorbox}{}
\mint{latex}{perpendicular bisector={A,B}}
\end{tcolorbox}

\emph{参数说明}

\begin{description}
  \item[A,B] 两点坐标 
\end{description}

构造 $AB$ 的中垂线, 默认起点为 $.5(A+B)+(B-A) \cdot \mathbf{i}$, 终点为 $.5(A+B)-(B-A) \cdot \mathbf{i}$.
可以对起始点进行调整, 见\ref{sec:modifiers}.

\emph{示例}

使用默认参数:

\showcode{snippets/commands/perpendicular-bisector1.tex}

指定两端的长度:

\showcode{snippets/commands/perpendicular-bisector2.tex}

指定系数:

\showcode{snippets/commands/perpendicular-bisector3.tex}

可以是负数, 这样就在相反方向:

\showcode{snippets/commands/perpendicular-bisector4.tex}

% ---------------------------------
\section{垂线 Perpendicular Line}

\emph{调用方式}

\begin{tcolorbox}{}
\mint{latex}{perpendicular={A,B,C}}
\end{tcolorbox}

\emph{参数说明}

\begin{description}
  \item[A,B,C] 三点坐标 
\end{description}

构造过 $C$ 垂直于 $AB$ 的直线(设垂足为 $D$), 默认起点为 $D+(B-A) \cdot \mathbf{i}$, 终点为 $D-(B-A) \cdot \mathbf{i}$.
可以对起始点进行调整, 见\ref{sec:modifiers}.

\emph{示例}

过直线外一点的垂线:

\showcode{snippets/commands/perpendicular-line1.tex}

过直线上一点的垂线:

\showcode{snippets/commands/perpendicular-line2.tex}

% ---------------------------------
\section{平行线 Parallel Line}

\emph{调用方式}

\begin{tcolorbox}{}
\mint{latex}{parallel={A,B,C}}
\end{tcolorbox}

\emph{参数说明}

过一点 $C$ 作直线 $AB$ 平行线, (如果$C$ 在 $AB$ 上, 则重合).

首先将点 $C$ 按向量 $AB$ 平移至 $D$. 
可以对起始点进行调整, 见\ref{sec:modifiers}.

\emph{示例}

指定起始点距离 $C$ 的位置, 方向是 $CD$, 负值代表相反方向:

\showcode{snippets/commands/parallel-line1.tex}

指定系数:

\showcode{snippets/commands/parallel-line2.tex}

% ---------------------------------
\section{延长线 Extend}

\emph{调用方式}

\begin{tcolorbox}{}
\mint{latex}{extend={A,B}}
\end{tcolorbox}

\emph{参数说明}

作线段 $AB$ 延长线,
可以对起始点进行调整, 见\ref{sec:modifiers}.
实际上, 该操作等价于:

\begin{minted}{latex}
parallel={A,B,A}
\end{minted}

\emph{示例}

\showcode{snippets/commands/extend.tex}

% ---------------------------------
\section{圆 Circle}

\emph{调用方式}

\begin{tcolorbox}{}
\mint{latex}{circle={O,A}}
\end{tcolorbox}

\emph{参数说明}

\begin{description}
  \item[O] 圆心
  \item[A] 圆上一点
\end{description}

构造圆心为 $O$, 经过 $A$ 的圆.

\emph{示例}

\showcode{snippets/commands/circle.tex}

% ---------------------------------
\section{直线与圆的切点 Tangent Point}

\emph{调用方式}

\begin{tcolorbox}{}
\mint{latex}{circle-tangent = {O,A,P}}
\end{tcolorbox}

\emph{参数说明}

\begin{description}
  \item[O]: 圆心坐标
  \item[A]: 为圆上任意一点
  \item[P]: 圆外一点坐标
\end{description}

过圆 ($O$ 为圆心, $A$ 为圆上任意一点) 外一点 $P$ 作切线, 
求得两个切点(存储在tp1 和 tp2 中, 其中 tp1 在向量 $OP$ 的左边).

\emph{示例}

\showcode{snippets/commands/tangent-2.tex}

\emph{调用方式}

\begin{tcolorbox}{}
\mint{latex}{tangent point={O,A,P}}
\end{tcolorbox}

\begin{remark*}
  该调用方式已过时, 每次只能求出一个切点, 另一个切点需要用对称作出.
\end{remark*}

\emph{参数说明}

\begin{description}
  \item[O]: 圆心坐标
  \item[A]: 为圆上任意一点
  \item[P]: 圆外一点坐标
\end{description}

过圆 ($O$ 为圆心, $A$ 为圆上任意一点) 外一点 $P$ 作切线, 求得一个切点(在向量 $OP$ 的左边),
另外一点可以通过对称(\mintinline{latex}|reflect={O,P,T}|)求得.

\emph{示例}

\showcode{snippets/commands/tangent-1.tex}

% ---------------------------------
\section{两圆的交点 Circle-Circle Intersection}

\emph{调用方式}

\begin{tcolorbox}{}
\mint{latex}{circle-circle={O1,A1,O2,A2}}
\end{tcolorbox}

\emph{参数说明}

求圆1 ($O_1$ 为圆心, $A_1$ 为圆上任意一点)
和圆2 ($O_2$ 为圆心, $A_2$ 为圆上任意一点) 
交点坐标(存储在 cc1 和 cc2 中, 以 $Q_1Q_2$ 为正方向, cc1 在左侧, cc2 在右侧).

\emph{示例}

\showcode{snippets/commands/circle-circle-intersection.tex}

% ---------------------------------
\section{圆与直线的交点 Circle-Line Intersection}

\emph{调用方式}

\begin{tcolorbox}{}
\mint{latex}{circle-line={O,A,P,Q}}
\end{tcolorbox}

\emph{参数说明}

求圆 ($O$ 为圆心, $A$ 为圆上任意一点)
和直线 ($P,Q$ 为直线上任意不重合的两点) 
交点坐标(存储在 cl1 和 cl2 中, 以 $PQ$ 为正方向, cl1 在 cl2 的正方向上).

\emph{示例}

\showcode{snippets/commands/circle-line-intersection.tex}

% ---------------------------------
\section{外位似中心 External Homothetic Center}

\emph{调用方式}

\begin{tcolorbox}{}
\mint{latex}{external center={O1,A1,O2,A2}}
\end{tcolorbox}

\emph{参数说明}

求圆1 ($O_1$ 为圆心, $A_1$ 为圆上任意一点)
和圆2 ($O_2$ 为圆心, $A_2$ 为圆上任意一点) 
的外位似中心 (external homothetic center)\cite{HOMO}.

\emph{示例}

作外公切线: 先求位似中心, 可以求得两圆的外公切线.

\showcode{snippets/commands/external-common-tangents.tex}

% ---------------------------------
\section{内位似中心 Internal Homothetic Center}

\emph{调用方式}

\begin{tcolorbox}{}
\mint{latex}{internal center={O1,A1,O2,A2}}
\end{tcolorbox}

\emph{参数说明}

求圆1 ($O_1$ 为圆心, $A_1$ 为圆上任意一点)
和圆2 ($O_2$ 为圆心, $A_2$ 为圆上任意一点) 
的内位似中心 (internal homothetic center)\cite{HOMO}.

\emph{示例}

作内公切线: 先求位似中心, 可以求得两圆的内公切线.

\showcode{snippets/commands/internal-common-tangents.tex}

% ---------------------------------
\section{根轴 Radical Axis}

\emph{调用方式}

\begin{tcolorbox}{}
\mint{latex}{radical axis={O1,A1,O2,A2}}
\end{tcolorbox}

\emph{参数说明}

构造两圆的根轴, 设与$O_1O_2$ 的交点为 $P$, 则默认起点为 $P+(O_2-O_1) \cdot \mathbf{i}$, 终点为 $P-(O_2-O_1) \cdot \mathbf{i}$.
可以对起始点进行调整, 见\ref{sec:modifiers}.

\emph{示例}

两圆相交:

\showcode{snippets/commands/radical-axis1.tex}

两圆外切:

\showcode{snippets/commands/radical-axis2.tex}

两圆外离:

\showcode{snippets/commands/radical-axis3.tex}

% ---------------------------------
\section{Partway Modifiers and Distance Modifiers}%\label{sec:modifiers}

\verbum{perpendicular bisector,perpendicular,parallel,radical axis}等线段图形可以对起始点进行调整, 调整参数如下\cite{CALC}:

\begin{description}
  \item[start modifier] (default: 0), 长度或系数, 如: 1cm 或 .75
  \item[end modifier] (default: 1), 长度或系数, 如: 1cm 或 .75
\end{description}
