\begin{titlepage}
  \begin{center}
    \vspace*{1cm}

    \Huge{\textbf{几何中的著名定理}}

    
          
    \vspace{1.5cm}
    \LARGE

    \vfill
          
    \vspace{0.8cm}
  
    \begin{center}
    \begin{tikzpicture}
      \tikzmath{
        \a = 4;
        \b = 3;
      }
      \coordinate (A1) at ({-(\a-\b)/2},{(\a-\b)/2});
      \coordinate (A2) at ({-(\a-\b)/2},{-(\a-\b)/2});
      \coordinate (A3) at ({(\a-\b)/2},{-(\a-\b)/2});
      \coordinate (A4) at ({(\a-\b)/2},{(\a-\b)/2});
      \coordinate (B1) at ({(\a+\b)/2},{(\a-\b)/2});
      \coordinate (B2) at ({-(\a-\b)/2},{(\a+\b)/2});
      \coordinate (B3) at ({-(\a+\b)/2},{-(\a-\b)/2});
      \coordinate (B4) at ({(\a-\b)/2},{-(\a+\b)/2});
      \draw[fill=green,opacity=.75] (A1) -- (B1) -- (B2);
      \draw[fill=magenta,opacity=.75] (A2) -- (B2) -- (B3);
      \draw[fill=cyan,opacity=.75] (A3) -- (B3) -- (B4);
      \draw[fill=red,opacity=.75] (A4) -- (B4) -- (B1);
      \draw[fill=yellow,opacity=.75] (A1) -- (A2) -- (A3) -- (A4);
      % \foreach \p in {A1,A2,A3,A4,B1,B2,B3,B4}
      %   \fill[orange] (\p) circle (2pt);
    \end{tikzpicture}
    \end{center}

    \vspace{1.5cm}
          
    \today
           
  \end{center}
\end{titlepage}