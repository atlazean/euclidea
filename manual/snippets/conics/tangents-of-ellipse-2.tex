\begin{tikzpicture}
  \clip (-6,-6) rectangle (6,6);
  \axes(-5:5, -5:5);

  \coordinate (F1) at (-3,-1);
  \coordinate (F2) at (2,0);
  \coordinate (P1) at (2,1);
  \coordinate (P) at (4,2);

  % 绘制椭圆
  \tikzset{ellipse/define ={F1,F2,P1}, }
  \draw [purple, thick, ellipse];
  
  % 作以 F1 为圆心, 2a 为半径的圆
  \coordinate (A) at 
    ($(F1)+(\pgfkeysvalueof{/tikz/ellipse/angle}:2*\pgfkeysvalueof{/tikz/ellipse/a} cm)$);
  \draw[teal,circle={F1,A}];
  
  % 作以 P 为圆心, PF2 为半径的圆
  \draw[red,circle={P,F2}];
  
  % 求两圆的交点(存储在 cc1 和 cc2 中)
  \tikzset{circle-circle={F1,A,P,F2},}
  \coordinate (Q) at (cc1);
  \coordinate (R) at (cc2);

  % 作出 F2Q, F2R 的中垂线即为椭圆的切线
  \draw[violet,thick,perpendicular bisector={F2, Q}];
  \draw[violet,thick,start modifier=-2cm,perpendicular bisector={F2, R}];

  \foreach \p/\placement in {F1/left,F2/right,
  P/above right,Q/above,R/below right}{
    \fill[red] (\p) circle (2pt);
    \draw (\p) node[\placement] {$\p$};
  }

\end{tikzpicture}