\begin{tikzpicture}
  \axes(-5:5, -5:5);

  \coordinate (F1) at (-3,-1);
  \coordinate (F2) at (2,0);
  \coordinate (P1) at (2,1);
  \coordinate (P) at (4,2);

  % 绘制椭圆
  \tikzset{ellipse/define ={F1,F2,P1}, }
  \draw [purple, thick, ellipse];

  % 获取存储的坐标(椭圆中心)
  \coordinate (C) at (\pgfkeysvalueof{/tikz/ellipse/xcenter} cm, 
    \pgfkeysvalueof{/tikz/ellipse/ycenter} cm);
  
  % 椭圆长轴的顶点
  \coordinate (A) at 
    ($(C)+(\pgfkeysvalueof{/tikz/ellipse/angle}:\pgfkeysvalueof{/tikz/ellipse/a} cm)$);

  % 作椭圆的辅助圆
  \draw[teal,circle={C,A}];
  
  % 作以 PF1 为直径的圆
  \coordinate (M) at ($(P)!.5!(F1)$);
  \draw[red,circle={M,P}];
  
  % 求两圆的交点(存储在 cc1 和 cc2 中)
  \tikzset{circle-circle={C,A,M,P},}
  \coordinate (Q) at (cc1);
  \coordinate (R) at (cc2);

  % 作出切线
  \draw[violet,thick] (P) -- (Q) (P) -- (R);

  \foreach \p/\placement in {F1/left,F2/right,
  P/above right,Q/above left,R/below right}{
    \fill[red] (\p) circle (2pt);
    \draw (\p) node[\placement] {$\p$};
  }

\end{tikzpicture}